\section{Hybrid Retrieval Strategy}
\label{appendix_sec:preliminaries}

A list of $k$ passages by merging returned passages from both a $\mathbf{C} = \left \{c_1, c_2, \ldots, c_C \right \}$ index of textual passages and a $\mathbf{T} = \left \{t_1, t_2, \ldots,t_T: t_j = \left ( s_j, p_j, o_j \right ) \right \}$ index representing a set of triples associated with the passages in $\mathbf{C}$, using Reciprocal Rank Fusion (RRF) \cite{Cormack2009}, can be obtained, as follows:
\begin{align}
h^k_{\text{base}}\left( \mathbf{q}, {\mathbf{C} \cup \mathbf{T}} \right) 
    &= \text{RRF} \Big ( h^k_{\text{base}}\left( \mathbf{q}, {\mathbf{C}}\right), \nonumber \\
    &\quad\quad h^k_{\text{base}}\left( \mathbf{q}, {\mathbf{T}} \right) \Big ),
    \label{eq:triple_passage_index_retrieve}
\end{align}
% we multiply $k$ for the $T$ index by 5.
where $h^k_{\text{base}}\left( \mathbf{q}, {\mathbf{C}}\right)$ and $h^k_{\text{base}}\left( \mathbf{q}, {\mathbf{T}}\right)$ are the passages retrieved from $\mathbf{C}$ and $\mathbf{T}$, after a base retrieval step on each index separately. In this case, each triple $\in h^k_{\text{base}}\left( \mathbf{q}, {\mathbf{T}} \right)$ is mapped to its corresponding passage, ensuring that top-$k$ unique passages are returned after considering the triple scores in $\mathbf{T}$.


\iffalse
Given an input query $\mathbf{q'}$ \pascual{Shouldn't be use q instead of $\mathbf{q'}$?}, a \textit{baseline} retrieval step includes selecting the most relevant passages, using a combination of hybrid retrieval steps. Each hybrid retrieval search step returns top-$k$ items from an index of interest $\mathbf{R} = \left \{r_1, \ldots, r_R \right \}$ s.t. $\mathbf{R} \in \left (\mathbf{C} \cup \mathbf{T} \right ) \setminus \left (\mathbf{C} \cap \mathbf{T} \right )$ \pascual{Isn't it unnecessary to include $(\mathbf{C} \cap \mathbf{T})$? (Almost) by definition there won't be any intersection, although I guess a passage could be just a triple? Note that in the following sections we only use $\mathbf{C} \cup \mathbf{T}$} by aggregating the results of semantic search and $\text{score}_{\text{BM25}}$ using Reciprocal Rank Fusion (RRF) \pascual{Should we explain what the RRF function does, or should be cite a paper that does so?}, as follows: 
\begin{align}
h^k_{\text{hybrid}}\left( \mathbf{q'}, {\mathbf{R}}\right ) = \text{RRF}\left(h^k_{\text{dense}}, h^k_{\text{BM25}}\right),
\end{align}
where $h^k_{\text{dense}} \subseteq \mathbf{R}$ and $h^k_{\text{BM25}}\subseteq \mathbf{R}$ are functions
returning sets of items from $\mathbf{R}$, in descending order according to $\text{score}_{\text{dense}}$ and $\text{score}_{\text{BM25}}$ respectively, s.t. 
$\text{score}_{\text{dense}}  \left( \mathbf{q'}, r_j \right ) \geq\text{score}_{\text{dense}}  \left( \mathbf{q'}, r_{j+1} \right )$ $\forall r_j \in h^k_{\text{dense}}$ and $j \in \left [ 1, k - 1\right ]$ and $\text{score}_{\text{BM25}}  \left( \mathbf{q'}, r_j \right ) \geq\text{score}_{\text{BM25}}  \left( \mathbf{q'}, r_{j+1} \right )$ $\forall r_j \in h^k_{\text{BM25}}$ and $j \in \left [ 1, k - 1\right ]$ \pascual{Maybe we can simplify all this math by just saying the lists are ordered in descending order of their respective scores}.
\fi
