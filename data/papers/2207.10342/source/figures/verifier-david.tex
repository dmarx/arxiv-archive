\begin{figure}[h!]
\centering
\begin{tikzpicture}
  % Define nodes
  \node[obs]           (Q) {$Q$};
  \node[latent, right=0.6cm of Q]         (T1) {$T_1$};
  \node[latent, right=0.6cm of T1]         (T2) {$T_{...}$};
  \node[latent, right=0.6cm of T2]         (T3) {$T_1$};
  \node[latent, right=0.6cm of T3]         (A) {$A$};
  
  \node[obs, above=0.6cm of T1]         (V1) {$V_1$};
  \node[obs, above=0.6cm of T2]         (V2) {$V_{...}$};
  \node[obs, above=0.6cm of T3]         (V3) {$V_3$};

  % Connect the nodes
  \edge {Q} {T1} ; %
  \edge {T1} {T2} ; %
  \edge {T2} {T3} ; %
  \edge {T3} {A} ; %
  \edge {T1} {V1} ; %
  \edge {T2} {V2} ; %
  \edge {T3} {V3} ; %
  \draw [->] (Q) to [out=-40,in=-140] (A);
%   \path[every node/.style={font=\sffamily\small}]
%     (Q) edge[bend right] node [left] {} (A);
%   \edge[bend right=30] {Q} {A} ; %
\end{tikzpicture}
\caption{The current way in which verifiers are diagrammed. Note that every node actually gets as input the concatenated strings of all of its parent nodes.}
\label{fig:verifier}
\end{figure}


