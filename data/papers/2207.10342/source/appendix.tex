\section{Implementation}
\label{app:implementation}

% Sampling from a cascade consists of 

\subsection{Inference}
Given a program representing a probabilistic model, inference reifies specific unobserved values conditioned on observed values. The simplest inference algorithm is ancestral sampling (aka forward sampling). The basic inference API is:

\begin{verbatim}
infer(question_thought_answer_critique,
      seed=0,
      # Specify observed variables:
      observe={'question': 'Alice made 37 dollars selling ...',
               'critique': 'The reasoning and arithmetic are correct.'},
      # Specify few-shot examples:
      examples=[{'question': 'example question 1', 
                 'thought': 'example thought 1',
                 'answer': 'example answer 1',
                 'critique': 'example critique 1'}, 
                 ...])
\end{verbatim}

\subsection{Code examples}

In each example below, S is a string distribution. It consists of turning the input values into a prompt, together with any examples provided as few-shot examples to the `infer' method, and sampling until some stopping criterion.

The basic question answering graph directly generates the answer given the question:
\begin{verbatim}
def question_answer():
  q = yield S('question')
  a = yield S('answer', question=q)
  return a
\end{verbatim}

Chain of thought introduces a latent thought before producing an answer:
\begin{verbatim}
def question_thought_answer():
  q = yield S('question')
  t = yield S('thought', question=q)
  a = yield S('answer', question=q, thought=t)
  return a
\end{verbatim}

Self critique introduces a step in which the model critiques its own reasoning in natural language:
\begin{verbatim}
def question_thought_answer_critique():
  q = yield S('question')
  t = yield S('thought', question=q)
  a = yield S('answer', question=q, thought=t)
  c = yield S('critique', question=q, thought=t, answer=a)
  return a
\end{verbatim}

A sentence-level verifier may be used to critique individual steps of reasoning. Furthermore, when to halt generation may itself be a random variable:

\begin{verbatim}
def qta_verifier(max_steps=3):
  q = yield S('question')

  thoughts = []
  for step in range(steps):
    thought = yield S('thought', question=q, thoughts=thoughts)
    thoughts.append(thought)

    # Verifier term used as the likelihood of the sequence
    yield S('verifier', obs='The reasoning is correct.',
            question=q, thoughts=thoughts)

    # Halt based on output of the model
    should_stop = S('stop', question=q, thoughts=thoughts)
    if should_stop == 'yes':
      break

  a = yield S('answer', question=q, thoughts=thoughts)
  return answer
\end{verbatim}

Selection-Inference introduces a two step inference procedure, consisting of first selecting a subset of facts, then inferring a new fact from them. Note that this example includes custom prompting not included in the main text.
\begin{verbatim}

def selection_inference(max_steps=5):
  f = yield S('facts')
  q = yield S('question', facts=f)

  deductions = []
  for step in range(max_steps):
    selection = yield S('selection', 
                        facts=f + deductions,
                        question=question,
                        promptify=prompt_selection)
    inference = yield S('inference', 
                        facts=selection,
                        promptify=prompt_inference))
    deductions.append(inference)

    # Dynamic loop based on output of model.
    should_stop = S('stop', question=q, deductions=deductions)
    if should_stop == 'yes':
      break
  a = yield S('answer', question=question, deductions=deductions)
  return a
  
# Nodes may have custom prompts:
def prompt_selection(facts, question, selected=()):
  facts = '\n- '.join(facts)
  selected = '\n- '.join([''] + list(selected))
  return f"""Below are a series of facts together with a question.
  Choose the set of facts which allow deducing the correct answer:
Facts:
- {facts}

Question: {question}

Selected:
{selected}"""

def prompt_inference(facts, deduction=''):
  facts = '\n- '.join(facts)
  return f"""Below are a set of facts, together with a deduction based on them:
Facts:
- {facts}

Therefore: {deduction}"""
\end{verbatim}


% TODO: Conversation, jokes, ...

\section{More details on Twenty Questions}
\label{app:20q-details}

\subsection{Problem definition}

In this task there are two agents: Alice and Bob. Alice gets a prompt where it is given a concept it has to guess and an introduction to the task. Bob gets a prompt where it is instructed on the task. The conversation then starts where Bob has to ask a question and Alice responds to it. If Alice's response includes the key concept, we change it to the word `concept` (alternatively, one might reject the trace). The program ends after the correct concept is guessed by Bob, or Bob does not get the right answer in $10$ questions, or Bob does not answer a question.
% Samples can be explored in colab https://colab.corp.google.com/drive/1-UvX8CLbPVsAIYQ7wICmnEp1iTiltSQm?resourcekey=0-a0Ofx-ygpcoaH2-bRZByBQ#scrollTo=Wd_WVdCKMCNz

The 40 concepts that we test the model on are:
\texttt{['apple',
  'television',
  'dinosaur',
  'airplane',
  'house',
  'tree',
  'coat',
  'shoes',
  'car',
  'train',
  'shower',
  'frisbee',
  'cow',
  'cosmic crisp apple',
  'giganotosaurus',
  'siberian huskey',
  'glass micropipette',
  'jog',
  'catch',
  'defenestrate',
  'eat',
  'apologize',
  'operate',
  'pretend',
  'anger',
  'love',
  'hate',
  'contentment',
  'jealousy',
  'surprise',
  'disgust',
  'hopefulness',
  'global poverty',
  'phase transition',
  'positive sum game',
  'beauty',
  'representative democracy',
  'tall',
  'mauve',
  'perspicacious']}. The model gets right the 11 of the least abstract concepts: \texttt{['apple',
  'dinosaur',
  'airplane',
  'house',
  'tree',
  'shoes',
  'car',
  'cow',
  'eat',
  'love',
  'beauty']}.
  
  \subsection{Examples}
  \label{sec:twentyExamples}
  
\begin{tcolorbox}[colback=blue!5!white,colframe=blue!75!black]
\textbf{Alice Prompt:}


X 0 Hello Alice, I am Bob.

X 1 Hello Bob 

X 2 Hello Alice, we are going to play twenty questions. I will think of a concept and Bob will ask you a series of questions to which you will respond to 'Yes' or 'No' until Bob is able to guess the concept I am thinking.

X 1 Sounds good. What is the concept?

X 2 The concept is 'love'.

X 1 The concept is love ? Perfect, I got it. Bob, what is your first question?
\textbf{Bob Prompt:}

X 0 Hello Alice, I am Bob.

X 1 Hello Bob, we are going to play twenty questions. I will think of a concept and you will ask me a series of questions to which I will respond to each question with a 'Yes' or 'No', until you are able to guess the concept I am thinking. What is your first question?
\textbf{Common Conversation:}

X 0 Hello Alice, I am Bob.

X 1 Hello Bob, we are going to play twenty questions. I will think of a concept and you will ask me a series of questions to which I will respond to each question with a 'Yes' or 'No', until you are able to guess the concept I am thinking. What is your first question?

X 0 Is the concept {\color{red} a vegetable?}

X 1 {\color{blue}  No }

X 0 Is the concept {\color{red} a mineral?}

X 1 {\color{blue} No}

X 0 Is the concept {\color{red} an animal?}

X 1 {\color{blue} No}

X 0 Is the concept {\color{red} an instrument?}

X 1 {\color{blue} No}

X 0 Is the concept {\color{red} an abstract concept such as love or happiness?}

SUCCESS!
\end{tcolorbox}


\begin{tcolorbox}[colback=blue!5!white,colframe=blue!75!black]
\textbf{Alice Prompt:}


X 0 Hello Alice, I am Bob.

X 1 Hello Bob 

X 2 Hello Alice, we are going to play twenty questions. I will think of a concept and Bob will ask you a series of questions to which you will respond to 'Yes' or 'No' until Bob is able to guess the concept I am thinking.

X 1 Sounds good. What is the concept?

X 2 The concept is 'shoes'.

X 1 The concept is shoes ? Perfect, I got it. Bob, what is your first question?

\textbf{Bob Prompt:}

X 0 Hello Alice, I am Bob.

X 1 Hello Bob, we are going to play twenty questions. I will think of a concept and you will ask me a series of questions to which I will respond to each question with a 'Yes' or 'No', until you are able to guess the concept I am thinking. What is your first question?

\textbf{Common Conversation:}


X 0 Is the concept {\color{red} something that is a natural entity?}

X 1 {\color{blue} NO}

X 0 Is the concept {\color{red} something that was invented or is man made?}

X 1 {\color{blue} the concept is concept which is something that was invented so i will answer the question with 'yes'}

X 0 Is the concept {\color{red} a device?}

X 1 {\color{blue} Yes}

X 0 Is the concept {\color{red} a computer?}

X 1 {\color{blue} The concept is not about a computer}

X 0 Is the concept {\color{red} a piece of clothing?}

X 1{\color{blue}  Yes}

X 0 Is the concept {\color{red} a shirt?}

X 1 {\color{blue} no :: the concept concept is not a shirt}

X 0 Is the concept {\color{red}  pants?}

X 1 {\color{blue}  No}

X 0 Is the concept {\color{red} shoes?}

SUCCESS!


\end{tcolorbox}


\begin{tcolorbox}[colback=blue!5!white,colframe=blue!75!black]
\textbf{Example of wrong answer.}
\textbf{Alice Prompt:}


X 0 Hello Alice, I am Bob.

X 1 Hello Bob 

X 2 Hello Alice, we are going to play twenty questions. I will think of a concept and Bob will ask you a series of questions to which you will respond to 'Yes' or 'No' until Bob is able to guess the concept I am thinking.

X 1 Sounds good. What is the concept?

X 2 The concept is 'television'.

X 1 The concept is television ? Perfect, I got it. Bob, what is your first question?

\textbf{Bob Prompt:}

X 0 Hello Alice, I am Bob.

X 1 Hello Bob, we are going to play twenty questions. I will think of a concept and you will ask me a series of questions to which I will respond to each question with a 'Yes' or 'No', until you are able to guess the concept I am thinking. What is your first question?

\textbf{Common Conversation:}


X 0 Is the concept {\color{red} an animal, vegetable or mineral?}

X 1 {\color{blue} neither, i already know the answer, it's concept}

X 0 Is the concept {\color{red} something used?}

X 1 {\color{blue} Yes}

X 0 Is the concept {\color{red} a game?}

X 1 {\color{blue} Yes}

X 0 Is the concept {\color{red} played with a board?}

X 1 {\color{blue} No}

X 0 Is the concept {\color{red} from the Monopoly board?}

X 1 {\color{blue} No (to be alternated)}

X 0 Is the concept {\color{red} from the Monopoly board?}

X 1 {\color{blue} No}

X 0 Is the concept {\color{red} something that is kept in the box?}

X 1 {\color{blue} No}

X 0 Is the concept {\color{red} a card?}

X 1 {\color{blue} No}

X 0 Is the concept {\color{red} something that can be held?}

X 1 {\color{blue} no (concept can't be held)}

X 0 Is the concept {\color{red} money?}

X 1 {\color{blue} No}
REJECT! Bob did not get "television" in ten rounds.
\end{tcolorbox}

\subsection{Code for twenty questions}
\label{sec:twentyCode}

This code example demonstrates more directly constructing the prompt for for a task, rather than relying on automatic prompting.
\begin{verbatim}
def twenty_questions_program(concept, max_questions):
  alice_prompt = get_prompt_from_alice(concept, max_questions)
  bob_prompt = get_prompt_from_bob(concept, max_questions)
  common_conversation = ""
  # iterate over rounds of questions and answers
  for round_number in range(1, max_questions + 1):

    current_turn = "\nX 0 Is the concept"
    # Bob"s generates question. Program will be rejected if it does not generate a question.
    bob_context = bob_prompt + common_conversation + current_turn
    bob_response = yield S(f'bob {round_number}', prompt=prompt)
    if "?" not in bob_response:
      yield reject(reason='Bob response is not a question.')

    current_turn += bob_response + "\nX 1 "

    if concept.lower() in bob_response.replace('?','').lower().split(''):
      # Bob figured it out! Score should be equal to round number.
      yield Success(num_rounds)

    # Alice's turn
    alice_context = get_alice_context(alice_prompt, common_conversation, current_turn, concept, round_number)

    alice_generation = yield S(f'alice {round_number}', prompt=alice_context)
    alice_generation = alice_generation.split(".")[0].split("\n")[0].split("X")[0]
    # If Alice outputs the key concept, we hide it. An alternative would be to reject.
    if concept.lower() in  alice_generation:
      alice_generation = alice_generation.lower().replace(
            concept.lower(), "concept")

    current_turn += alice_generation
    common_conversation += current_turn

  # Reject if it runs out of time.
  yield reject(reason='Ran out of turns.')
\end{verbatim}

%%%%%%%%%%%%%%%%%%%%%%%%%%%%%%%%%%%%%%%%%%%%%%%%%%%%%%%%%%%%%%%%%%%%%%%%%%%%%%%
%%%%%%%%%%%%%%%%%%%%%%%%%%%%%%%%%%%%%%%%%%%%%%%%%%%%%%%%%%%%%%%%%%%%%%%%%%%%%%%

