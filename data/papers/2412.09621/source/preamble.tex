

\usepackage{overpic}
\usepackage{enumitem} %
\usepackage{overpic} %
\usepackage{color}
\usepackage{xspace}
\usepackage{shortbold}
\usepackage{verbatim}


\definecolor{turquoise}{cmyk}{0.65,0,0.1,0.3}
\definecolor{purple}{rgb}{0.65,0,0.65}
\definecolor{dark_green}{rgb}{0, 0.5, 0}
\definecolor{red}{rgb}{0.8, 0.2, 0.2}
\definecolor{darkred}{rgb}{0.6, 0.1, 0.05}
\definecolor{blueish}{rgb}{0.0, 0.3, .6}
\definecolor{light_gray}{rgb}{0.7, 0.7, .7}
\definecolor{pink}{rgb}{1, 0, 1}
\definecolor{greyblue}{rgb}{0.25, 0.25, 1}
\definecolor{msorange}{rgb}{0.93,0.49,0.19}
\definecolor{msblue}{rgb}{0.33,0.56,0.76}

\definecolor{darkyellow}{rgb}{0.796,0.901,0}
\newcommand{\duster}{DUSt3R\xspace}
\newcommand{\method}{DynaDUSt3R\xspace}
\newcommand{\dataset}{Stereo4D\xspace}
\newcommand{\monster}{MonST3R\xspace}
\newcommand{\todo}[1]{{\color{red}#1}}
\newcommand{\TODO}[1]{\textbf{\color{red}[TODO: #1]}}
\newcommand{\bfpar}[1]{{\vspace{1.2mm} \par \noindent \bf{{#1}}}}
\newcommand{\itpar}[1]{{\vspace{1.2mm} \par \noindent \it{{#1}}}}
\newcommand{\parnobf}[1]{\bfpar{#1.}}
\newcommand{\parnoit}[1]{\itpar{#1}}

\newcommand{\at}[1]{{\color{blueish}#1}} %
\newcommand{\AT}[1]{{\color{blueish}{\bf [AT: #1]}}} %
\newcommand{\At}[1]{\marginpar{\tiny{\textcolor{blueish}{#1}}}} %
\newcommand{\gh}[1]{{\color{dark_green}#1}}
\newcommand{\Gh}[1]{\marginpar{\tiny{\textcolor{dark_green}{#1}}}}
\newcommand{\GH}[1]{{\color{dark_green}{\bf [AT: #1]}}}
\newcommand{\dfnote}[1]{{\bf{\textcolor{red}{DF: #1}}}}
\newcommand{\ahnote}[1]{{\bf{\textcolor{red}{AH: #1}}}}
\newcommand{\ljnote}[1]{{\bf{\textcolor{blue}{LJ: #1}}}}
\newcommand{\nsnote}[1]{\textcolor{magenta}{[\textbf{NS:} #1]}}


\definecolor{BlueCam}{rgb}{0.10196, 0.52157, 1.0}
\definecolor{GreenCam}{rgb}{0, 0.5, 0}
\definecolor{PurpleCam}{rgb}{0.74,0.415,0.831}

\newcommand{\CIRCLE}[1]{\raisebox{.5pt}{\footnotesize \textcircled{\raisebox{-.6pt}{#1}}}}

\DeclareMathOperator*{\argmin}{arg\,min}
\DeclareMathOperator*{\argmax}{arg\,max}
\newcommand{\loss}[1]{\mathcal{L}_\mathsf{#1}}
\newcommand{\expect}{\mathbb{E}}
\newcommand{\real}{\mathbb{R}}

\newcommand{\Fig}[1]{Fig.~\ref{fig:#1}}
\newcommand{\Figure}[1]{Figure~\ref{fig:#1}}
\newcommand{\Tab}[1]{Tab.~\ref{tab:#1}}
\newcommand{\Table}[1]{Table~\ref{tab:#1}}
\newcommand{\eq}[1]{(\ref{eq:#1})}
\newcommand{\Eq}[1]{Eq.~\ref{eq:#1}}
\newcommand{\Equation}[1]{Equation~\ref{eq:#1}}
\newcommand{\Sec}[1]{Sec.~\ref{sec:#1}}
\newcommand{\Section}[1]{Section~\ref{sec:#1}}
\newcommand{\Appendix}[1]{Appendix~\ref{app:#1}}

\usepackage{blindtext}
\newcommand{\lorem}[1]{\todo{\blindtext[#1]}}

\renewcommand{\paragraph}[1]{\vspace{1em}\noindent\textbf{#1}.}


\makeatletter
\DeclareRobustCommand\onedot{\futurelet\@let@token\@onedot}
\def\@onedot{\ifx\@let@token.\else.\null\fi\xspace}

\def\eg{\emph{e.g}\onedot} \def\Eg{\emph{E.g}\onedot}
\def\ie{\emph{i.e}\onedot} \def\Ie{\emph{I.e}\onedot}
\def\cf{\emph{cf}\onedot} \def\Cf{\emph{Cf}\onedot}
\def\etc{\emph{etc}\onedot} \def\vs{\emph{vs}\onedot}
\def\wrt{w.r.t\onedot} \def\dof{d.o.f\onedot}
\def\iid{i.i.d\onedot} \def\wolog{w.l.o.g\onedot}
\def\etal{\emph{et al}\onedot}
\makeatother


\newcommand\blfootnote[1]{ %
  \begingroup
  \renewcommand\thefootnote{}\footnote{#1} %
  \addtocounter{footnote}{-1} %
  \endgroup
}

