\section{Conclusion}
We proved and demonstrated the counter-intuitive phenomenon that distortion and perceptual quality are at odds with each other. Namely, the lower the distortion of an algorithm, the more its distribution must deviate from the statistics of natural scenes. We showed empirically that this tradeoff exists for many popular distortion measures, including those considered to be well-correlated with human perception. Therefore, any distortion measure alone, is unsuitable for assessing image restoration methods. Our novel methodology utilizes a pair of NR and FR metrics to place each algorithm on the perception-distortion plane, facilitating a more informative comparison of image restoration methods.

\vspace{0.2cm}
\noindent \textbf{Acknowledgements\ \ \ }
%
This research was supported by the Israel Science Foundation (grant no.~852/17), and by the Technion Ollendorff Minerva Center.
