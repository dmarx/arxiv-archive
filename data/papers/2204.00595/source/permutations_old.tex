
\newcommand{\baseb}[3]{\parens{{#1},{#2}}_{{#3}}}
\newcommand{\mx}[1]{\mathbf{#1}}
\newcommand{\floors}[1]{\left \lfloor #1 \right \rfloor}
\newcommand{\parens}[1]{\left( {#1}\right)}

\section{Converting $\vL$ to $\vR$ matrices via simple permutations}
\label{sec:permutation}
\subsection{General square matrices}


Let $1\le b\le n$ be integers such that $b$ divides $n$.

We will need to express each index $0\le i<n$ in `block-form'. More specifically,


\begin{definition}\label{def:$i$}
Let $0\le i<n$ and let $b \ge 1$ be an integer that divides $n$. Then define
\[i_0=i\mod{b},\]
and
\[i_1=\floors{\frac in}.\] 
We use the notation $i\equiv\baseb{i_1}{i_0}{b}$ to denote the representation of  above. In particular, if $i\equiv(i_1,i_0)_{b}$,
then we have
\[
        i = i_1 \cdot b + i_0
\]
\end{definition}


\begin{definition}[Square $\vR$] For $0\le i< \frac {n}{b}$, let $\mx{R}_{i}\in\R^{b \times b }$ be a $b \times b $ ``block" matrix. Then define the matrix $\vR$ with {\em block size} $b$ as follows:
\begin{equation}
 \label{eq:def-R}
  \vR = \diag\left(\vR_0, \dots, \vR_{\frac {n}{b}-1}\right).
\end{equation}
\end{definition}
Note that the number of possible non-zero values in $\vR$ is $\frac {n}{b}\cdot b^2=nb$ locations.


We re-state the above formulation equivalently as:
\begin{proposition}

\label{prop:R-eqv-def}
$\vR$ satisfies~\Cref{eq:def-R} if and only if the follow holds for any
$0\le i,j< n$. Let $i\equiv\baseb{i_1}{i_0}{b}$ and $j\equiv\baseb{j_1}{j_0}{b}$.  Then
\begin{enumerate}
    \item\label{item:zero-loc-R} if $i_1\ne j_1$, then $\vR[i,j]=0$. 
    \item \label{item:non-zero-loc-R} Else (i.e. if $i_1=j_1$), then $\vR[i,j]=\vR_{i_1}[i_0,j_0]$.
\end{enumerate}

\end{proposition}



\begin{definition}\label{def:Matrix L}
Square matrix $\vL$ with {\em block size} $b$ will have the form: 
    \begin{equation}
        	\label{eq:def-L}
    \vL=
    	\begin{bmatrix}
    		\mx{D}_{0,0} & \dots & \mx{D}_{0,\frac{n}{b} -1} \\
    		\vdots & \ddots & \vdots \\
    		\mx{D}_{\frac{n}{b} -1,0} & \dots & \mx{D}_{\frac{n}{b} -1,\frac{n}{b} -1}
    	\end{bmatrix}
    \end{equation}
    where each $\mx{D}_{i,j}\in\R^{b\times b}$ is a $b \times b$ diagonal matrix.
\end{definition}
Note that the number of possible non-zero values in $\vL$ is $\parens{\frac nb}^2\cdot b=\frac{n^2}b$ locations.



We re-state the above formulation equivalently as:
\begin{proposition}
\label{prop:L-eqv-def}
$\vL$ satisfies~\Cref{eq:def-L} if and only if the follow holds for any
$0\le i,j< n$. Let $i\equiv\baseb{i_1}{i_0}{b}$ and $j\equiv\baseb{j_1}{j_0}{b}$. Then 
\begin{enumerate}
    \item\label{item:zero-loc-L} if $i_0\ne j_0$, then $\vL[i,j]=0$. 
    \item \label{item:non-zero-loc-L} Else, (i.e. if $i_0=j_0$), then $\vL[i,j]=\vD_{i_1,j_1}[i_0,i_0]$.
\end{enumerate}
\end{proposition}


We need the definition of a particular class of permutations:
\begin{definition}
\label{def:sigma-b}
Let $1\le b\le n$ such that $b$ divides $n$.  Let $i\equiv\baseb{i_1}{i_0}{b}$. Define
    \begin{equation}
            \label{eq:sigma_b-def}
        \sigma_b(i) = i_0\cdot\frac{n}{b} + i_1.
    \end{equation}
That is, $\sigma_{b}(i)\equiv \baseb{i_0}{i_1}{\frac {n}{b}}$.
\end{definition}


We will argue the following:
\begin{theorem} Let $1\le b\le n$ such that $b$ divides $n$.
Let $\vP_b$ be the permutation matrix defined by the permutation $\sigma_b$. Let $\vL$ be as in~\Cref{eq:def-L}. Then we have
\[\vR'=\vP_b\vL\vP_{\frac nb},\]
where $\vR'$ has the same form as in~\Cref{eq:def-R}  with block size $\frac nb$.
\end{theorem}

\begin{proof}
We first note that multiplying an $n\times n$ matrix on the right (and left resp.) by $\vP_{\frac nb}$ (and $\vP_b$ resp.) permutues the columns (and rows resp.) of the matrix according to $\sigma_b$.\footnote{This uses the fact that $\parens{\sigma_b}^{-1}=\sigma_{\frac nb}$.} This implies that for any $0\le i,j<n$:
\begin{equation}
\label{eq:L-permuted}
\vR'[\sigma_b(i),\sigma_b(j)]=\vL[i,j].
\end{equation}
To complete the proof, we will argue that $\vR'$ satisfies the two conditions in~\Cref{prop:R-eqv-def}.

Towards this end, let $0\le i,j<n$ be arbitrary indices and further, define $i=\baseb{i_1}{i_0}{b}$ and $j=\baseb{j_1}{j_0}{b}$. Then note that $\sigma_b(i)=\baseb{i_0}{i_1}{\frac nb}$ and $\sigma_b(j)=\baseb{j_0}{j_1}{\frac nb}$.

By~\Cref{prop:L-eqv-def}, we have that if $i_0\ne j_0$, then $\vL[i,j]=0$. Note that $i_0\ne j_0$ satisfies the pre-condition for base size $\frac nb$ for indices $(\sigma_b(i),\sigma_b(j))$ in item~\ref{item:zero-loc-R} in~\Cref{prop:R-eqv-def}.  Then   by~\cref{eq:L-permuted}, we have that $\vR'[\sigma_b(i),\sigma_b(j)]=0$, which satisfies item~\ref{item:zero-loc-R} in~\Cref{prop:R-eqv-def}.

Now consider the case that $i_0=j_0$, then by item~\ref{item:non-zero-loc-L} in~\Cref{prop:L-eqv-def}, we have that $\vL[i,j]=\vD_{i_1,j_1}[i_0,i_0]$.  Note that $i_0= j_0$ satisfies the pre-condition for base size $\frac nb$ for indices $(\sigma_b(i),\sigma_b(j))$ in item~\ref{item:non-zero-loc-R} in~\Cref{prop:R-eqv-def} if we define $\vR'_{i_0}\in\R^{\frac nb\times\frac nb}$ as follows:
\[\vR'_{i_0}[i_1,j_1]=\vD_{i_1,j_1}[i_0,i_0].\] 
Note that the above implies that 
\[\vR'=\diag\parens{\vR'_0,\dots,\vR'_{b-1}},\]
where $\vR'_{\cdot}$ is as defined in the above paragraph.
\end{proof}



\subsection{General rectangular matrices}
For the rest of the section, we will assume that $n_1, n_2, n_3, b_1, b_2 , b_3 \ge 1$ are integers such that 
\begin{itemize}
\item $b_i$ divides $n_i$ for all $1\le i\le 3$) and 
\item $\frac{n_1}{b_1} = \frac{n_2}{b_2}$
\end{itemize}

\begin{definition}\label{def:$i$-rect}
Let $0\le i<n$ and let $b \ge 1$ be an integer that divides $n$. Then define
\[i_0=i\mod{b},\]
and
\[i_1=\floors{\frac in}.\] 
We use the notation $i\equiv (i_1,i_0)_{b,n}$ to denote the representation of  above. In particular, if $i\equiv(i_1,i_0)_{b,n}$,
then we have
\[
        i = i_1 \cdot b + i_0
\]
\end{definition}

We begin with the definition of a rectangular $\vR$:
\begin{definition}
Define $\vR\in\R^{n_2\times n_1}$ with {\em block size} $b_2\times b_1$ as follows. For $0\le i< \frac{n}{b_1}$, let $\vR_{i}\in\R^{b_2 \times b_1}$ be a $b_2 \times b_1$. Then define the matrix $\vR$ with {\em block size} $b_2 \times b_1$ as follows:
\begin{equation}
 \label{eq:def-rect-R}
  \vR = \diag\left(\vR_0, \dots, \vR_{\frac {n_1}{b_1}-1}\right).
\end{equation}
\end{definition}
Recall that we have assumed $\frac {n_1}{b_1}=\frac{n_2}{n_2}$, so~\cref{eq:def-rect-R} is well defined.
Note that the number of possible non-zero values in rectangular $\vR$ is $\frac {n_1}{b_1}\cdot b_1 \times b_2 =n_1b_2$ locations.



We re-state the above definition equivalently as:
\begin{proposition} \label{prop:rect-R-eqv-def} 
$\vR\in\R^{n_2\times n_1}$ satisfies~\Cref{eq:def-rect-R} (with $\frac {n_1}{b_1}=\frac{n_2}{n_2}$) if and only if the following holds for any
$0\le i < n_2$ and $0\le j< n_1$. Let $i\equiv\baseb{i_1}{i_0}{b_2,n_2}$ and $j\equiv\baseb{j_1}{j_0}{b_1,n_1}$.  Then
\begin{enumerate}
    \item\label{item:rect-zero-loc-R} if $i_1\ne j_1$, then $\vR[i,j]=0$. 
    \item \label{item:rect-non-zero-loc-R} Else (i.e. if $i_1=j_1$), then $\vR[i,j]=\vR_{i_1}[i_0,j_0]$.
\end{enumerate}

\end{proposition}

Before we define the `rectangular' $\vL$, we first need to define the notion of a `wrapped diagonal' matrix:
\begin{definition} 
\label{def:wrapped-diag}
A {\em wrapped diagonal} matrix $\mx{S}_{i,j}\in\R^{b_3\times b_2}$ is defined as follows. Assume $b_2\le b_3$. Then for any $0\le i<b_3$ and $0\le j<b_2$, we have the following. If $i\mod{b_2}\ne j$, then $\vS[i,j]=0$. If $b_2>b_3$, then apply the previous definition to $\vS^{\top}$.

\end{definition}


We are now read to define the notion of rectangular $\vL$"
\begin{definition}\label{def:rect-Matrix L}
Rectangular matrix $\vL\in\R^{n_3\times n_2}$ with {\em block size} $b_3 \times b_2$ will have the form: 
    \begin{equation}
        	\label{eq:rect-def-L}
    \vL=
    	\begin{bmatrix}
    		\mx{S}_{0,0} & \dots & \mx{S}_{0,\frac{n_2}{b_2} -1} \\
    		\vdots & \ddots & \vdots \\
    		\mx{S}_{\frac{n_3}{b_3} -1,0} & \dots & \mx{S}_{\frac{n_3}{b_3} -1,\frac{n_2}{b_2} -1}
    	\end{bmatrix},
    \end{equation}
    
where each $\vS_{\cdot,\cdot}$ is a wrapped diagonal matrix (as in~\Cref{def:wrapped-diag}).
\end{definition}
Note that the number of possible non-zero values in $\vL$ when $b_2 > b_3$ is $\parens{\frac {n_2}{b_2}\cdot \frac{n_3}{b_3}} \max(b_2,b_3)=\frac{n_2 \cdot n_3}{\min(b_2,b_3)}$ locations. 


We re-state the above definition equivalently as:


\begin{proposition}
\label{prop:rect-L-eqv-def}
$\vL\in\R^{n_3\times n_2}$ satisfies~\Cref{eq:rect-def-L} if and only if the following holds for any
$0\le i < n_3$ and $0 \le j< n_2$. Let $i\equiv\baseb{i_1}{i_0}{b_3,n_3}$ and $j\equiv\baseb{j_1}{j_0}{b_2,n_2}$.  Assuming $b_2 \le b_3$, we have:
\begin{enumerate}
    \item\label{item:rect-zero-loc-L} if $i_0\mod{b_2}\ne j_0$, then $\vL[i,j]=0$. 
    \item \label{item:rect-non-zero-loc-L} Else, (i.e. if $i_0\mod{b_2}=j_0$), then $\vL[i,j]=\vS_{i_1,j_1}[i_0,j_0]$.
\end{enumerate}
If $b_2>b_3$, then in the above, the condition $i_0\mod{b_2}\ne j_0$ gets replaced by $j_0\mod{b_2}\ne i_0$.
\end{proposition}

We need the definition of a particular class of permutations:
\begin{definition}
\label{def:rect-sigma-b}
Let $1\le b\le n$ such that $b$ divides $n$.  Let $i\equiv\baseb{i_1}{i_0}{b,n}$. Define
    \begin{equation}
            \label{eq:rect-sigma_b-def}
        \sigma_{b,n}(i) = i_0\cdot\frac{n}{b} + i_1.
    \end{equation}
That is, $\sigma_{b,n}(i)\equiv \baseb{i_0}{i_1}{\frac {n}{b} , n}$. Further, we define the permutation matrix corresponding to $\sigma_{b,n}$ by $\vP_{b,n}$.
\end{definition}

We are now ready to prove our main result in this section, which essentially follows from the observation that if we permute the rows and columns of $\vL$ such that the row/column block size in $\vL$ becomes the number of row/columns blocks in the permuted matrix (and vice-versa) then the permuted matrix has the form of $\vR$.


\begin{theorem} Let $1\le b,n_2,n_3$ be such that $b$ divides $n_2$ and $n_3$.
Let $\vL\in\R^{n_3\times n_2}$ be as in~\Cref{eq:rect-def-L} with 
\[b_2=b_3=b.\] 
Then we have
\[\vR'=\vP_{b_3,n_3}\cdot\vL\cdot\vP_{\frac {n_2}{b_2},n_2},\]
where $\vR'$ has the same form as in~\Cref{eq:def-rect-R} with block size $\frac{n_3}{b_3} \times \frac{n_2}{b_2}$.
\end{theorem}


\begin{proof}
In our proof, we will use $b_2$ and $b_3$ as separate variables (even though they are both $b$) for notational consistency with definition of $\vL$.

We recall  that multiplying an $m\times n$ matrix on the right (and left resp.) by $\vP_{\frac nb,n}$ (and $\vP_{b,m}$ resp.) permutes the columns (and rows resp.) of the matrix according to $\sigma_{b,n}$ (and $\sigma_{b,m}$) respectively.\footnote{This uses the fact that $\parens{\sigma_{b,n}}^{-1}=\sigma_{\frac nb,n}$.} This implies that for any $0\le i,j<n$:
\begin{equation}
\label{eq:rect-L-permuted}
\vR'[\sigma_{b_3,n_3}(i),\sigma_{b_2,n_2}(j)]=\vL[i,j].
\end{equation}


Recall that we have $b_2=b_3=b$, so in the definition of $\vL$  we are in the $b_2 \le b_3$ case.
To complete the proof, we will argue that $\vR'$ satisfies the two conditions in~\Cref{prop:rect-R-eqv-def}.\footnote{Note that we also need that the ratios of the row/colomn length to the row/column block sizes are the same. I.e., in our case we need that $\frac{n_3}{\frac{n_3}{b_3}}=\frac{n_2}{\frac{n_2}{b_2}}$, which is true by our assumption that $b_2=b_3$.}

Towards this end, let $0\le i,j<n$ be arbitrary indices and further, define $i=\baseb{i_1}{i_0}{b_3,n_3}$ and $j=\baseb{j_1}{j_0}{b_2,n_2}$. Then note that $\sigma_{b_3,n_3}(i)=\baseb{i_0}{i_1}{\frac {n_3}{b_3},n_3}$ and $\sigma_{b_2,n_2}(j)=\baseb{j_0}{j_1}{\frac {n_2}{b_2},n_2}$.

By~\Cref{prop:rect-L-eqv-def}, we have that if $i_0\mod{b_2}\ne j_0$, then $\vL[i,j]=0$. Note  that since $b_2=b_3$ and $i_0,j_0<b_2$, the condition $i_0\mod{b_2}\ne j_0$ is equivalent to saying $i_0\ne j_0$. Note that $i_0\ne j_0$ satisfies the pre-condition for base size $\frac {n_3}{b_3}\times \frac{n_2}{b_2}$ for indices $(\sigma_{b_3,n_3}(i),\sigma_{b_2,n_2}(j))$ in item~\ref{item:rect-zero-loc-R} in~\Cref{prop:rect-R-eqv-def}.  Then   by~\cref{eq:rect-L-permuted}, we have that $\vR'[\sigma_{b_3,n_3}(i),\sigma_{b_2,n_2}(j)]=0$, which satisfies item~\ref{item:rect-zero-loc-R} in~\Cref{prop:rect-R-eqv-def}.

Now consider the case that $i_0=j\mod b_2$, which by the observation in the above paragraph is the same as $i_0=j_0$. Then by item~\ref{item:rect-non-zero-loc-L} in~\Cref{prop:rect-L-eqv-def}, we have that $\vL[i,j]=\vS_{i_1,j_1}[i_0,j_0]$.  Note that $i_0= j_0$ satisfies the pre-condition for base size $\frac{n_3}{b_3}\times \frac{n_2}{b_2}$ for indices $(\sigma_b(i),\sigma_b(j))$ in item~\ref{item:rect-non-zero-loc-R} in~\Cref{prop:rect-R-eqv-def} if we define $\vR'_{i_0}\in\R^{\frac{n_3}{b_3}\times\frac{n_2}{b_2}}$ as follows:
\[\vR'_{i_0}[i_1,j_1]=\vS_{i_1,j_1}[i_0,j_0].\] 

Note that the above implies that 
\[\vR'=\diag\parens{\vR'_0,\dots,\vR'_{b-1}},\]
where $\vR'_{\cdot}$ is as defined in the above paragraph (and recall $b=b_2=b_3$.
\end{proof}