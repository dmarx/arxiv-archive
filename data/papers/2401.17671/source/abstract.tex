Recent advancements in artificial intelligence have sparked interest in the parallels between large language models (LLMs) and human neural processing, particularly in language comprehension. While prior research has established similarities in the representation of LLMs and the brain, the underlying computational principles that cause this convergence, especially in the context of evolving LLMs, remain elusive. Here, we examined a diverse selection of high-performance LLMs with similar parameter sizes to investigate the factors contributing to their alignment with the brain's language processing mechanisms. We find that as LLMs achieve higher performance on benchmark tasks, they not only become more brain-like as measured by higher performance when predicting neural responses from LLM embeddings, but also their hierarchical feature extraction pathways map more closely onto the brain’s while using fewer layers to do the same encoding. We also compare the feature extraction pathways of the LLMs to each other and identify new ways in which high-performing models have converged toward similar hierarchical processing mechanisms. Finally, we show the importance of contextual information in improving model performance and brain similarity. Our findings reveal the converging aspects of language processing in the brain and large language models and offer new directions for developing LLMs that align more closely with human cognitive processing. 