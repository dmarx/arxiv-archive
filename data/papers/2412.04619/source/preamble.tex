%%%%%%%%%%%%%% ADDITIONAL PACKAGES %%%%%%%%%%%%%%
\usepackage{microtype}
\usepackage{graphicx}
\usepackage{subfigure}
\usepackage{booktabs}
\usepackage[dvipsnames]{xcolor}
\usepackage{caption}
\usepackage{subcaption}
\usepackage{wrapfig}
\usepackage{graphics}
\usepackage{enumitem}
\usepackage{amsmath}
\usepackage{amssymb}
\usepackage{mathtools}
\usepackage{amsthm}
\usepackage[skip=10pt plus1pt, indent=0pt]{parskip}
\let\svthefootnote\thefootnote
\newcommand\freefootnote[1]{%
  \let\thefootnote\relax%
  \footnotetext{#1}%
  \let\thefootnote\svthefootnote%
}
\usepackage{multirow}
\usepackage{algorithm}
\usepackage{algpseudocode}
\usepackage{duckuments}
\usepackage{pifont}% http://ctan.org/pkg/pifont


\makeatletter
\let\blx@rerun@biber\relax
\makeatother
%%%%%%%%%%%%%%%%%%%%%%%%%%%%%%%%
% ADDITIONAL COMMANDS 
%%%%%%%%%%%%%%%%%%%%%%%%%%%%%%%%
% \newcommand{\DDD}{$\textup{D}^3$\xspace}
\providecommand{\abs}[1]{\lvert#1\rvert}
\providecommand{\norm}[1]{\lVert#1\rVert}
\providecommand{\inprod}[1]{\langle#1\rangle}
\providecommand{\set}[1]{\{#1\}}
\providecommand{\seq}[1]{<#1>}
\providecommand{\bydef}{\overset{\text{def}}{=}}
\DeclareMathOperator*{\argmax}{\arg\!\max}
\DeclareMathOperator*{\argmin}{\arg\!\min}

\providecommand{\R}{\ensuremath{\mathbb{R}}}
\providecommand{\vx}{\vec{x}}
\providecommand{\vy}{\vec{y}}
\providecommand{\vz}{\vec{z}}
\providecommand{\vm}{\vec{m}}
\providecommand{\vw}{\vec{w}}
\providecommand{\vs}{\vec{s}}
\providecommand{\vW}{\vec{W}}
\providecommand{\vw}{\vec{w}}
\providecommand{\vr}{\vec{r}}
\providecommand{\vJ}{\vec{J}}
\providecommand{\vX}{\vec{X}}


\providecommand{\mA}{\mat{A}} 

\renewcommand{\vec}[1]{\ensuremath{\boldsymbol{#1}}}
\providecommand{\mat}[1]{\ensuremath{\boldsymbol{#1}}}

\providecommand{\vsig}{\vec{\sigma}}
\providecommand{\s}{\sigma}
% \DeclareMathOperator*{\argmax}{\arg\!\max}
% \DeclareMathOperator*{\argmin}{\arg\!\min}
\newcommand{\explain}[2]{\overset{\text{\tiny{#1}}}{#2}} %for i.i.d etc
\renewcommand{\P}{\mathbb{P}} %% for probabilities
\newcommand{\EX}{\mathbb{E}} %% for expectations

\newcommand{\gauss}[2]{\mathcal{N}\left( #1,#2 \right)} %% for Gaussian distribution
\providecommand{\loss}{\mathcal{L}}
\newcommand*\diff{\mathop{}\!\mathrm{d}} %for dx
% \newcommand{\Tr}{\mathrm{Tr}}
\newcommand*{\vertbar}{\rule[-1ex]{0.5pt}{2.5ex}}
\newcommand*{\horzbar}{\rule[.5ex]{2.5ex}{0.5pt}}


\newcommand{\errbar}[2]{#1 {\scriptsize (#2)}}
%\newcommand{\errbar}[2]{$#1 \pm #2$}} % Recovers previous formar 

\newcommand{\cmark}{\ding{51}}%
\newcommand{\xmark}{\ding{55}}%
%%%%%%%%%%%%%%%%%%%%%%%%%%%%%%%%
% COMMENTS SECTION 
%%%%%%%%%%%%%%%%%%%%%%%%%%%%%%%%
\newcommand\para[1]{\textcolor{RoyalBlue}{(\textbf{#1})}}
