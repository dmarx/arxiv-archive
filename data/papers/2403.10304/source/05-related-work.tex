\section{Related Work}%
\label{sec:related-work}


There are three classes of work related to KIF\@.


%\bigskip


First, there are the ontology-based data access (OBDA) systems~\cite{Xiao-G-2018}.
%%
These are systems such as Ontop~\cite{Xiao-G-2020} which, given a mapping between a data source and target ontology, are able to evaluate over the data source queries written in terms of the ontology.
%%
In ODBA, the data source is a relational database, the ontology is written in a DL-Lite~\cite{Calvanese-D-2007} language (e.g., OWL~2 QL), the query language is SPARQL, and query evaluation is done under the SPARQL~1.1 entailment regime~\cite{Xiao-G-2018}.
%%
Also, OBDA usually means virtual access: the SPARQL query is transformed into an SQL query on-the-fly and evaluated over the database.
%%
(To be feasible, the SQL query needs to be heavily optimized, as the transformation from SPARQL often leads to a blow-up in size~\cite{Gu-Z-2022}.)


Different from OBDA systems, KIF fixes the target syntax and (possibly) the vocabulary to those of Wikidata.
%%
Although it doesn't attempt to do any kind of reasoning on its own, its SPARQL store can be used seamlessly with any reasoning-enabled endpoint.
%%
Also, KIF doesn't attempt to provide an interface for arbitrarily complex queries.
%%
% Instead, it provides a pattern language defined in terms of Wikidata's data model which inherits its restrictions.
%%
The \code{filter()} method, in particular, was inspired by the work on linked data fragments~\cite{Verborgh-R-2014} and TPF~\cite{Verborgh-R-2016}.
%%
It is a lightweight filtering interface which can be used reliably by applications.


Another thing that distinguishes the KIF from OBDA systems is that the latter largely ignore the problem of query federation~\cite{Gu-Z-2022}.
%%
(One notable exception is Squerall~\cite{Mami-M-N-2019}.)
%%
That said, OBDA systems such as Ontop can be used to enable KIF to interface with relational databases, as illustrated in Section~\ref{sec:use-case-and-evaluation}.


%\bigskip


The second class of work related to KIF are RDF integration systems based on SPARQL query-rewriting~\cite{Correndo-G-2010,Makris-K-2012}.
%%
These are similar to OBDA systems but target RDF\@.
%%
The system is given an RDF source, a target ontology, a mapping between the source schema and the target ontology, and a SPARQL query written in terms of the ontology.
%%
Its job is to translate the original SPARQL query into a new SPARQL query and evaluated it over the RDF source on-the-fly.


The problem of SPARQL-rewriting (or ontology-mediated query translation) is closely related to the ontology matching problem.
%%
But despite the vast literature on ontology matching~\cite{Osman-I-2021}, there is little research on using the produced mappings for querying RDF sources, especially when federations of sources are considered~\cite{Cheng-S-2024}.
%%
An early work that uses SPARQL rewriting for integrating multiple RDF sources is~\cite{Makris-K-2012}.
%%
However, it doesn't cover the result reconciliation problem, i.e., using the mappings to reconcile the results of the queries in the terms of the target ontology.
%%
A more recent proposal which covers query-translation and result reconciliation is~\cite{Cheng-S-2024}.
%%
The previous remarks on the differences between KIF and OBDA systems also apply to SPARQL rewriting systems.


%\bigskip


The third class of related work are knowledge graph construction (KGC) systems.
%%
These are systems like SPARQL-Generate~\cite{Lefrancois-M-2017} and SPARQL Anything~\cite{Asprino-L-2023}, which use SPARQL to describe the transformation of non-RDF sources into an RDF dataset.
%%
For this purpose, SPARQL-Generate extends the syntax of SPARQL, while SPARQL Anything overrides the SERVICE clause.
%%
Both of these systems allow users to query non-RDF resources on-the-fly.
%%
But different from KIF, they don't attempt perform any kind of ontology-mediated mapping.
%%
The user must use the vocabulary of each source and specify the desired transformations explicitly, for each query.


%\bigskip


One problem largely ignored by the three classes of work cited above is the representation and tracking of provenance.
%%
In contrast, KIF tackles this problem from the start: statements carry provenance information which is preserved while they traverse the framework.


% LocalWords:  KIF OBDA Ontop ODBA DL QL TPF Squerall KGC



%%% Local Variables:
%%% mode: latex
%%% TeX-engine: xetex
%%% TeX-master: "main"
%%% eval: (visual-line-mode 1)
%%% End:
