\secspace
\section{Background}
\label{sec:back}

Here, we first summarize Glaze and then IMPRESS purification method. 

\para{Glaze protection against style mimicry. } Glaze seeks to protect artist's 
artwork from AI mimicry by adding small 
perturbations on these artwork to confuse diffusion models. 
Given an artwork $x$ and target style $T$ that is different from the artist's, 
Glaze first uses a pretrained style transfer model $\Omega$ to compute a style 
transferred version of the artwork. We denote such image as $\Omega(x, T)$. Then, Glaze 
computes a cloak $\delta_x$ that optimize the latent representation of Glazed artwork
($x + \delta_x$) to be similar to the style transferred artwork ($\Omega(x, T)$). 
The Glaze optimization effectively moves the original image to a new position in 
the high dimensional latent space, causing model to learn an incorrect art style. 
Glaze calculates the latent space using the 
feature extractor ($\mathcal{E}$) from a diffusion model.
Formally, we write the Glaze 
optimization as solving the following:
\begin{align}
    \min_{\delta_x} ||\mathcal{E}(x + \delta_x) - \mathcal{E}(\Omega(x, T))||_2 \\
    \text{subject to } \text{LPIPS}(x + \delta_x, x) < p_{G} \notag
\end{align}
\noindent We use LPIPS, a popular human-perceived visual distortion metric~\cite{zhang2018unreasonable}, to bound the perturbation 
within a budget $p_{G}$. 

\para{IMPRESS Purification Method. } IMPRESS adds additional perturbation on top of a Glazed artwork 
hoping to ``purify'' the Glaze
effect -- recovering the precise latent representation of original artwork. 
First, the authors empirically find that when passing Glazed images through an image autoencoder, 
the output image looks more different from the input image, compared to the output 
when inputting a clean image to the same autoencoder. Then authors assume removing this particular 
discrepancy would guide them to find the original (non-Glazed) image. 

IMPRESS purification  
optimizes perturbations on Glaze images such that purified images
behave similarly to clean images
when passing through an autoencoder. The authors assume the optimization process will guide 
the image to move back to the original latent space of the non-Glazed image. 
Formally, IMPRESS purification optimize a perturbation $\delta_{pur}$ on 
a Glaze image $x_{glazed}$:

\begin{align}
    \min_{\delta_{pur}} ||(x_{glazed} + \delta_{pur}) - \text{VAE}(x_{glazed} + \delta_{pur}) ||_2^2 \\
    \text{subject to } \text{LPIPS}(x_{glazed} + \delta_{pur}, x_{glazed}) < p_{I} \notag
\end{align}

\noindent $\text{VAE}$ is an image autoencoder, which consists of an encoder $\mathcal{E}$ followed by a decoder $D$. IMPRESS uses the same
autoencoder as the stable diffusion model. The 
perturbation $\delta_{pur}$ is bounded by a LPIPS 
perturbation budget $p_{I}$. 
