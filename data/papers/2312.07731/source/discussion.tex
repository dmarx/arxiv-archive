\secspace
\section{Discussion}
\label{sec:discuss}

We conclude with a discussion of Glaze, then describe some potential implications of 
attempts to remove Glaze. 

\para{Challenges of evaluating mimicry protection. } Evaluating mimicry performance is
challenging with automated metrics. As we shown in \S\ref{sec:issues}, CLIP-based metric
do not work well. Another popular image quaility metric, Fréchet inception 
distance (FID), is also shown to be erronous at evaluating generative models~\cite{podell2023sdxl}. 
The standard evaluation approach today is to rely on human inspectors~\cite{podell2023sdxl}, 
which is not only expensive, but also produces results with high
variance. Particularly for art mimicry, 
the average human user (found on MTurk or Profilic) often does not have sufficient expertise
to judge whether a mimicry is successful or not, while sourcing to professional artists
can be prohibitively expensive. We hope future work can address this challenge by
designing automated metrics that can faithfully approximate ratings from human
artists. 

\para{Implications on Copyright Law.} Technical research results and their applications rarely have
legal implications in practice. In the case of Glaze and image copyrights,
however, there are interesting open questions with respect to copyright
law. More specifically, Section 1201 of US Copyright law (enacted in 1998 as
part of the DMCA)  ``prohibits circumvention of technological protection
measures employed by or on behalf of copyright owners to control access to
their works.''\footnote{``Legal Liability for Removal or Alteration of Copyright Management Information Under the 
DMCA,'' John DiGiacomo, September 2020, 
\url{https://revisionlegal.com/copyright/legal-liability-for-removal-or-alteration-of-copyright-management-information-under-the-dmca/}.}

For artists who apply Glaze to prevent unauthorized use of their art
images for training, applying Glaze to one's art likely qualifies as an
explicit statement of ``opt-out'' and a measure to control access to
copyrighted work. In this uncertain legal and regulatory landscape, it is an
interesting legal question to explore whether Section 1201 applies to the use
of Glaze and attempted counter-measures to Glaze.


\section*{Acknowledgements}  
We are grateful to Bochuan Cao, Changjiang Li, Ting Wang, 
Jinyuan Jia, Bo Li, Jinghui Chen for providing us code and 
discussing their attack with us prior to paper publication.


