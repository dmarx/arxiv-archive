\section{Discussion and Future Work}
\label{sec:disc}

In this work, we introduce \texttt{HypStructure}, a simple-yet-effective structural regularization framework for incorporating the label hierarchy into the representation space using hyperbolic geometry. In particular, we demonstrate that accurately embedding the hierarchical relationships leads to empirical improvements in both classification as well as the OOD detection tasks, while also learning \emph{hierarchy-informed} features that are more interpretable and exhibit less distortion with respect to the label hierarchy. We are also the first to formally characterize properties of \emph{hierarchy-informed} features via an eigenvalue analysis, and also relate it to the OOD detection task, to the best of our knowledge. We acknowledge that our proposed method depends on the availability or construction of an external hierarchy for computing the HypCPCC objective. If the hierarchy is unavailable or contains noise, this could present challenges. Therefore, it is important to evaluate how injecting noisy hierarchies into CPCC-based methods impacts downstream tasks. While the current work uses the \Poincare ball model, exploring the representation trade-offs and empirical performances using other models of hyperbolic geometry in \texttt{HypStructure}, such as the Lorentz model \citep{nickel2018learning} is an interesting future direction. Further theoretical investigation into establishing the error bounds of CPCC style structured regularization objectives is of interest as well.