\begin{table*}[ht!]
\label{tab:prompts}
    \centering
    \adjustbox{width=\textwidth}{
    \begin{tabular}{@{}p{5cm}p{5cm}p{5cm}@{}}
        \toprule
        \textbf{Task} & \textbf{Prompt} & \textbf{Description} \\
        \midrule
        \textbf{Plan Generation} & 
        \begin{minipage}[t]{5cm}
        \vspace{0pt}
        Generate an explanation and analysis, and plan to generate a prompt for writing a swi-prolog code for the last task. The 3 sections should be exactly outlined. Your plan should show enough intermediate reasoning steps towards the answer. Construct the plan as much as you can and describe the logic specifically. When constructing the plan for the code prompt, actively use swi prolog search capabilities.
        \end{minipage}
        & 
        \begin{minipage}[t]{5cm}
        \vspace{0pt}
        Detailed instructions for generating an outline and plan, with an emphasis on reasoning steps and using Prolog's search capabilities.
        \end{minipage}
        \\
        \midrule
        \textbf{Code Generation} & 
        \begin{minipage}[t]{5cm}
        \vspace{0pt}
        Write a Prolog code to solve using the plan. If there are unknown or stochastic atoms or predicates, fill in the values for them as a logical assumption and add a comment in the same line Assumed atom/predicate". Do not use write and read commands within the code. The code should be very detailed and utilize swi prolog capabilities to the fullest. To run the program, at the end create a predicate named "query" that returns the correct numerical answer. The last line of the program should be the commented-out driver predicate "query". Write only the code.
        \end{minipage}
        & 
        \begin{minipage}[t]{5cm}
        \vspace{0pt}
        Instructions for generating a Prolog code based on the plan with assumptions for unknown atoms. Emphasizes code details and a final ``query'' predicate.
        \end{minipage}
        \\
        \midrule
        \textbf{Simulated Search} & 
        \begin{minipage}[t]{5cm}
        \vspace{0pt}
        Ignoring the read commands, explicitly write out the search paths that are explored by the code:
        \#\#\#\#
        Here are the paths [Starting Search Simulation]: 
        \#\#\#\# 
        [Path 1]:
        \end{minipage}
        & 
        \begin{minipage}[t]{5cm}
        \vspace{0pt}
        A task to simulate and display the search paths that the Prolog code would follow during execution.
        \end{minipage}
        \\
        \midrule
        \textbf{Final Answer} & 
        \begin{minipage}[t]{5cm}
        \vspace{0pt}
        Given the plan, the code and the explored search paths answer the question above. Answer with the correct numerical answer.
        \#\#\#\#\# Here is the answer:
        \end{minipage}
        & 
        \begin{minipage}[t]{5cm}
        \vspace{0pt}
        Final prompt asking for the correct numerical answer based on the previous steps.
        \end{minipage}
        \\
        \bottomrule
    \end{tabular}
    }
    \caption{Table of Prompts for Plan, Code, Simulated Search, and Final Answer generation for GSM8K \citep{DBLP:journals/corr/abs-2110-14168}.}
\end{table*}
