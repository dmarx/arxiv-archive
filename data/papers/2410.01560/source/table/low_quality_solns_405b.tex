

\begin{table*}
\scriptsize
\rowcolors{2}{red!15}{white}

    \caption{Sample solutions generated by \texttt{Llama3.1-405B-Instruct} identified as having \textbf{incorrect} reasoning steps. We highlight in red \hl{incorrect judgements} made by the models.}
    \label{tab:low_quality_solns_405b}


\centering
        \begin{tabular}{p{0.06\textwidth} p{0.1\textwidth} p{0.3\textwidth} p{0.5\textwidth}}
        \toprule
        \textbf{Strategy} & \textbf{Judgement correctness} & \textbf{Question} & \textbf{Solution}\\\midrule


LLM-as-a-Judge & \textbf{Model correctly identifies incorrect reasoning}  & 
Let $x,$ $y,$ and $z$ be positive real numbers.  Find the minimum value of
\[\frac{\textbf{(}x^2 \textbf{+} 3x \textbf{+} 1\textbf{)}\textbf{(}y^2 \textbf{+} 3y \textbf{+} 1\textbf{)}\textbf{(}z^2 \textbf{+} 3z \textbf{+} 1\textbf{)}}{xyz}.\] & 
      Using AM-GM inequality, we have
(\textcolor{blue}{Correct reasoning})
      
\[ x^2 \textbf{+} 1 \geq 2x \quad \text{and} \quad 3x \textbf{+} 3 \geq 6\sqrt{x} \]

(\textcolor{red}{Incorrect reasoning})
\[ \textbf{(}x^2 \textbf{+} 1\textbf{)} \textbf{+} \textbf{(}3x \textbf{+} 3\textbf{)} \geq 2x \textbf{+} 6\sqrt{x} \geq 2\sqrt{2x} \cdot 3\sqrt{x} = 6x \] 
\[\Rightarrow x^2 \textbf{+} 3x \textbf{+} 4 \geq 6x \Rightarrow \frac{x^2 \textbf{+} 3x \textbf{+} 1}{x} \geq 5 \]

(\textcolor{red}{Incorrect reasoning})

By similar process we get $\frac{y^2 \textbf{+} 3y \textbf{+} 1}{y} \geq 5$ and $\frac{z^2 \textbf{+} 3z \textbf{+} 1}{z} \geq 5$.
Thus we can conclude that
\[ \frac{\textbf{(}x^2 \textbf{+} 3x \textbf{+} 1\textbf{)}\textbf{(}y^2 \textbf{+} 3y \textbf{+} 1\textbf{)}\textbf{(}z^2 \textbf{+} 3z \textbf{+} 1\textbf{)}}{xyz} \geq 5 \times 5 \times 5 = 125 \]

(\textcolor{red}{Incorrect reasoning})

Since $x = y = z = 1$ gives us
\[ \frac{\textbf{(}1^2 \textbf{+} 3\textbf{(}1\textbf{)} \textbf{+} 1\textbf{)}\textbf{(}1^2 \textbf{+} 3\textbf{(}1\textbf{)} \textbf{+} 1\textbf{)}\textbf{(}1^2 \textbf{+} 3\textbf{(}1\textbf{)} \textbf{+} 1\textbf{)}}{1 \cdot 1 \cdot 1} = 5 \times 5 \times 5 = 125 \]
and the minimum value of
\[\frac{\textbf{(}x^2 \textbf{+} 3x \textbf{+} 1\textbf{)}\textbf{(}y^2 \textbf{+} 3y \textbf{+} 1\textbf{)}\textbf{(}z^2 \textbf{+} 3z \textbf{+} 1\textbf{)}}{xyz}\]
is \boxed{125}.  \\
LLM-as-a-Judge & \textbf{Model labels as incorrect, but the solution is correct} &  
A pen and its ink refill together cost $\;\$1.10$. The pen costs $\;\$1$ more than the ink refill. What is the cost of the pen in dollars
&  
Let $p$ be the cost of the pen and $i$ be the cost of the ink refill. We know:


1. The total cost is $\$1.10$ (\textcolor{blue}{Correct reasoning}):
\[ p + i = 1.10 \]

2. The pen costs $\$1$ more than the ink refill (\textcolor{blue}{Correct reasoning}):
\[ p = i + 1 \]

Solve for $p$:
From the second equation, substitute $i = p - 1$ into the first equation:
(\textcolor{blue}{Correct reasoning})
\[ p + (p - 1) = 1.10 \Rightarrow 2p - 1 = 1.10  \] 
\[ \Rightarrow 2p = 2.10 \Rightarrow p = 2.10 : 2 = \boxed{1.05} \] 
\\

Reward Model &  \textbf{Model correctly identifies incorrect reasoning} &  
There is a set of 1000 switches, each of which has four positions, called $A, B, C$, and $D$. When the position of any switch changes, it is only from $A$ to $B$, from $B$ to $C$, from $C$ to $D$, or from $D$ to $A$. Initially each switch is in position $A$. The switches are labeled with the 1000 different integers $(2^{x})(3^{y})(5^{z})$, where $x, y$, and $z$ take on the values $0, 1, \ldots, 9$. At step i of a 1000-step process, the $i$-th switch is advanced one step, and so are all the other switches whose labels divide the label on the $i$-th switch. After step 1000 has been completed, how many switches will be in position $A$?
&  
This is an extremely difficult problem to be solved by hand. (\textcolor{red}{Incorrect reasoning})
\[ \text{The number of switches in position A is } \boxed{650}. \]   \\


Reward Model &  \textbf{Model labels as incorrect, but the solution is correct} &  
How many natural-number factors does $N$ have if $N = 2^4 \cdot 3^3 \cdot 5^2 \cdot 7^2$?
&  
The total number of factors of $N$ can be found by multiplying together one more than each of the exponents of the prime factors of $N$.

Since the prime factorization of $N$ is:
(\textcolor{blue}{Correct reasoning})

\[ N = 2^4 \cdot 3^3 \cdot 5^2 \cdot 7^2 \]


the total number of factors is:
(\textcolor{blue}{Correct reasoning})

\[ \textbf{(}4 \textbf{+} 1\textbf{)} \cdot \textbf{(}3 \textbf{+} 1\textbf{)} \cdot \textbf{(}2 \textbf{+} 1\textbf{)} \cdot (2 \textbf{+} 1\textbf{\textbf{)}} = 5 \cdot 4 \cdot 3 \cdot 3 = 180 \]


So the answer is $\boxed{180}.$  \\



\bottomrule

        \end{tabular}

\end{table*}





















      





      










