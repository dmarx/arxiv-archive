\vspace{-0.1in}
\section{Preliminary}
\vspace{-0.1in}
%The issue of copyright infringement in text-to-image (T2I) models has gained significant attention as these models become more prevalent. Early studies, such as those by Heikkilä [1], highlighted the risks of T2I models replicating copyrighted images from detailed prompts, emphasizing the need for robust prevention mechanisms. O'Leary [2] discussed the ethical and legal challenges of AI-generated content, advocating for clearer guidelines and regulations. Vincent [3] demonstrated that even general prompts could lead T2I models to generate images similar to copyrighted content, suggesting solutions like filtering training data and post-generation checks. Gao et al. [4] developed an adversarial attack framework to test T2I models, revealing vulnerabilities in many commercial services and calling for improved robustness and compliance protocols.
\paragraph{Copyright.}
Copyright is a legal protection provided to the owners of "original works of authorship", such as literature, music, and art~\citep{uscopyright2024uscopyright, uspto2024copyright}. This protection is granted to owners under the laws with the \textit{exclusive right to reproduce, or distribute} their works for a certain period of time~\citep{cornell106, uscopyright2024uscopyright}. Reproduction includes making copies of the work in any form, and distribution involves making the work available to the public through selling or lending copies. While the use of copyrighted data in AI models has been tacitly accepted for educational purposes, the rise of commercial AI systems has brought significant attention to the issue of copyright infringement~\citep{lawsuit1,lawsuit2NYTimes,lawsuit3Getty}. Opinions on the legal aspects of AI vary, but ethically, generative AI should not violate any of these rights to protect the intellectual property of the owners. 
In academia, numerous efforts have been made for copyright protection, e.g., training data protection~\citep{zhong2023copyright, shan2023glaze}, theoretical guarantees~\citep{bousquet2020synthetic, elkin2023can, vyas2023provable}, guided generation~\citep{schramowski2022safe, kumari2023ablating} and mechanism design~\citep{zhou2024plug, golatkar2024cpr, deng2024economic}. Despite the efforts, we reveals that commercial T2I systems still infringe copyrights despite careful alignment and red-teaming mechanisms. %Our result also suggests that future protection approaches should be studied under the context of carefully constructed prompts from strong attackers. 

\vspace{-0.12in}
\paragraph{Memorization in T2I models.}
Memorization has been known to occur in T2I models, sometimes producing near-exact reproductions of images from the training dataset~\citep{somepalli2023understanding}.~\citet{carlini2023extractdm} introduce the membership inference attack to extract the training dataset of diffusion models, and several works~\citep{somepalli2023diffusion, wen2024detecting, wang2024diagnosis} have been proposed to mitigate these memorization issues. Despite memorization is a well-known phenomenon, the quantitative evaluation of copyright violation in commercial T2I systems is under-explored. Thus, we propose an Automatic Prompt Generation Pipeline (APGP) to induce copyright infringement in these commercial T2I systems to evaluate the copyright violation using a single target image.

\vspace{-0.12in}
\paragraph{Prompt attack in T2I models.}
Previous attack approaches demonstrate the vulnerabilities in T2I diffusion models by attacking prompts to either generate different objects~\citep{maus2023black} or create potentially harmful images~\citep{yang2023sneakyprompt, zhai2024discovering}. Previous studies~\citep{zhang2023investigating} have explored high-risk prompts that increase copyright risks by pruning tokens based on attention scores, highlighting potential copyright risks but not causing direct infringement. In contrast, our method targets commercial T2I systems without accessing their weights, effectively "jailbreaking" these systems to demonstrate vulnerabilities related to exact copyright infringement.