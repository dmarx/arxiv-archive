\section{Introduction}

3D Gaussian Splatting~(3DGS)~\cite{kerbl3Dgaussians} has recently conquered the field of 3D reconstruction and image-based rendering. By representing a scene with a large set of tiny Gaussians, 3DGS allows for fast reconstruction and rendering, while nicely capturing fine details and complex light effects. Compared to earlier neural rendering methods such as NeRFs~\cite{mildenhall2020nerf}, 3DGS is 
much more efficient for both the reconstruction and rendering stages by a large margin. 


However, like NeRFs, Vanilla 3DGS does not allow easy edition of the reconstructed scene. The Gaussians are unstructured, disconnected from each other, and it is not clear how a designer can manipulate them, for example to animate the scene. Very recently, SuGaR~\cite{guedon2023sugar} showed how to extract a mesh from the output of 3DGS. Then, by constraining Gaussians to stay on the mesh, it is possible to edit the scene with traditional Computer Graphics tools for manipulating meshes. But by flattening the Gaussians onto the mesh, SuGaR loses the rendering quality possible with 3DGS for fuzzy materials and volumetric effects.


In this paper, we introduce Gaussian Frosting--or Frosting, for short--a hybrid representation of 3D scenes that is editable as a mesh while providing a rendering quality at least equal, sometimes superior to 3DGS. The key idea of Frosting is to augment a mesh with a layer containing Gaussians. The thickness of the Frosting layer adapts locally to the material of the scene: The layer should be thin for flat surfaces to avoid undesirable volumetric effects, and thicker around fuzzy materials for realistic rendering. As shown in Figure~\ref{fig:teaser}, using the Frosting representation, we can not only retrieve a highly accurate editable mesh but also render complex volumetric effects in real-time. 


Frosting is reminiscent of the Adaptive Shells representation~\cite{wang-siggraphasia2023-adaptive-shells}, which relies on two explicit triangle meshes extracted from a Signed Distance Function to control volumetric effects. Still, Frosting allies a rendering quality superior to the quality of Adaptive Shells to the speed efficiency of 3DGS for reconstruction and rendering, while being easily editable as it depends on a single mesh.


While the Frosting representation is simple, it is challenging to define the local thickness of its layer. To extract the mesh, we essentially rely on SuGaR, which we improved with a technique~(described in the supplementary material) to automatically tune a critical hyperparameter. To estimate the Frosting thickness, we introduce a method to define an inner and outer bound for the Frosting layer at each vertex of this mesh based on the Gaussians initially retrieved by 3DGS around the mesh.  Finally, we populate the Frosting layer with randomly sampled Gaussians and optimize these Gaussians constrained to stay within the layer. We also propose a simple method to automatically adjust in real-time the parameters of the Gaussians when animating the mesh.

In summary, we propose a simple yet powerful surface representation that captures complex volumetric effects, can be edited with traditional tools, and can be rendered in real-time. We also propose a method to build this representation from images, based on recent developments on Gaussian Splatting. We will release our code including a viewer usable in browser as an additional contribution. 