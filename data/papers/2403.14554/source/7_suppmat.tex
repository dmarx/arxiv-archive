\newpage
\centerline{\large Supplementary Material}

\vspace{0.3in}
\noindent
In this supplementary material, we provide the following elements:
\begin{itemize}
    \item A description of our method to improve the surface reconstruction from SuGaR~\cite{guedon2023sugar}.
    \item Additional details about our strategy to initialize the Frosting layer and automatically adjust Gaussians' parameters when deforming, editing, or animating our representation.
\end{itemize}
% TODO
We also provide a \href{https://anttwo.github.io/frosting/}{\underline{video}} that offers an overview of the approach and showcases additional qualitative results. Specifically, the video demonstrates how Frosting can be used to edit, combine or animate Gaussian Splatting representations.

\setcounter{section}{+6}
\section{Improving surface reconstruction}

% \subsubsection{Improving surface reconstruction.} 

We improve the surface reconstruction method from SuGaR by proposing a way to automatically adjust the hyperparameter of the Poisson surface reconstruction~\cite{kazhdan-2006-poissonsurfacereconstruction} stage.

Poisson surface reconstruction first recovers an underlying occupancy field $\chi:\IR^3\mapsto [0,1]$
and applies a marching algorithm on $\chi$, which allows for a much better mesh reconstruction than the density function. This approach allows for high scalability as the marching algorithm is applied only in voxels located close to the point cloud. 

To estimate $\chi$, Poisson surface reconstruction discretizes the scene into $2^D \times 2^D \times 2^D$ cells by adapting an octree with depth $D\in \IN$ to the input samples.  $D$ is a hyperparameter provided by the user: The higher $D$, the higher the resolution of the mesh.

By default, SuGaR~\cite{guedon2023sugar} uses a large depth $D=10$ for any scene, as it guarantees a high level of details. However, if the resolution is too high with respect to the complexity of the geometry and the size of the details in the scene, the shapes of the Gaussians become visible as ellipsoidal bumps on the surface of the mesh, and create incorrect bumps or self-intersections. More importantly, holes can also appear in the geometry when $D$ is too large with regards to the density of the Gaussians and the sampled point cloud.

% However, depending on the spatial extent of the scene, on the complexity of the geometry, or even on the distance between the camera and the objects to reconstruct, a parameter $D$ that is too large can produce artifacts in the scene, as shown in Figure~\ref{fig:mesh-comparison} and mentioned in GitHub issues of SuGaR's implementation. 


We therefore introduce a method to automatically select $D$. A simple strategy would be to adjust the depth of the octree such that the size of a cell is approximately equal or larger than the average size of the Gaussians in the scene, normalized by the spatial extent of the point cloud used for reconstruction. Unfortunately, this does not work well in practice: We found that whatever the scene (real or synthetic) or the number of Gaussians to represent it, Gaussian Splatting optimization systematically converges toward a varied collection of Gaussian sizes, so that there is no noticeable difference or pattern in the distribution of sizes between scenes.

We noticed that the distance between Gaussians is much more representative of the geometrical complexity of the scene and thus a reliable cue to fix $D$. Indeed, a large but very detailed shape can be reconstructed using Gaussians with large size, if these Gaussians are close to each other. On the contrary, whatever their size, if the centers of the Gaussian are too far from each other, then the rendered geometry will look rough. 

Consequently, to first evaluate the geometrical complexity of a scene, we propose to compute, for each Gaussian $g$ in the scene, the distance between $g$ and its nearest neighbor Gaussian. We use these distances to define the following geometrical complexity score $CS$:
%
\begin{equation}
    CS = Q_{0.1} \left( \left\{\min_{g'\neq g} \frac{\|\mu_g - \mu_{g'}\|_2}{L}\right\}_{g\in \calG} \right) \> ,
    \label{eq:complexity_score}
\end{equation}
%
where $\calG$ is the set of all 3D Gaussians in the scene, $L$ is the length of the longest edge of the bounding box of the point cloud to use in Poisson reconstruction, and $Q_{0.1}$ is the function that returns the 0.1-quantile of a list. We use the 0.1-quantile rather than the average because Gaussians that have a neighbor close to them generally encode details in the scene, which provide a much more reliable and less noisy criterion than using the overall average. We also use a quantile rather than a minimum to be robust to extreme values. In short, this complexity score $CS$ is a canonical distance between the closest Gaussians in the scene, i.e., the distance between neighbor Gaussians that reconstruct details in the scene. 

% \input{figures/complexity_score}

Since the normalized length of a cell in the octree is $2^{-\bar{D}}$ and this score represents a canonical normalized distance between Gaussians representing details in the scene, we can compute a natural optimal depth $\bar{D}$ for the Poisson reconstruction algorithm:
%
\begin{equation}
    \bar{D} = \lfloor -\log_2 \left(\gamma \times CS\right) \rfloor \> ,
    \label{eq:optimal-depth}
\end{equation}
%
where $\gamma > 0$ is a hyperparameter that does not depend on the scene and its geometrical complexity. This formula guarantees that the size of the cells is as close as possible but greater than $\gamma \times CS$. Decreasing the value of $\gamma$ increases the resolution of the reconstruction. But for a given $\gamma$, whatever the dataset or the complexity of the scene, this formula enforces the scene to be reconstructed with a similar level of smoothness. 

Choosing $\gamma$ is therefore much easier than having to tune $D$ as it is not dependent on the scene. 
In practice, we use $\gamma=100$ for all the scenes. Our experiments validates that this method to fix $D$ results in greater rendering performance.



\section{Initializing the frosting layer}

\subsection{Sampling Gaussians in the frosting layer} 

\subsubsection{Sampling more Gaussians in thicker parts of the frosting.} For a given budget $N$ of Gaussians provided by the user, we initialize $N$ Gaussians in the scene by sampling $N$ 3D centers $\mu_g$ in the frosting layer. Specifically, for sampling a single Gaussian, we first randomly select a prismatic cell with a probability proportional to its volume. Then, we sample random coordinates that sum up to 1. 
%
This sampling allows for allocating more Gaussians in areas with fuzzy and complex geometry, where more volumetric rendering is needed. However, flat parts  in the layer may also need a large number of Gaussians to recover texture details. Therefore, in practice, we instantiate $N/2$ Gaussians with uniform probabilities in the prismatic cells, and $N/2$ Gaussians with probabilities proportional to the volume of the cell.

\subsubsection{Contracting volumes in unbounded scenes.} In real unbounded scenes, 3D Gaussians located far away from the center of the scene can have a significantly large volume despite their limited participation in the final rendering. This can lead to an unnecessarily large number of Gaussians being sampled in the frosting layer far away from the training camera poses. To address this issue, we propose distributing distant Gaussians proportionally to disparity (inverse distance) rather than distance.

When sampling Gaussians in practice, we start by contracting the volumes of the prismatic cells. We achieve this by applying a continuous transformation $f:\IR^3 \rightarrow \IR^3$ to the vertices of the outer and inner bounds of the frosting layer. 
%
Then, we compute the volumes of the resulting ``contracted'' prismatic cells and use these adjusted volumes for sampling Gaussians within the frosting layer, as previously described. The transformation function $f$ aims to contract the volume of prismatic cells located far away from the center of the scene. We define $f$ using a formula similar to the contraction transformation introduced in Mip-NeRF~360~\cite{barron2022mipnerf360}:
\begin{equation}
    f(x) = \begin{cases}
        x & \text{if} \>\>\>\> \|x-c\| \leq l\\
        c + l \times \left(2 - \frac{l}{\|x-c\|}\right) \left(\frac{x-c}{\|x-c\|}\right) & \text{if} \>\>\>\> \|x-c\| > l
    \end{cases} \>,
    \label{eq:contraction}
\end{equation}
where $c\in \IR^3$ is the center of the bounding box containing all training camera positions, and $l\in\IR_+$ is equal to half the length of the diagional of the same bounding box. 
%
We choose the bounding box of the camera positions as our reference scale because both 3D Gaussian Splatting~\cite{kerbl3Dgaussians} and SuGaR~\cite{guedon2023sugar} use this same reference for scaling learning rates and distinguishing foreground from background in unbounded scenes.

\subsection{Avoiding self-intersections in the frosting layer}

In the main paper, we define the inner and outer bounds of the frosting layer by adding inner and outer shifts $\innershift_i$ and $\outershift_i$ to the vertices $\vec{v}_i$ of the base mesh. This results in two bounding surfaces with vertices $\vec{v}_i + \innershift_i$ and $\vec{v}_i + \outershift_i$. 
%
In practice, we wish to minimize self-intersections within the frosting layer, specifically avoiding prismatic cells intersecting with each other.

While self-intersections do not directly impact rendering quality, they can lead to artifacts during scene editing or animation. Consider the scenario where different cells intersect. In such cases, moving a specific triangle of the base mesh may not affect all Gaussians intersecting the surrounding cell: Some Gaussians may belong to prismatic cells associated with different triangles, resulting in artifacts due to their failure to follow local motion or scene edits.

To mitigate self-intersections, we adopt an indirect approach for initializing the shifts $\innershift$ and $\outershift$. 
Instead of using the final computed values directly, we start with shifts equal to zero and progressively increase them until reaching their final values. 
As soon as an inner vertex (or outer vertex) of a prismatic cell is detected to intersect another cell, we stop further increases in its inner shift (or outer shift). 
This straightforward process significantly reduces self-intersections in the frosting layer while maintaining rendering performance.

By following this approach, we ensure that the frosting layer remains free from unwanted artifacts while preserving efficient rendering capabilities.

\section{Adjusting Gaussians' parameters for edition}

When editing or animating the scene, we automatically adjust Gaussians' parameters. Specifically, in a given prismatic cell with center $\vec{c}$ and six vertices $\vec{v}_i$ for $0\leq i<6$, we first estimate the local transformation at each vertex $\vec{v}_i$ by computing the rotation and rescaling of the vector $(c - \vec{v}_i)$. 

To compute the local rotations at vertex $\vec{v}_i$, we use an axis-angle representation where the axis angle is the normalized cross-product between the previous and the current values of the vector $(c - \vec{v}_i)$.
%
The local rescaling transformation at vertex $\vec{v}_i$ is computed as the transformation that scales along axis $(c - \vec{v}_i)$ with the appropriate factor but leaves other axes unchanged.

To update the scaling factors and rotation of a Gaussian $g$, we first apply each of these six transformations on the three main axes of the Gaussian. We then average the resulting axes using the barycentric coordinates of the center $\mu_g$ of the Gaussian. We finally orthonormalize the three resulting axes.
