\clearpage
\section{Methodology \& Guidelines}
\label{sec:methodology}

We develop the following methodology to guide the collection of these tools and resources, as well as our analysis.
First, we've divided the model development pipeline into several phases, illustrated in \cref{fig:cheatsheet-flow}.
Despite the visual representation, we acknowledge these phases are frequently interrelated rather than sequential.
We also include categories that have been identified by existing scholarship as necessary to responsible development, even though they are frequently omitted from development pipelines: such as documentation, environmental impact estimation, or risks and harms evaluation.
Next, authors are allocated to these phases based on their areas of expertise, or to modalities (text, vision, or speech) across a few phases of development.
Each author surveys the literature in their segment of model development to find a mix of (a) relevant tools, in the form of repositories, APIs, or interfaces, (b) scientific literature that directly surveys or guides development decisions, and (c) frameworks or standards for development (such as documentation or risk taxonomies).
Due to the breadth of categories, the literature survey is inevitably non-exhaustive, but reflects a dedicated search for the most prominent tools in each area.
Certain phases of development, such as model training, have thousands of available repositories, so we curate a criteria for inclusion (outlined below) that prioritizes certain qualities: popularity, usefulness, and advancing responsible practices.

% \vspace{-2mm}
\paragraph{Criteria for Inclusion.}
\label{sec:criteria}
These principles for inclusion, described below, are incomplete and subjective, but we still believe sufficiently rigorous to make for a thorough and useful compilation of resources.
The resources are selected based on a literature review for each phase of foundation model development. 
Inclusion is predicated on a series of considerations, including: the popularity of the tool on Hugging Face or GitHub, the perceived helpfulness as a development tool, the extent and quality of the documentation, the insights brought to the development process, and, in some cases, the lack of awareness a useful resource has received in the AI community.
For an example of this last consideration, in \cref{sec:finetune} we try to include more Finetuning Data Catalogs for lower resource languages, that often receive less attention than ones featured more prominently on Hugging Face's Dataset Hub.
Rather than sharing primarily academic literature as in most surveys, we focus on tools, such as data catalogs, search/analysis tools, evaluation repositories, and, selectively, literature that summarizes, surveys or guides important development decisions.
Further, we hope to make the coverage more comprehensive with an open call for community contributions.

% \vspace{-2mm}
% \paragraph{How to contribute?}
% \alon{We probably need to comment this paragraph out or rewrite it. Specifically, including the website will break anonymity}
% We intend for the web version of the cheatsheet to be a crowdsourced, interactive, living document, with search and filter tools.
% To contribute to this resource, follow instructions given in the website: \href{http://fmcheatsheet.org}{fmcheatsheet.org}.
% Contributions should follow the criteria for inclusion outlined above, scoped to resources, such as tools, artifacts, or helpful context papers, that directly inform responsible foundation model development.
% Project maintainers will review suggested contributions with these criteria.
% Significant contributions will be recognized in the web search tool, and in follow up write-ups to this guide.
% Contributed content will be reviewed by project maintainers, to assess that the resources fit the aspirational criteria for inclusion outlined above.

% \vspace{-2mm}
\paragraph{Scope \& Limitations.} 
We've compiled resources, tools, and papers that guide model development, and which we believe will be especially helpful to nascent (and often experienced) foundation model developers.
However, this guide is \emph{far from exhaustive}---and here's what to consider when using it:

\begin{itemize}[itemsep=1pt]%, wide=5pt]
    \item \textbf{Temporal scope.} Foundation model development is a rapidly evolving science. \textbf{This document is only a sample, dated to March 2024.}
    % \alon{Need to remove references to the website}
    % The full cheatsheet is more comprehensive and open for on-going public contributions: \href{http://fmcheatsheet.org}{fmcheatsheet.org}.
    \item \textbf{Applicable developers.} We scope these resources to mid-to-small foundation model developers. Large organizations, with commercial services and/or wide user bases, have broader considerations for responsible development that what is outlined in this work.
    \item \textbf{Development phases \& modalities} We've scoped our data modalities only to \textbf{text, vision, and speech}, and to the phases of development outlined in \cref{fig:cheatsheet-flow}.
    We support multilingual resources, but acknowledge there remain significant community-wide gaps in awareness and adoption here..
    % \item A cheatsheet \textbf{cannot be comprehensive}. We prioritize resources we have found helpful, and rely heavily on survey papers and repositories to point out the many other awesome works which deserve consideration, especially for developers who plan to dive deeper into a topic.
    \item \textbf{We remind users to follow standard practice in assessing the security and viability of each tool, particularly for their circumstance.} At times we have provided resources with conflicting advice, as it is helpful to be aware of divided community perspectives. Our notes throughout are designed to contextualize these resources, to help guide the reader's judgement.
\end{itemize}

\clearpage

% \vspace{-2mm}
% \paragraph{Who are we?}
% \alon{I think this section probably needs to be removed for anonymity purposes.}
% An organic collective of volunteers have contributed to this cheatsheet, spanning developers of several notable datasets (MasakhaNER \citep{adelani2021masakhaner}, the Pile \citep{gao2020pile}, ROOTS \citep{NEURIPS2022_ce9e92e3}, Dolma \citep{dolma}, the Flan Collection \citep{longpre2023flan}, the Data Provenance Collection \citep{longpre2023data}), models (OpenFlamingo \citep{awadalla2023openflamingo}, Pythia \citep{biderman2023pythia}, Flan-PaLM \citep{chung2022scaling}, RWKV \citep{peng2023rwkv}), and benchmarks (LM Eval Harness \citep{eval-harness}, HELM \citep{liang2022holistic}) in the community.
% Most importantly, contributors (\cref{app:contributions}) have carefully assembled the tools and resources that have enabled them to build this infrastructure responsibly with an eye on social and scientific impact.