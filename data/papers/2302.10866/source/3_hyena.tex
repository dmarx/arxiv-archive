\section{Hyena: Definition and Properties}\label{allofhyena}
%
In this section, we define Hyena, a class of \textit{data-controlled} operators consisting of a recurrence of multiplicative gating interactions and long convolutions. Instead of seeking an approximation to attention, we guide our design by intentionally incorporating key computational properties of attention, including the decoupling of sequence length and parameter counts.
%
\subsection{${\sf Hyena}$ Recurrences}\label{hyena_op}
%
At a high level, ${\sf Hyena}$ consists of the following steps (setting $D = 1$ for clarity):
\begin{itemize}
    \item[$i.$] Compute a set of $N + 1$ linear projections of the input, similarly to attention. The number of projections $(v_t, x^1_t, \dots, x^N_t)$ need not be three. One projection takes the role of value, such that a linear input-output function can be defined as $y = \sH(u) v$ for some $\sH(u)$.
    \item[$ii.$] The matrix $\sH(u)$ is defined by interleaving implicit long convolutions and element-wise multiplication with one projection $x^i$ at a time, until all projections are exhausted. Evaluation of $\sH(u) v$ is done efficiently \textbf{without materializing} $\sH(u)$. By doing so, we implicitly define a data-controlled operator as a factorization of a matrix. The long convolutions forming $\sH(u)$ are parametrized implicitly to retain sublinear parameter scaling in sequence length.
\end{itemize}
%
Next, we formally define ${\sf Hyena}$, starting with its computational model. We leave the analysis of its data-controlled matrix form for the latter part of the section.
%
\begin{tcolorbox}[enhanced, sharp corners, drop fuzzy shadow, frame hidden, colback=yellow!15]
\begin{definition}[Order--$N$ $\sf Hyena$ Operator] 
Let $(v, x^1, \cdots, x^N)$ be projections of the input and let $h^1,\dots, h^N$ be a set of learnable filters. The ${\sf Hyena}_N$ operator is defined by the recurrence:
%
\begin{equation}\label{eq:hyena}
    \begin{aligned}
        z^1_t &={\ocra v}_t\\
        z^{n+1}_t & = x_t^n (h^n * z^n)_t & n=1,\dots,N\\
        {\color{blue!70}y}_t &= z_t^{N+1}
    \end{aligned} 
\end{equation}
\end{definition}
\end{tcolorbox}
%
\begin{remark}
The time complexity of a $\sf Hyena$ recurrence is $\mathcal{O}(N L \log_2 L)$. The input-output map can be rewritten as 
\[
y = x^N \cdot (h^N * ( x^{N-1} \cdot (h^{N-1} * (\cdots))))
\]
where each convolution is performed through the Fourier domain in $\mathcal{O}(L \log_2 L)$.
\end{remark}
%
Interestingly, the element-wise product in time domain corresponds to convolution in frequency domain, i.e.
%
\[ x_tu_t = (\hat x * \hat u)_t, \]
%
where $\hat x,\hat u$ denote the DFT of $x$ and $u$, respectively. Thus, $\sf Hyena$ is alternatively applying convolutions in the time and then the frequency domain (or alternatively applying element-wise products in the time and frequency domain). One potential explanation for the effectiveness of this procedure is that the convolution in the time domain (element-wise multiplication in the frequency domain) increases the memory length, allowing for a broader context to be taken into account. On the other hand, the element-wise multiplication in the time domain (convolution in the frequency domain) allows for more fine-grained selection of specific frequency components of the signal.
%
\subsection{${\sf Hyena}$ Matrices} 
%
${\sf Hyena}$ operators build on the ${\sf H3}$ mechanism developed by \citep{dao2022hungry}. For clarity of exposition, we once again consider the SISO case ($D=1$). Let $\sD_q$ and $\sD_k$ be the $L$-by-$L$ diagonal matrices whose respective main diagonal entries are the respective entries of $q$ and $k$. ${\sf H3}$ realizes a surrogate attention matrix with a data-controlled, parametrized decomposition in four terms:
%
\begin{equation}\label{eq:linear_attention}
    \begin{aligned}
        \sA(q,k) &= \sD_q \sS_\psi \sD_k \sS_\varphi \\
        {\sf H3}(q, k, v) &= \sA(q,k) v
    \end{aligned}
\end{equation}
%
where $\sS_\varphi,\sS_\psi$ are the Toeplitz matrices of learnable \textbf{causal} filters $\varphi,\psi$ parametrized via SSMs\footnote{For consistency with our discussion, we have swapped $k$ and $v$ compared to the notation in \citep{dao2022hungry}.}. Alongside the $qkv$-projections the filters constitute our degrees of freedom in the layer design. This decomposition allows evaluation of \eqref{eq:linear_attention} in just $\cO(L \log_2 L)$ time (two FFT convolutions and two element-wise products), i.e.
%
\begin{equation}
    \begin{aligned}
        z_{t} &= k_{t}(\varphi * v)_t \\
        y_{t} &= q_{t}(\psi * z)_t
    \end{aligned}
\end{equation}
%
$\small \sf Hyena$ represents a generalization of \eqref{eq:linear_attention} for an arbitrary number of projections -- not limited to three -- and with implicit free-form long filters for the convolutions. The resulting recurrence \eqref{eq:hyena} can be also represented in matrix form $y=\sH(u)v$. Let $\sD_x^n=\diag(x^n)\in\R^{L\x L}$ and let $\sS_h^n$ be the Toeplitz matrix corresponding to filter $h^n$. The resulting $\sf Hyena$ recurrence is linear in $v$ and can be rewritten in matrix form:
%
\[
    y = \sH(u)v = \sD_x^N\sS_h^N \cdots \sD_x^2\sS_h^2\sD_{x}^1\sS_h^1 v
\]
%
Figure~\ref{fig:hyena_matrices} visualizes an example decomposition.
%
\begin{remark}[$\sf Hyena$ generalizes $\sf H3$ and $\sf GSS$.]
    The $\sf H3$ mechanism \citep{dao2022hungry} corresponds to ${\sf Hyena}_2$ and $\sf GSS$ \citep{mehta2022long} is ${\sf Hyena}_{1}$, with a particular choice of parametrization for the long convolutions (SSMs).
\end{remark}
%
Analysis of the $\sf H3$ mechanism as a decomposition $\sD_q\sS_\psi\sD_k\sS_\varphi$ of its surrogate attention matrix\footnote{Some of this analysis is reported in the Appendix.} clarifies a connection to fast evaluation algorithms for matrix-vector multiplications. In particular, the generalization of \eqref{eq:linear_attention} to an arbitrary order is inspired by fast evaluation algorithms for structured dense matrices based on \textit{butterfly} decompositions \citep{li2015butterfly,dao2019learning, dao2022monarch}, with length of the decomposition closely tied to its expressivity (in the classes of matrices it can represent). The ${\sf Hyena}$ operator blends data control with a special case of butterfly decomposition.         
%
\begin{remark}
${\sf Hyena}$ operators have unbounded context. Namely, they are not artificially restricted by e.g., locality, and can learn long-range dependencies between any of the elements of $v$ via long convolutions, which we discuss next.
\end{remark}
%
\subsection{${\sf Hyena}$ Filters}\label{hyena_ker}
%
Here we provide details on the convolution parametrization. We represent the filters of each ${\sf Hyena}$ operator as a map from the time (or space) domain $t$ to values $h_t$, and learn it with a shallow feed-forward neural network ({$\sf FFN$}):
%
\begin{equation}\label{filt}
    h_t = {\sf Window}(t)\cdot({\sf FFN} \circ {\sf PositionalEncoding}) (t)
\end{equation}
%
This approach builds on the neural implicit representation literature \citep{mildenhall2021nerf,sitzmann2020implicit}, which has found application in long convolution layers \citep{romero2021ckconv, romero2021flexconv}. One advantage of \eqref{filt} is given by the decoupling of filter length and parameter cost. 
%
\paragraph{Specializing filters in Hyena}
%
The window and positional encoding functions are used to specialize filters in ${\sf Hyena}$ operators, biasing them towards a specific type. Figure \ref{fig:modul} provides an important example: we choose at least one of the convolutions in ${\sf Hyena}$ to be shaped towards exponential decay, mirroring the findings of \citep{li2022makes} in other applications.
%
\begin{figure}[t]
    \centering
    % This file was created with tikzplotlib v0.10.1.
\begin{tikzpicture}[font=\small]


\definecolor{darkgray176}{RGB}{176,176,176}
\definecolor{dodgerblue}{RGB}{30,144,255}
\definecolor{tomato}{RGB}{255,99,71}

\begin{groupplot}[group style={group size=3 by 1, horizontal sep=0.1cm}]
\nextgroupplot[
width=.4\linewidth, height=3cm,
tick align=outside,
tick pos=left,
title={\(\displaystyle \small{\sf FFN}(t)\)},
title style={yshift=-0.25cm},
x grid style={darkgray176},
xmin=0, xmax=63,
xtick style={color=black},
xtick={-20,0,20,40,60,80},
xticklabels={
  \(\displaystyle {\ensuremath{-}20}\),
  \(\displaystyle {0}\),
  \(\displaystyle {20}\),
  \(\displaystyle {40}\),
  \(\displaystyle {60}\),
  \(\displaystyle {80}\)
},
xlabel={$\small\sf Sequence~Length$},
xlabel style={yshift=0.4cm},
y grid style={darkgray176},
ymin=-1, ymax=1,
ymin=-1, ymax=1,
ytick style={
    /pgfplots/major tick length=0pt,
},
xtick style={
    /pgfplots/major tick length=0pt,
},
ytick={-0.5, 0.5},
xtick={},
xticklabels={},
yticklabels={}
]
\path [draw=tomato, line width=0.5pt]
(axis cs:0,0)
--(axis cs:0,0.0827276557683945);

\path [draw=tomato, line width=0.5pt]
(axis cs:1,0)
--(axis cs:1,-0.286112189292908);

\path [draw=tomato, line width=0.5pt]
(axis cs:2,0)
--(axis cs:2,0.00687235593795776);

\path [draw=tomato, line width=0.5pt]
(axis cs:3,0)
--(axis cs:3,-0.747102022171021);

\path [draw=tomato, line width=0.5pt]
(axis cs:4,0)
--(axis cs:4,0.0641668513417244);

\path [draw=tomato, line width=0.5pt]
(axis cs:5,0)
--(axis cs:5,-0.646571576595306);

\path [draw=tomato, line width=0.5pt]
(axis cs:6,0)
--(axis cs:6,0.0942377746105194);

\path [draw=tomato, line width=0.5pt]
(axis cs:7,0)
--(axis cs:7,0.701024413108826);

\path [draw=tomato, line width=0.5pt]
(axis cs:8,0)
--(axis cs:8,0.218265861272812);

\path [draw=tomato, line width=0.5pt]
(axis cs:9,0)
--(axis cs:9,-0.193984031677246);

\path [draw=tomato, line width=0.5pt]
(axis cs:10,0)
--(axis cs:10,-0.128884062170982);

\path [draw=tomato, line width=0.5pt]
(axis cs:11,0)
--(axis cs:11,0.294504374265671);

\path [draw=tomato, line width=0.5pt]
(axis cs:12,0)
--(axis cs:12,0.597060859203339);

\path [draw=tomato, line width=0.5pt]
(axis cs:13,0)
--(axis cs:13,0.0985148549079895);

\path [draw=tomato, line width=0.5pt]
(axis cs:14,0)
--(axis cs:14,-0.355021268129349);

\path [draw=tomato, line width=0.5pt]
(axis cs:15,0)
--(axis cs:15,0.451927185058594);

\path [draw=tomato, line width=0.5pt]
(axis cs:16,0)
--(axis cs:16,0.051352247595787);

\path [draw=tomato, line width=0.5pt]
(axis cs:17,0)
--(axis cs:17,-0.58028256893158);

\path [draw=tomato, line width=0.5pt]
(axis cs:18,0)
--(axis cs:18,0.37768816947937);

\path [draw=tomato, line width=0.5pt]
(axis cs:19,0)
--(axis cs:19,-0.397529572248459);

\path [draw=tomato, line width=0.5pt]
(axis cs:20,0)
--(axis cs:20,-0.292246282100677);

\path [draw=tomato, line width=0.5pt]
(axis cs:21,0)
--(axis cs:21,0.220617383718491);

\path [draw=tomato, line width=0.5pt]
(axis cs:22,0)
--(axis cs:22,0.196079254150391);

\path [draw=tomato, line width=0.5pt]
(axis cs:23,0)
--(axis cs:23,0.0956322178244591);

\path [draw=tomato, line width=0.5pt]
(axis cs:24,0)
--(axis cs:24,0.45118436217308);

\path [draw=tomato, line width=0.5pt]
(axis cs:25,0)
--(axis cs:25,-0.375823110342026);

\path [draw=tomato, line width=0.5pt]
(axis cs:26,0)
--(axis cs:26,-0.551573514938354);

\path [draw=tomato, line width=0.5pt]
(axis cs:27,0)
--(axis cs:27,0.430078268051147);

\path [draw=tomato, line width=0.5pt]
(axis cs:28,0)
--(axis cs:28,0.279720634222031);

\path [draw=tomato, line width=0.5pt]
(axis cs:29,0)
--(axis cs:29,0.13493886590004);

\path [draw=tomato, line width=0.5pt]
(axis cs:30,0)
--(axis cs:30,-0.149687111377716);

\path [draw=tomato, line width=0.5pt]
(axis cs:31,0)
--(axis cs:31,0.148303136229515);

\path [draw=tomato, line width=0.5pt]
(axis cs:32,0)
--(axis cs:32,-0.219151318073273);

\path [draw=tomato, line width=0.5pt]
(axis cs:33,0)
--(axis cs:33,0.43656313419342);

\path [draw=tomato, line width=0.5pt]
(axis cs:34,0)
--(axis cs:34,-0.168442666530609);

\path [draw=tomato, line width=0.5pt]
(axis cs:35,0)
--(axis cs:35,0.187510102987289);

\path [draw=tomato, line width=0.5pt]
(axis cs:36,0)
--(axis cs:36,0.108915574848652);

\path [draw=tomato, line width=0.5pt]
(axis cs:37,0)
--(axis cs:37,-0.336748242378235);

\path [draw=tomato, line width=0.5pt]
(axis cs:38,0)
--(axis cs:38,0.2141472697258);

\path [draw=tomato, line width=0.5pt]
(axis cs:39,0)
--(axis cs:39,-0.031447671353817);

\path [draw=tomato, line width=0.5pt]
(axis cs:40,0)
--(axis cs:40,-0.348572671413422);

\path [draw=tomato, line width=0.5pt]
(axis cs:41,0)
--(axis cs:41,-0.17571160197258);

\path [draw=tomato, line width=0.5pt]
(axis cs:42,0)
--(axis cs:42,-0.250227987766266);

\path [draw=tomato, line width=0.5pt]
(axis cs:43,0)
--(axis cs:43,-0.0833175778388977);

\path [draw=tomato, line width=0.5pt]
(axis cs:44,0)
--(axis cs:44,0.0600676089525223);

\path [draw=tomato, line width=0.5pt]
(axis cs:45,0)
--(axis cs:45,-0.286332786083221);

\path [draw=tomato, line width=0.5pt]
(axis cs:46,0)
--(axis cs:46,0.0833164602518082);

\path [draw=tomato, line width=0.5pt]
(axis cs:47,0)
--(axis cs:47,0.352049469947815);

\path [draw=tomato, line width=0.5pt]
(axis cs:48,0)
--(axis cs:48,-0.602340042591095);

\path [draw=tomato, line width=0.5pt]
(axis cs:49,0)
--(axis cs:49,0.355826735496521);

\path [draw=tomato, line width=0.5pt]
(axis cs:50,0)
--(axis cs:50,-0.0874901711940765);

\path [draw=tomato, line width=0.5pt]
(axis cs:51,0)
--(axis cs:51,0.290616244077682);

\path [draw=tomato, line width=0.5pt]
(axis cs:52,0)
--(axis cs:52,0.196222603321075);

\path [draw=tomato, line width=0.5pt]
(axis cs:53,0)
--(axis cs:53,0.328381299972534);

\path [draw=tomato, line width=0.5pt]
(axis cs:54,0)
--(axis cs:54,-0.50971782207489);

\path [draw=tomato, line width=0.5pt]
(axis cs:55,0)
--(axis cs:55,0.476836383342743);

\path [draw=tomato, line width=0.5pt]
(axis cs:56,0)
--(axis cs:56,-0.223778992891312);

\path [draw=tomato, line width=0.5pt]
(axis cs:57,0)
--(axis cs:57,0.101644188165665);

\path [draw=tomato, line width=0.5pt]
(axis cs:58,0)
--(axis cs:58,0.260815739631653);

\path [draw=tomato, line width=0.5pt]
(axis cs:59,0)
--(axis cs:59,0.323934018611908);

\path [draw=tomato, line width=0.5pt]
(axis cs:60,0)
--(axis cs:60,-0.025722473859787);

\path [draw=tomato, line width=0.5pt]
(axis cs:61,0)
--(axis cs:61,-0.672560632228851);

\path [draw=tomato, line width=0.5pt]
(axis cs:62,0)
--(axis cs:62,-0.533866047859192);

\path [draw=tomato, line width=0.5pt]
(axis cs:63,0)
--(axis cs:63,0.319244116544724);

\addplot [semithick, tomato, mark=*, , mark size=0.5, mark options={solid}, only marks]
table {%
0 0.0827276557683945
1 -0.286112189292908
2 0.00687235593795776
3 -0.747102022171021
4 0.0641668513417244
5 -0.646571576595306
6 0.0942377746105194
7 0.701024413108826
8 0.218265861272812
9 -0.193984031677246
10 -0.128884062170982
11 0.294504374265671
12 0.597060859203339
13 0.0985148549079895
14 -0.355021268129349
15 0.451927185058594
16 0.051352247595787
17 -0.58028256893158
18 0.37768816947937
19 -0.397529572248459
20 -0.292246282100677
21 0.220617383718491
22 0.196079254150391
23 0.0956322178244591
24 0.45118436217308
25 -0.375823110342026
26 -0.551573514938354
27 0.430078268051147
28 0.279720634222031
29 0.13493886590004
30 -0.149687111377716
31 0.148303136229515
32 -0.219151318073273
33 0.43656313419342
34 -0.168442666530609
35 0.187510102987289
36 0.108915574848652
37 -0.336748242378235
38 0.2141472697258
39 -0.031447671353817
40 -0.348572671413422
41 -0.17571160197258
42 -0.250227987766266
43 -0.0833175778388977
44 0.0600676089525223
45 -0.286332786083221
46 0.0833164602518082
47 0.352049469947815
48 -0.602340042591095
49 0.355826735496521
50 -0.0874901711940765
51 0.290616244077682
52 0.196222603321075
53 0.328381299972534
54 -0.50971782207489
55 0.476836383342743
56 -0.223778992891312
57 0.101644188165665
58 0.260815739631653
59 0.323934018611908
60 -0.025722473859787
61 -0.672560632228851
62 -0.533866047859192
63 0.319244116544724
};
\addplot [semithick, black, opacity=0.5]
table {%
0 0
63 0
};
\path [draw=dodgerblue, line width=0.5pt]
(axis cs:0,0)
--(axis cs:0,-0.157364428043365);

\path [draw=dodgerblue, line width=0.5pt]
(axis cs:1,0)
--(axis cs:1,-0.427482098340988);

\path [draw=dodgerblue, line width=0.5pt]
(axis cs:2,0)
--(axis cs:2,0.142682164907455);

\path [draw=dodgerblue, line width=0.5pt]
(axis cs:3,0)
--(axis cs:3,0.13648484647274);

\path [draw=dodgerblue, line width=0.5pt]
(axis cs:4,0)
--(axis cs:4,-0.280892670154572);

\path [draw=dodgerblue, line width=0.5pt]
(axis cs:5,0)
--(axis cs:5,-0.206416189670563);

\path [draw=dodgerblue, line width=0.5pt]
(axis cs:6,0)
--(axis cs:6,0.547141075134277);

\path [draw=dodgerblue, line width=0.5pt]
(axis cs:7,0)
--(axis cs:7,0.554675102233887);

\path [draw=dodgerblue, line width=0.5pt]
(axis cs:8,0)
--(axis cs:8,0.0811909660696983);

\path [draw=dodgerblue, line width=0.5pt]
(axis cs:9,0)
--(axis cs:9,0.132868245244026);

\path [draw=dodgerblue, line width=0.5pt]
(axis cs:10,0)
--(axis cs:10,0.0273473262786865);

\path [draw=dodgerblue, line width=0.5pt]
(axis cs:11,0)
--(axis cs:11,-0.144984915852547);

\path [draw=dodgerblue, line width=0.5pt]
(axis cs:12,0)
--(axis cs:12,-0.175465166568756);

\path [draw=dodgerblue, line width=0.5pt]
(axis cs:13,0)
--(axis cs:13,-0.596456289291382);

\path [draw=dodgerblue, line width=0.5pt]
(axis cs:14,0)
--(axis cs:14,-0.645349979400635);

\path [draw=dodgerblue, line width=0.5pt]
(axis cs:15,0)
--(axis cs:15,-0.649974942207336);

\path [draw=dodgerblue, line width=0.5pt]
(axis cs:16,0)
--(axis cs:16,0.199487075209618);

\path [draw=dodgerblue, line width=0.5pt]
(axis cs:17,0)
--(axis cs:17,-0.274360179901123);

\path [draw=dodgerblue, line width=0.5pt]
(axis cs:18,0)
--(axis cs:18,0.203906893730164);

\path [draw=dodgerblue, line width=0.5pt]
(axis cs:19,0)
--(axis cs:19,0.152370482683182);

\path [draw=dodgerblue, line width=0.5pt]
(axis cs:20,0)
--(axis cs:20,0.174620300531387);

\path [draw=dodgerblue, line width=0.5pt]
(axis cs:21,0)
--(axis cs:21,0.522812962532043);

\path [draw=dodgerblue, line width=0.5pt]
(axis cs:22,0)
--(axis cs:22,0.0771784633398056);

\path [draw=dodgerblue, line width=0.5pt]
(axis cs:23,0)
--(axis cs:23,0.0874599814414978);

\path [draw=dodgerblue, line width=0.5pt]
(axis cs:24,0)
--(axis cs:24,0.614305555820465);

\path [draw=dodgerblue, line width=0.5pt]
(axis cs:25,0)
--(axis cs:25,0.12136672437191);

\path [draw=dodgerblue, line width=0.5pt]
(axis cs:26,0)
--(axis cs:26,-0.781261801719666);

\path [draw=dodgerblue, line width=0.5pt]
(axis cs:27,0)
--(axis cs:27,-0.270010888576508);

\path [draw=dodgerblue, line width=0.5pt]
(axis cs:28,0)
--(axis cs:28,-0.0632222071290016);

\path [draw=dodgerblue, line width=0.5pt]
(axis cs:29,0)
--(axis cs:29,0.0328346192836761);

\path [draw=dodgerblue, line width=0.5pt]
(axis cs:30,0)
--(axis cs:30,-0.151270389556885);

\path [draw=dodgerblue, line width=0.5pt]
(axis cs:31,0)
--(axis cs:31,0.0907290130853653);

\path [draw=dodgerblue, line width=0.5pt]
(axis cs:32,0)
--(axis cs:32,-0.102379135787487);

\path [draw=dodgerblue, line width=0.5pt]
(axis cs:33,0)
--(axis cs:33,0.172654956579208);

\path [draw=dodgerblue, line width=0.5pt]
(axis cs:34,0)
--(axis cs:34,-0.272154688835144);

\path [draw=dodgerblue, line width=0.5pt]
(axis cs:35,0)
--(axis cs:35,-0.442658245563507);

\path [draw=dodgerblue, line width=0.5pt]
(axis cs:36,0)
--(axis cs:36,-0.277874201536179);

\path [draw=dodgerblue, line width=0.5pt]
(axis cs:37,0)
--(axis cs:37,-0.601137280464172);

\path [draw=dodgerblue, line width=0.5pt]
(axis cs:38,0)
--(axis cs:38,-0.184094831347466);

\path [draw=dodgerblue, line width=0.5pt]
(axis cs:39,0)
--(axis cs:39,-0.611811995506287);

\path [draw=dodgerblue, line width=0.5pt]
(axis cs:40,0)
--(axis cs:40,0.268652617931366);

\path [draw=dodgerblue, line width=0.5pt]
(axis cs:41,0)
--(axis cs:41,0.366888105869293);

\path [draw=dodgerblue, line width=0.5pt]
(axis cs:42,0)
--(axis cs:42,-0.221810460090637);

\path [draw=dodgerblue, line width=0.5pt]
(axis cs:43,0)
--(axis cs:43,-0.288709104061127);

\path [draw=dodgerblue, line width=0.5pt]
(axis cs:44,0)
--(axis cs:44,-0.200994670391083);

\path [draw=dodgerblue, line width=0.5pt]
(axis cs:45,0)
--(axis cs:45,0.205075562000275);

\path [draw=dodgerblue, line width=0.5pt]
(axis cs:46,0)
--(axis cs:46,-0.0766182839870453);

\path [draw=dodgerblue, line width=0.5pt]
(axis cs:47,0)
--(axis cs:47,-0.244192555546761);

\path [draw=dodgerblue, line width=0.5pt]
(axis cs:48,0)
--(axis cs:48,0.0179197043180466);

\path [draw=dodgerblue, line width=0.5pt]
(axis cs:49,0)
--(axis cs:49,0.0930630937218666);

\path [draw=dodgerblue, line width=0.5pt]
(axis cs:50,0)
--(axis cs:50,-0.445823669433594);

\path [draw=dodgerblue, line width=0.5pt]
(axis cs:51,0)
--(axis cs:51,0.254653960466385);

\path [draw=dodgerblue, line width=0.5pt]
(axis cs:52,0)
--(axis cs:52,0.336896419525146);

\path [draw=dodgerblue, line width=0.5pt]
(axis cs:53,0)
--(axis cs:53,0.276561886072159);

\path [draw=dodgerblue, line width=0.5pt]
(axis cs:54,0)
--(axis cs:54,-0.0854888334870338);

\path [draw=dodgerblue, line width=0.5pt]
(axis cs:55,0)
--(axis cs:55,-0.203203588724136);

\path [draw=dodgerblue, line width=0.5pt]
(axis cs:56,0)
--(axis cs:56,-0.0140467137098312);

\path [draw=dodgerblue, line width=0.5pt]
(axis cs:57,0)
--(axis cs:57,0.209479957818985);

\path [draw=dodgerblue, line width=0.5pt]
(axis cs:58,0)
--(axis cs:58,-0.0496293902397156);

\path [draw=dodgerblue, line width=0.5pt]
(axis cs:59,0)
--(axis cs:59,1.01678144931793);

\path [draw=dodgerblue, line width=0.5pt]
(axis cs:60,0)
--(axis cs:60,0.648102879524231);

\path [draw=dodgerblue, line width=0.5pt]
(axis cs:61,0)
--(axis cs:61,-0.590502798557281);

\path [draw=dodgerblue, line width=0.5pt]
(axis cs:62,0)
--(axis cs:62,-0.0975861623883247);

\path [draw=dodgerblue, line width=0.5pt]
(axis cs:63,0)
--(axis cs:63,0.428968280553818);

\addplot [semithick, dodgerblue, mark=*, , mark size=0.5, mark options={solid}, only marks]
table {%
0 -0.157364428043365
1 -0.427482098340988
2 0.142682164907455
3 0.13648484647274
4 -0.280892670154572
5 -0.206416189670563
6 0.547141075134277
7 0.554675102233887
8 0.0811909660696983
9 0.132868245244026
10 0.0273473262786865
11 -0.144984915852547
12 -0.175465166568756
13 -0.596456289291382
14 -0.645349979400635
15 -0.649974942207336
16 0.199487075209618
17 -0.274360179901123
18 0.203906893730164
19 0.152370482683182
20 0.174620300531387
21 0.522812962532043
22 0.0771784633398056
23 0.0874599814414978
24 0.614305555820465
25 0.12136672437191
26 -0.781261801719666
27 -0.270010888576508
28 -0.0632222071290016
29 0.0328346192836761
30 -0.151270389556885
31 0.0907290130853653
32 -0.102379135787487
33 0.172654956579208
34 -0.272154688835144
35 -0.442658245563507
36 -0.277874201536179
37 -0.601137280464172
38 -0.184094831347466
39 -0.611811995506287
40 0.268652617931366
41 0.366888105869293
42 -0.221810460090637
43 -0.288709104061127
44 -0.200994670391083
45 0.205075562000275
46 -0.0766182839870453
47 -0.244192555546761
48 0.0179197043180466
49 0.0930630937218666
50 -0.445823669433594
51 0.254653960466385
52 0.336896419525146
53 0.276561886072159
54 -0.0854888334870338
55 -0.203203588724136
56 -0.0140467137098312
57 0.209479957818985
58 -0.0496293902397156
59 1.01678144931793
60 0.648102879524231
61 -0.590502798557281
62 -0.0975861623883247
63 0.428968280553818
};
\addplot [semithick, black, opacity=0.5]
table {%
0 0
63 0
};

\nextgroupplot[
width=.4\linewidth, height=3cm,
tick align=outside,
tick pos=left,
title={\(\displaystyle \small\sf Window\)},
title style={yshift=-0.21cm},
x grid style={darkgray176},
xmin=0, xmax=63,
xtick style={color=black},
xtick={-25,0,25,50,75},
ymin=-1.2, ymax=1.2,
ytick style={
    /pgfplots/major tick length=0pt,
},
xtick style={
    /pgfplots/major tick length=0pt,
},
xlabel={$\small\sf Sequence~Length$},
xlabel style={yshift=0.4cm},
ytick={},
xtick={},
xticklabels={},
yticklabels={}
]
\path [draw=tomato, line width=0.5pt]
(axis cs:0,0)
--(axis cs:0,1);

\path [draw=tomato, line width=0.5pt]
(axis cs:1,0)
--(axis cs:1,0.92950975894928);

\path [draw=tomato, line width=0.5pt]
(axis cs:2,0)
--(axis cs:2,0.86398845911026);

\path [draw=tomato, line width=0.5pt]
(axis cs:3,0)
--(axis cs:3,0.803085684776306);

\path [draw=tomato, line width=0.5pt]
(axis cs:4,0)
--(axis cs:4,0.746476054191589);

\path [draw=tomato, line width=0.5pt]
(axis cs:5,0)
--(axis cs:5,0.693856775760651);

\path [draw=tomato, line width=0.5pt]
(axis cs:6,0)
--(axis cs:6,0.64494663476944);

\path [draw=tomato, line width=0.5pt]
(axis cs:7,0)
--(axis cs:7,0.599484205245972);

\path [draw=tomato, line width=0.5pt]
(axis cs:8,0)
--(axis cs:8,0.557226479053497);

\path [draw=tomato, line width=0.5pt]
(axis cs:9,0)
--(axis cs:9,0.517947435379028);

\path [draw=tomato, line width=0.5pt]
(axis cs:10,0)
--(axis cs:10,0.481437206268311);

\path [draw=tomato, line width=0.5pt]
(axis cs:11,0)
--(axis cs:11,0.447500586509705);

\path [draw=tomato, line width=0.5pt]
(axis cs:12,0)
--(axis cs:12,0.415956169366837);

\path [draw=tomato, line width=0.5pt]
(axis cs:13,0)
--(axis cs:13,0.386635333299637);

\path [draw=tomato, line width=0.5pt]
(axis cs:14,0)
--(axis cs:14,0.359381347894669);

\path [draw=tomato, line width=0.5pt]
(axis cs:15,0)
--(axis cs:15,0.334048479795456);

\path [draw=tomato, line width=0.5pt]
(axis cs:16,0)
--(axis cs:16,0.310501337051392);

\path [draw=tomato, line width=0.5pt]
(axis cs:17,0)
--(axis cs:17,0.28861403465271);

\path [draw=tomato, line width=0.5pt]
(axis cs:18,0)
--(axis cs:18,0.268269568681717);

\path [draw=tomato, line width=0.5pt]
(axis cs:19,0)
--(axis cs:19,0.249359175562859);

\path [draw=tomato, line width=0.5pt]
(axis cs:20,0)
--(axis cs:20,0.231781795620918);

\path [draw=tomato, line width=0.5pt]
(axis cs:21,0)
--(axis cs:21,0.215443447232246);

\path [draw=tomato, line width=0.5pt]
(axis cs:22,0)
--(axis cs:22,0.200256764888763);

\path [draw=tomato, line width=0.5pt]
(axis cs:23,0)
--(axis cs:23,0.18614062666893);

\path [draw=tomato, line width=0.5pt]
(axis cs:24,0)
--(axis cs:24,0.173019528388977);

\path [draw=tomato, line width=0.5pt]
(axis cs:25,0)
--(axis cs:25,0.160823345184326);

\path [draw=tomato, line width=0.5pt]
(axis cs:26,0)
--(axis cs:26,0.149486869573593);

\path [draw=tomato, line width=0.5pt]
(axis cs:27,0)
--(axis cs:27,0.138949528336525);

\path [draw=tomato, line width=0.5pt]
(axis cs:28,0)
--(axis cs:28,0.129154950380325);

\path [draw=tomato, line width=0.5pt]
(axis cs:29,0)
--(axis cs:29,0.12005078792572);

\path [draw=tomato, line width=0.5pt]
(axis cs:30,0)
--(axis cs:30,0.111588381230831);

\path [draw=tomato, line width=0.5pt]
(axis cs:31,0)
--(axis cs:31,0.103722497820854);

\path [draw=tomato, line width=0.5pt]
(axis cs:32,0)
--(axis cs:32,0.0964110940694809);

\path [draw=tomato, line width=0.5pt]
(axis cs:33,0)
--(axis cs:33,0.0896150544285774);

\path [draw=tomato, line width=0.5pt]
(axis cs:34,0)
--(axis cs:34,0.083298072218895);

\path [draw=tomato, line width=0.5pt]
(axis cs:35,0)
--(axis cs:35,0.0774263739585876);

\path [draw=tomato, line width=0.5pt]
(axis cs:36,0)
--(axis cs:36,0.0719685703516006);

\path [draw=tomato, line width=0.5pt]
(axis cs:37,0)
--(axis cs:37,0.066895492374897);

\path [draw=tomato, line width=0.5pt]
(axis cs:38,0)
--(axis cs:38,0.0621800161898136);

\path [draw=tomato, line width=0.5pt]
(axis cs:39,0)
--(axis cs:39,0.0577969327569008);

\path [draw=tomato, line width=0.5pt]
(axis cs:40,0)
--(axis cs:40,0.0537228137254715);

\path [draw=tomato, line width=0.5pt]
(axis cs:41,0)
--(axis cs:41,0.0499358810484409);

\path [draw=tomato, line width=0.5pt]
(axis cs:42,0)
--(axis cs:42,0.0464158914983273);

\path [draw=tomato, line width=0.5pt]
(axis cs:43,0)
--(axis cs:43,0.0431440249085426);

\path [draw=tomato, line width=0.5pt]
(axis cs:44,0)
--(axis cs:44,0.0401028022170067);

\path [draw=tomato, line width=0.5pt]
(axis cs:45,0)
--(axis cs:45,0.0372759476304054);

\path [draw=tomato, line width=0.5pt]
(axis cs:46,0)
--(axis cs:46,0.0346483588218689);

\path [draw=tomato, line width=0.5pt]
(axis cs:47,0)
--(axis cs:47,0.0322059877216816);

\path [draw=tomato, line width=0.5pt]
(axis cs:48,0)
--(axis cs:48,0.0299357660114765);

\path [draw=tomato, line width=0.5pt]
(axis cs:49,0)
--(axis cs:49,0.0278255939483643);

\path [draw=tomato, line width=0.5pt]
(axis cs:50,0)
--(axis cs:50,0.0258641615509987);

\path [draw=tomato, line width=0.5pt]
(axis cs:51,0)
--(axis cs:51,0.0240409914404154);

\path [draw=tomato, line width=0.5pt]
(axis cs:52,0)
--(axis cs:52,0.0223463363945484);

\path [draw=tomato, line width=0.5pt]
(axis cs:53,0)
--(axis cs:53,0.0207711383700371);

\path [draw=tomato, line width=0.5pt]
(axis cs:54,0)
--(axis cs:54,0.0193069763481617);

\path [draw=tomato, line width=0.5pt]
(axis cs:55,0)
--(axis cs:55,0.0179460234940052);

\path [draw=tomato, line width=0.5pt]
(axis cs:56,0)
--(axis cs:56,0.0166810005903244);

\path [draw=tomato, line width=0.5pt]
(axis cs:57,0)
--(axis cs:57,0.0155051574110985);

\path [draw=tomato, line width=0.5pt]
(axis cs:58,0)
--(axis cs:58,0.014412191696465);

\path [draw=tomato, line width=0.5pt]
(axis cs:59,0)
--(axis cs:59,0.0133962770923972);

\path [draw=tomato, line width=0.5pt]
(axis cs:60,0)
--(axis cs:60,0.0124519672244787);

\path [draw=tomato, line width=0.5pt]
(axis cs:61,0)
--(axis cs:61,0.0115742282941937);

\path [draw=tomato, line width=0.5pt]
(axis cs:62,0)
--(axis cs:62,0.0107583561912179);

\path [draw=tomato, line width=0.5pt]
(axis cs:63,0)
--(axis cs:63,0.00999999977648258);

\addplot [semithick, tomato, mark=*, , mark size=0.5, mark options={solid}, only marks]
table {%
0 1
1 0.92950975894928
2 0.86398845911026
3 0.803085684776306
4 0.746476054191589
5 0.693856775760651
6 0.64494663476944
7 0.599484205245972
8 0.557226479053497
9 0.517947435379028
10 0.481437206268311
11 0.447500586509705
12 0.415956169366837
13 0.386635333299637
14 0.359381347894669
15 0.334048479795456
16 0.310501337051392
17 0.28861403465271
18 0.268269568681717
19 0.249359175562859
20 0.231781795620918
21 0.215443447232246
22 0.200256764888763
23 0.18614062666893
24 0.173019528388977
25 0.160823345184326
26 0.149486869573593
27 0.138949528336525
28 0.129154950380325
29 0.12005078792572
30 0.111588381230831
31 0.103722497820854
32 0.0964110940694809
33 0.0896150544285774
34 0.083298072218895
35 0.0774263739585876
36 0.0719685703516006
37 0.066895492374897
38 0.0621800161898136
39 0.0577969327569008
40 0.0537228137254715
41 0.0499358810484409
42 0.0464158914983273
43 0.0431440249085426
44 0.0401028022170067
45 0.0372759476304054
46 0.0346483588218689
47 0.0322059877216816
48 0.0299357660114765
49 0.0278255939483643
50 0.0258641615509987
51 0.0240409914404154
52 0.0223463363945484
53 0.0207711383700371
54 0.0193069763481617
55 0.0179460234940052
56 0.0166810005903244
57 0.0155051574110985
58 0.014412191696465
59 0.0133962770923972
60 0.0124519672244787
61 0.0115742282941937
62 0.0107583561912179
63 0.00999999977648258
};
\addplot [semithick, black, opacity=0.5]
table {%
0 0
63 0
};
\path [draw=dodgerblue, line width=0.5pt]
(axis cs:0,0)
--(axis cs:0,1);

\path [draw=dodgerblue, line width=0.5pt]
(axis cs:1,0)
--(axis cs:1,0.783754229545593);

\path [draw=dodgerblue, line width=0.5pt]
(axis cs:2,0)
--(axis cs:2,0.614270687103271);

\path [draw=dodgerblue, line width=0.5pt]
(axis cs:3,0)
--(axis cs:3,0.481437206268311);

\path [draw=dodgerblue, line width=0.5pt]
(axis cs:4,0)
--(axis cs:4,0.377328455448151);

\path [draw=dodgerblue, line width=0.5pt]
(axis cs:5,0)
--(axis cs:5,0.295732766389847);

\path [draw=dodgerblue, line width=0.5pt]
(axis cs:6,0)
--(axis cs:6,0.231781795620918);

\path [draw=dodgerblue, line width=0.5pt]
(axis cs:7,0)
--(axis cs:7,0.18165996670723);

\path [draw=dodgerblue, line width=0.5pt]
(axis cs:8,0)
--(axis cs:8,0.142376765608788);

\path [draw=dodgerblue, line width=0.5pt]
(axis cs:9,0)
--(axis cs:9,0.111588381230831);

\path [draw=dodgerblue, line width=0.5pt]
(axis cs:10,0)
--(axis cs:10,0.0874578729271889);

\path [draw=dodgerblue, line width=0.5pt]
(axis cs:11,0)
--(axis cs:11,0.0685454681515694);

\path [draw=dodgerblue, line width=0.5pt]
(axis cs:12,0)
--(axis cs:12,0.0537228025496006);

\path [draw=dodgerblue, line width=0.5pt]
(axis cs:13,0)
--(axis cs:13,0.0421054735779762);

\path [draw=dodgerblue, line width=0.5pt]
(axis cs:14,0)
--(axis cs:14,0.0330003462731838);

\path [draw=dodgerblue, line width=0.5pt]
(axis cs:15,0)
--(axis cs:15,0.0258641559630632);

\path [draw=dodgerblue, line width=0.5pt]
(axis cs:16,0)
--(axis cs:16,0.0202711429446936);

\path [draw=dodgerblue, line width=0.5pt]
(axis cs:17,0)
--(axis cs:17,0.0158875938504934);

\path [draw=dodgerblue, line width=0.5pt]
(axis cs:18,0)
--(axis cs:18,0.0124519672244787);

\path [draw=dodgerblue, line width=0.5pt]
(axis cs:19,0)
--(axis cs:19,0.00975928455591202);

\path [draw=dodgerblue, line width=0.5pt]
(axis cs:20,0)
--(axis cs:20,0.00764887919649482);

\path [draw=dodgerblue, line width=0.5pt]
(axis cs:21,0)
--(axis cs:21,0.00599484331905842);

\path [draw=dodgerblue, line width=0.5pt]
(axis cs:22,0)
--(axis cs:22,0.00469848094508052);

\path [draw=dodgerblue, line width=0.5pt]
(axis cs:23,0)
--(axis cs:23,0.00368245528079569);

\path [draw=dodgerblue, line width=0.5pt]
(axis cs:24,0)
--(axis cs:24,0.00288613932207227);

\path [draw=dodgerblue, line width=0.5pt]
(axis cs:25,0)
--(axis cs:25,0.00226202351041138);

\path [draw=dodgerblue, line width=0.5pt]
(axis cs:26,0)
--(axis cs:26,0.00177287100814283);

\path [draw=dodgerblue, line width=0.5pt]
(axis cs:27,0)
--(axis cs:27,0.00138949486427009);

\path [draw=dodgerblue, line width=0.5pt]
(axis cs:28,0)
--(axis cs:28,0.00108902284409851);

\path [draw=dodgerblue, line width=0.5pt]
(axis cs:29,0)
--(axis cs:29,0.000853526056744158);

\path [draw=dodgerblue, line width=0.5pt]
(axis cs:30,0)
--(axis cs:30,0.000668954569846392);

\path [draw=dodgerblue, line width=0.5pt]
(axis cs:31,0)
--(axis cs:31,0.000524296134244651);

\path [draw=dodgerblue, line width=0.5pt]
(axis cs:32,0)
--(axis cs:32,0.000410919630667195);

\path [draw=dodgerblue, line width=0.5pt]
(axis cs:33,0)
--(axis cs:33,0.000322060077451169);

\path [draw=dodgerblue, line width=0.5pt]
(axis cs:34,0)
--(axis cs:34,0.000252415891736746);

\path [draw=dodgerblue, line width=0.5pt]
(axis cs:35,0)
--(axis cs:35,0.000197831992409192);

\path [draw=dodgerblue, line width=0.5pt]
(axis cs:36,0)
--(axis cs:36,0.000155051631736569);

\path [draw=dodgerblue, line width=0.5pt]
(axis cs:37,0)
--(axis cs:37,0.00012152246927144);

\path [draw=dodgerblue, line width=0.5pt]
(axis cs:38,0)
--(axis cs:38,9.52437330852263e-05);

\path [draw=dodgerblue, line width=0.5pt]
(axis cs:39,0)
--(axis cs:39,7.4647665314842e-05);

\path [draw=dodgerblue, line width=0.5pt]
(axis cs:40,0)
--(axis cs:40,5.85054112889338e-05);

\path [draw=dodgerblue, line width=0.5pt]
(axis cs:41,0)
--(axis cs:41,4.58538561360911e-05);

\path [draw=dodgerblue, line width=0.5pt]
(axis cs:42,0)
--(axis cs:42,3.59381810994819e-05);

\path [draw=dodgerblue, line width=0.5pt]
(axis cs:43,0)
--(axis cs:43,2.81666962109739e-05);

\path [draw=dodgerblue, line width=0.5pt]
(axis cs:44,0)
--(axis cs:44,2.20757628994761e-05);

\path [draw=dodgerblue, line width=0.5pt]
(axis cs:45,0)
--(axis cs:45,1.73019689100329e-05);

\path [draw=dodgerblue, line width=0.5pt]
(axis cs:46,0)
--(axis cs:46,1.35604896058794e-05);

\path [draw=dodgerblue, line width=0.5pt]
(axis cs:47,0)
--(axis cs:47,1.06280995169072e-05);

\path [draw=dodgerblue, line width=0.5pt]
(axis cs:48,0)
--(axis cs:48,8.32980822451646e-06);

\path [draw=dodgerblue, line width=0.5pt]
(axis cs:49,0)
--(axis cs:49,6.52852122584591e-06);

\path [draw=dodgerblue, line width=0.5pt]
(axis cs:50,0)
--(axis cs:50,5.11675534653477e-06);

\path [draw=dodgerblue, line width=0.5pt]
(axis cs:51,0)
--(axis cs:51,4.01028182750451e-06);

\path [draw=dodgerblue, line width=0.5pt]
(axis cs:52,0)
--(axis cs:52,3.14307453663787e-06);

\path [draw=dodgerblue, line width=0.5pt]
(axis cs:53,0)
--(axis cs:53,2.46339754994551e-06);

\path [draw=dodgerblue, line width=0.5pt]
(axis cs:54,0)
--(axis cs:54,1.93069786291744e-06);

\path [draw=dodgerblue, line width=0.5pt]
(axis cs:55,0)
--(axis cs:55,1.51319238739234e-06);

\path [draw=dodgerblue, line width=0.5pt]
(axis cs:56,0)
--(axis cs:56,1.18597188247804e-06);

\path [draw=dodgerblue, line width=0.5pt]
(axis cs:57,0)
--(axis cs:57,9.29510292735358e-07);

\path [draw=dodgerblue, line width=0.5pt]
(axis cs:58,0)
--(axis cs:58,7.28507473013451e-07);

\path [draw=dodgerblue, line width=0.5pt]
(axis cs:59,0)
--(axis cs:59,5.70970712487906e-07);

\path [draw=dodgerblue, line width=0.5pt]
(axis cs:60,0)
--(axis cs:60,4.47500639211285e-07);

\path [draw=dodgerblue, line width=0.5pt]
(axis cs:61,0)
--(axis cs:61,3.50730772424868e-07);

\path [draw=dodgerblue, line width=0.5pt]
(axis cs:62,0)
--(axis cs:62,2.7488667342368e-07);

\path [draw=dodgerblue, line width=0.5pt]
(axis cs:63,0)
--(axis cs:63,2.15443563433837e-07);

\addplot [semithick, dodgerblue, mark=*, , mark size=0.5, mark options={solid}, only marks]
table {%
0 1
1 0.783754229545593
2 0.614270687103271
3 0.481437206268311
4 0.377328455448151
5 0.295732766389847
6 0.231781795620918
7 0.18165996670723
8 0.142376765608788
9 0.111588381230831
10 0.0874578729271889
11 0.0685454681515694
12 0.0537228025496006
13 0.0421054735779762
14 0.0330003462731838
15 0.0258641559630632
16 0.0202711429446936
17 0.0158875938504934
18 0.0124519672244787
19 0.00975928455591202
20 0.00764887919649482
21 0.00599484331905842
22 0.00469848094508052
23 0.00368245528079569
24 0.00288613932207227
25 0.00226202351041138
26 0.00177287100814283
27 0.00138949486427009
28 0.00108902284409851
29 0.000853526056744158
30 0.000668954569846392
31 0.000524296134244651
32 0.000410919630667195
33 0.000322060077451169
34 0.000252415891736746
35 0.000197831992409192
36 0.000155051631736569
37 0.00012152246927144
38 9.52437330852263e-05
39 7.4647665314842e-05
40 5.85054112889338e-05
41 4.58538561360911e-05
42 3.59381810994819e-05
43 2.81666962109739e-05
44 2.20757628994761e-05
45 1.73019689100329e-05
46 1.35604896058794e-05
47 1.06280995169072e-05
48 8.32980822451646e-06
49 6.52852122584591e-06
50 5.11675534653477e-06
51 4.01028182750451e-06
52 3.14307453663787e-06
53 2.46339754994551e-06
54 1.93069786291744e-06
55 1.51319238739234e-06
56 1.18597188247804e-06
57 9.29510292735358e-07
58 7.28507473013451e-07
59 5.70970712487906e-07
60 4.47500639211285e-07
61 3.50730772424868e-07
62 2.7488667342368e-07
63 2.15443563433837e-07
};
\addplot [semithick, black, opacity=0.5]
table {%
0 0
63 0
};

\nextgroupplot[
width=.4\linewidth, height=3cm,
tick align=outside,
tick pos=left,
title={\(\displaystyle {\small\sf Window} \circ {\small\sf FFN}(t)\)},
title style={yshift=-0.3cm},
x grid style={darkgray176},
xmin=0, xmax=63,
xtick style={color=black},
xtick={-25,0,25,50,75},
xticklabels={
  \(\displaystyle {\ensuremath{-}25}\),
  \(\displaystyle {0}\),
  \(\displaystyle {25}\),
  \(\displaystyle {50}\),
  \(\displaystyle {75}\)
},
xlabel={$\small\sf Sequence~Length$},
xlabel style={yshift=0.4cm},
y grid style={darkgray176},
ymin=-0.5, ymax=0.5,
ytick style={
    /pgfplots/major tick length=0pt,
},
xtick style={
    /pgfplots/major tick length=0pt,
},
ytick={},
xtick={},
xticklabels={},
yticklabels={}
]
\path [draw=tomato, line width=0.5pt]
(axis cs:0,0)
--(axis cs:0,0.0827276557683945);

\path [draw=tomato, line width=0.5pt]
(axis cs:1,0)
--(axis cs:1,-0.265944063663483);

\path [draw=tomato, line width=0.5pt]
(axis cs:2,0)
--(axis cs:2,0.00593763636425138);

\path [draw=tomato, line width=0.5pt]
(axis cs:3,0)
--(axis cs:3,-0.599986910820007);

\path [draw=tomato, line width=0.5pt]
(axis cs:4,0)
--(axis cs:4,0.0478990189731121);

\path [draw=tomato, line width=0.5pt]
(axis cs:5,0)
--(axis cs:5,-0.448628067970276);

\path [draw=tomato, line width=0.5pt]
(axis cs:6,0)
--(axis cs:6,0.060778334736824);

\path [draw=tomato, line width=0.5pt]
(axis cs:7,0)
--(axis cs:7,0.420253068208694);

\path [draw=tomato, line width=0.5pt]
(axis cs:8,0)
--(axis cs:8,0.121623516082764);

\path [draw=tomato, line width=0.5pt]
(axis cs:9,0)
--(axis cs:9,-0.100473530590534);

\path [draw=tomato, line width=0.5pt]
(axis cs:10,0)
--(axis cs:10,-0.0620495826005936);

\path [draw=tomato, line width=0.5pt]
(axis cs:11,0)
--(axis cs:11,0.13179087638855);

\path [draw=tomato, line width=0.5pt]
(axis cs:12,0)
--(axis cs:12,0.248351141810417);

\path [draw=tomato, line width=0.5pt]
(axis cs:13,0)
--(axis cs:13,0.0380893237888813);

\path [draw=tomato, line width=0.5pt]
(axis cs:14,0)
--(axis cs:14,-0.127588018774986);

\path [draw=tomato, line width=0.5pt]
(axis cs:15,0)
--(axis cs:15,0.150965586304665);

\path [draw=tomato, line width=0.5pt]
(axis cs:16,0)
--(axis cs:16,0.015944940969348);

\path [draw=tomato, line width=0.5pt]
(axis cs:17,0)
--(axis cs:17,-0.167477697134018);

\path [draw=tomato, line width=0.5pt]
(axis cs:18,0)
--(axis cs:18,0.101322241127491);

\path [draw=tomato, line width=0.5pt]
(axis cs:19,0)
--(axis cs:19,-0.0991276428103447);

\path [draw=tomato, line width=0.5pt]
(axis cs:20,0)
--(axis cs:20,-0.0677373707294464);

\path [draw=tomato, line width=0.5pt]
(axis cs:21,0)
--(axis cs:21,0.0475305691361427);

\path [draw=tomato, line width=0.5pt]
(axis cs:22,0)
--(axis cs:22,0.0392661988735199);

\path [draw=tomato, line width=0.5pt]
(axis cs:23,0)
--(axis cs:23,0.0178010407835245);

\path [draw=tomato, line width=0.5pt]
(axis cs:24,0)
--(axis cs:24,0.0780637040734291);

\path [draw=tomato, line width=0.5pt]
(axis cs:25,0)
--(axis cs:25,-0.0604411289095879);

\path [draw=tomato, line width=0.5pt]
(axis cs:26,0)
--(axis cs:26,-0.0824529975652695);

\path [draw=tomato, line width=0.5pt]
(axis cs:27,0)
--(axis cs:27,0.0597591735422611);

\path [draw=tomato, line width=0.5pt]
(axis cs:28,0)
--(axis cs:28,0.0361273027956486);

\path [draw=tomato, line width=0.5pt]
(axis cs:29,0)
--(axis cs:29,0.0161995179951191);

\path [draw=tomato, line width=0.5pt]
(axis cs:30,0)
--(axis cs:30,-0.0167033430188894);

\path [draw=tomato, line width=0.5pt]
(axis cs:31,0)
--(axis cs:31,0.0153823718428612);

\path [draw=tomato, line width=0.5pt]
(axis cs:32,0)
--(axis cs:32,-0.0211286190897226);

\path [draw=tomato, line width=0.5pt]
(axis cs:33,0)
--(axis cs:33,0.0391226299107075);

\path [draw=tomato, line width=0.5pt]
(axis cs:34,0)
--(axis cs:34,-0.0140309492126107);

\path [draw=tomato, line width=0.5pt]
(axis cs:35,0)
--(axis cs:35,0.0145182274281979);

\path [draw=tomato, line width=0.5pt]
(axis cs:36,0)
--(axis cs:36,0.00783849786967039);

\path [draw=tomato, line width=0.5pt]
(axis cs:37,0)
--(axis cs:37,-0.022526940330863);

\path [draw=tomato, line width=0.5pt]
(axis cs:38,0)
--(axis cs:38,0.01331568043679);

\path [draw=tomato, line width=0.5pt]
(axis cs:39,0)
--(axis cs:39,-0.0018175789155066);

\path [draw=tomato, line width=0.5pt]
(axis cs:40,0)
--(axis cs:40,-0.0187263041734695);

\path [draw=tomato, line width=0.5pt]
(axis cs:41,0)
--(axis cs:41,-0.00877431407570839);

\path [draw=tomato, line width=0.5pt]
(axis cs:42,0)
--(axis cs:42,-0.0116145554929972);

\path [draw=tomato, line width=0.5pt]
(axis cs:43,0)
--(axis cs:43,-0.00359465554356575);

\path [draw=tomato, line width=0.5pt]
(axis cs:44,0)
--(axis cs:44,0.00240887934342027);

\path [draw=tomato, line width=0.5pt]
(axis cs:45,0)
--(axis cs:45,-0.0106733255088329);

\path [draw=tomato, line width=0.5pt]
(axis cs:46,0)
--(axis cs:46,0.00288677867501974);

\path [draw=tomato, line width=0.5pt]
(axis cs:47,0)
--(axis cs:47,0.0113381007686257);

\path [draw=tomato, line width=0.5pt]
(axis cs:48,0)
--(axis cs:48,-0.0180315114557743);

\path [draw=tomato, line width=0.5pt]
(axis cs:49,0)
--(axis cs:49,0.00990109052509069);

\path [draw=tomato, line width=0.5pt]
(axis cs:50,0)
--(axis cs:50,-0.00226285983808339);

\path [draw=tomato, line width=0.5pt]
(axis cs:51,0)
--(axis cs:51,0.00698670279234648);

\path [draw=tomato, line width=0.5pt]
(axis cs:52,0)
--(axis cs:52,0.00438485620543361);

\path [draw=tomato, line width=0.5pt]
(axis cs:53,0)
--(axis cs:53,0.00682085333392024);

\path [draw=tomato, line width=0.5pt]
(axis cs:54,0)
--(axis cs:54,-0.00984110962599516);

\path [draw=tomato, line width=0.5pt]
(axis cs:55,0)
--(axis cs:55,0.00855731684714556);

\path [draw=tomato, line width=0.5pt]
(axis cs:56,0)
--(axis cs:56,-0.00373285752721131);

\path [draw=tomato, line width=0.5pt]
(axis cs:57,0)
--(axis cs:57,0.00157600909005851);

\path [draw=tomato, line width=0.5pt]
(axis cs:58,0)
--(axis cs:58,0.00375892641022801);

\path [draw=tomato, line width=0.5pt]
(axis cs:59,0)
--(axis cs:59,0.00433950964361429);

\path [draw=tomato, line width=0.5pt]
(axis cs:60,0)
--(axis cs:60,-0.000320295395795256);

\path [draw=tomato, line width=0.5pt]
(axis cs:61,0)
--(axis cs:61,-0.00778437033295631);

\path [draw=tomato, line width=0.5pt]
(axis cs:62,0)
--(axis cs:62,-0.00574352126568556);

\path [draw=tomato, line width=0.5pt]
(axis cs:63,0)
--(axis cs:63,0.00319244107231498);

\addplot [semithick, tomato, mark=*, , mark size=0.5, mark options={solid}, only marks]
table {%
0 0.0827276557683945
1 -0.265944063663483
2 0.00593763636425138
3 -0.599986910820007
4 0.0478990189731121
5 -0.448628067970276
6 0.060778334736824
7 0.420253068208694
8 0.121623516082764
9 -0.100473530590534
10 -0.0620495826005936
11 0.13179087638855
12 0.248351141810417
13 0.0380893237888813
14 -0.127588018774986
15 0.150965586304665
16 0.015944940969348
17 -0.167477697134018
18 0.101322241127491
19 -0.0991276428103447
20 -0.0677373707294464
21 0.0475305691361427
22 0.0392661988735199
23 0.0178010407835245
24 0.0780637040734291
25 -0.0604411289095879
26 -0.0824529975652695
27 0.0597591735422611
28 0.0361273027956486
29 0.0161995179951191
30 -0.0167033430188894
31 0.0153823718428612
32 -0.0211286190897226
33 0.0391226299107075
34 -0.0140309492126107
35 0.0145182274281979
36 0.00783849786967039
37 -0.022526940330863
38 0.01331568043679
39 -0.0018175789155066
40 -0.0187263041734695
41 -0.00877431407570839
42 -0.0116145554929972
43 -0.00359465554356575
44 0.00240887934342027
45 -0.0106733255088329
46 0.00288677867501974
47 0.0113381007686257
48 -0.0180315114557743
49 0.00990109052509069
50 -0.00226285983808339
51 0.00698670279234648
52 0.00438485620543361
53 0.00682085333392024
54 -0.00984110962599516
55 0.00855731684714556
56 -0.00373285752721131
57 0.00157600909005851
58 0.00375892641022801
59 0.00433950964361429
60 -0.000320295395795256
61 -0.00778437033295631
62 -0.00574352126568556
63 0.00319244107231498
};
\addplot [semithick, black, opacity=0.5]
table {%
0 0
63 0
};
\path [draw=dodgerblue, line width=0.5pt]
(axis cs:0,0)
--(axis cs:0,-0.157364428043365);

\path [draw=dodgerblue, line width=0.5pt]
(axis cs:1,0)
--(axis cs:1,-0.335040897130966);

\path [draw=dodgerblue, line width=0.5pt]
(axis cs:2,0)
--(axis cs:2,0.0876454710960388);

\path [draw=dodgerblue, line width=0.5pt]
(axis cs:3,0)
--(axis cs:3,0.0657088831067085);

\path [draw=dodgerblue, line width=0.5pt]
(axis cs:4,0)
--(axis cs:4,-0.105988800525665);

\path [draw=dodgerblue, line width=0.5pt]
(axis cs:5,0)
--(axis cs:5,-0.0610440298914909);

\path [draw=dodgerblue, line width=0.5pt]
(axis cs:6,0)
--(axis cs:6,0.126817345619202);

\path [draw=dodgerblue, line width=0.5pt]
(axis cs:7,0)
--(axis cs:7,0.100762262940407);

\path [draw=dodgerblue, line width=0.5pt]
(axis cs:8,0)
--(axis cs:8,0.0115597071126103);

\path [draw=dodgerblue, line width=0.5pt]
(axis cs:9,0)
--(axis cs:9,0.0148265520110726);

\path [draw=dodgerblue, line width=0.5pt]
(axis cs:10,0)
--(axis cs:10,0.0023917390499264);

\path [draw=dodgerblue, line width=0.5pt]
(axis cs:11,0)
--(axis cs:11,-0.00993805937469006);

\path [draw=dodgerblue, line width=0.5pt]
(axis cs:12,0)
--(axis cs:12,-0.00942648015916348);

\path [draw=dodgerblue, line width=0.5pt]
(axis cs:13,0)
--(axis cs:13,-0.0251140743494034);

\path [draw=dodgerblue, line width=0.5pt]
(axis cs:14,0)
--(axis cs:14,-0.0212967731058598);

\path [draw=dodgerblue, line width=0.5pt]
(axis cs:15,0)
--(axis cs:15,-0.016811054199934);

\path [draw=dodgerblue, line width=0.5pt]
(axis cs:16,0)
--(axis cs:16,0.00404383102431893);

\path [draw=dodgerblue, line width=0.5pt]
(axis cs:17,0)
--(axis cs:17,-0.00435892306268215);

\path [draw=dodgerblue, line width=0.5pt]
(axis cs:18,0)
--(axis cs:18,0.0025390419177711);

\path [draw=dodgerblue, line width=0.5pt]
(axis cs:19,0)
--(axis cs:19,0.00148702692240477);

\path [draw=dodgerblue, line width=0.5pt]
(axis cs:20,0)
--(axis cs:20,0.00133564963471144);

\path [draw=dodgerblue, line width=0.5pt]
(axis cs:21,0)
--(axis cs:21,0.00313418172299862);

\path [draw=dodgerblue, line width=0.5pt]
(axis cs:22,0)
--(axis cs:22,0.000362621532985941);

\path [draw=dodgerblue, line width=0.5pt]
(axis cs:23,0)
--(axis cs:23,0.000322067469824106);

\path [draw=dodgerblue, line width=0.5pt]
(axis cs:24,0)
--(axis cs:24,0.00177297147456557);

\path [draw=dodgerblue, line width=0.5pt]
(axis cs:25,0)
--(axis cs:25,0.000274534395430237);

\path [draw=dodgerblue, line width=0.5pt]
(axis cs:26,0)
--(axis cs:26,-0.00138507643714547);

\path [draw=dodgerblue, line width=0.5pt]
(axis cs:27,0)
--(axis cs:27,-0.000375178729882464);

\path [draw=dodgerblue, line width=0.5pt]
(axis cs:28,0)
--(axis cs:28,-6.88504296704195e-05);

\path [draw=dodgerblue, line width=0.5pt]
(axis cs:29,0)
--(axis cs:29,2.80252024822403e-05);

\path [draw=dodgerblue, line width=0.5pt]
(axis cs:30,0)
--(axis cs:30,-0.00010119302169187);

\path [draw=dodgerblue, line width=0.5pt]
(axis cs:31,0)
--(axis cs:31,4.75688721053302e-05);

\path [draw=dodgerblue, line width=0.5pt]
(axis cs:32,0)
--(axis cs:32,-4.2069597839145e-05);

\path [draw=dodgerblue, line width=0.5pt]
(axis cs:33,0)
--(axis cs:33,5.56052691536024e-05);

\path [draw=dodgerblue, line width=0.5pt]
(axis cs:34,0)
--(axis cs:34,-6.86961720930412e-05);

\path [draw=dodgerblue, line width=0.5pt]
(axis cs:35,0)
--(axis cs:35,-8.75719633768313e-05);

\path [draw=dodgerblue, line width=0.5pt]
(axis cs:36,0)
--(axis cs:36,-4.30848485848401e-05);

\path [draw=dodgerblue, line width=0.5pt]
(axis cs:37,0)
--(axis cs:37,-7.30516840121709e-05);

\path [draw=dodgerblue, line width=0.5pt]
(axis cs:38,0)
--(axis cs:38,-1.75338791450486e-05);

\path [draw=dodgerblue, line width=0.5pt]
(axis cs:39,0)
--(axis cs:39,-4.56703382951673e-05);

\path [draw=dodgerblue, line width=0.5pt]
(axis cs:40,0)
--(axis cs:40,1.57176327775232e-05);

\path [draw=dodgerblue, line width=0.5pt]
(axis cs:41,0)
--(axis cs:41,1.68232345458819e-05);

\path [draw=dodgerblue, line width=0.5pt]
(axis cs:42,0)
--(axis cs:42,-7.97146458353382e-06);

\path [draw=dodgerblue, line width=0.5pt]
(axis cs:43,0)
--(axis cs:43,-8.13198130344972e-06);

\path [draw=dodgerblue, line width=0.5pt]
(axis cs:44,0)
--(axis cs:44,-4.43711087427801e-06);

\path [draw=dodgerblue, line width=0.5pt]
(axis cs:45,0)
--(axis cs:45,3.5482109979057e-06);

\path [draw=dodgerblue, line width=0.5pt]
(axis cs:46,0)
--(axis cs:46,-1.03898139514058e-06);

\path [draw=dodgerblue, line width=0.5pt]
(axis cs:47,0)
--(axis cs:47,-2.59530270341202e-06);

\path [draw=dodgerblue, line width=0.5pt]
(axis cs:48,0)
--(axis cs:48,1.49267705751299e-07);

\path [draw=dodgerblue, line width=0.5pt]
(axis cs:49,0)
--(axis cs:49,6.07564402343996e-07);

\path [draw=dodgerblue, line width=0.5pt]
(axis cs:50,0)
--(axis cs:50,-2.28117073675094e-06);

\path [draw=dodgerblue, line width=0.5pt]
(axis cs:51,0)
--(axis cs:51,1.02123419765121e-06);

\path [draw=dodgerblue, line width=0.5pt]
(axis cs:52,0)
--(axis cs:52,1.0588905752229e-06);

\path [draw=dodgerblue, line width=0.5pt]
(axis cs:53,0)
--(axis cs:53,6.81281846937054e-07);

\path [draw=dodgerblue, line width=0.5pt]
(axis cs:54,0)
--(axis cs:54,-1.65053108958091e-07);

\path [draw=dodgerblue, line width=0.5pt]
(axis cs:55,0)
--(axis cs:55,-3.07486118344968e-07);

\path [draw=dodgerblue, line width=0.5pt]
(axis cs:56,0)
--(axis cs:56,-1.66590066186245e-08);

\path [draw=dodgerblue, line width=0.5pt]
(axis cs:57,0)
--(axis cs:57,1.94713777545985e-07);

\path [draw=dodgerblue, line width=0.5pt]
(axis cs:58,0)
--(axis cs:58,-3.61553809113957e-08);

\path [draw=dodgerblue, line width=0.5pt]
(axis cs:59,0)
--(axis cs:59,5.8055240970134e-07);

\path [draw=dodgerblue, line width=0.5pt]
(axis cs:60,0)
--(axis cs:60,2.90026463289905e-07);

\path [draw=dodgerblue, line width=0.5pt]
(axis cs:61,0)
--(axis cs:61,-2.07107504479609e-07);

\path [draw=dodgerblue, line width=0.5pt]
(axis cs:62,0)
--(axis cs:62,-2.68251358903626e-08);

\path [draw=dodgerblue, line width=0.5pt]
(axis cs:63,0)
--(axis cs:63,9.24184533346306e-08);

\addplot [semithick, dodgerblue, mark=*, , mark size=0.5, mark options={solid}, only marks]
table {%
0 -0.157364428043365
1 -0.335040897130966
2 0.0876454710960388
3 0.0657088831067085
4 -0.105988800525665
5 -0.0610440298914909
6 0.126817345619202
7 0.100762262940407
8 0.0115597071126103
9 0.0148265520110726
10 0.0023917390499264
11 -0.00993805937469006
12 -0.00942648015916348
13 -0.0251140743494034
14 -0.0212967731058598
15 -0.016811054199934
16 0.00404383102431893
17 -0.00435892306268215
18 0.0025390419177711
19 0.00148702692240477
20 0.00133564963471144
21 0.00313418172299862
22 0.000362621532985941
23 0.000322067469824106
24 0.00177297147456557
25 0.000274534395430237
26 -0.00138507643714547
27 -0.000375178729882464
28 -6.88504296704195e-05
29 2.80252024822403e-05
30 -0.00010119302169187
31 4.75688721053302e-05
32 -4.2069597839145e-05
33 5.56052691536024e-05
34 -6.86961720930412e-05
35 -8.75719633768313e-05
36 -4.30848485848401e-05
37 -7.30516840121709e-05
38 -1.75338791450486e-05
39 -4.56703382951673e-05
40 1.57176327775232e-05
41 1.68232345458819e-05
42 -7.97146458353382e-06
43 -8.13198130344972e-06
44 -4.43711087427801e-06
45 3.5482109979057e-06
46 -1.03898139514058e-06
47 -2.59530270341202e-06
48 1.49267705751299e-07
49 6.07564402343996e-07
50 -2.28117073675094e-06
51 1.02123419765121e-06
52 1.0588905752229e-06
53 6.81281846937054e-07
54 -1.65053108958091e-07
55 -3.07486118344968e-07
56 -1.66590066186245e-08
57 1.94713777545985e-07
58 -3.61553809113957e-08
59 5.8055240970134e-07
60 2.90026463289905e-07
61 -2.07107504479609e-07
62 -2.68251358903626e-08
63 9.24184533346306e-08
};
\addplot [semithick, black, opacity=0.5]
table {%
0 0
63 0
};
\end{groupplot}

\end{tikzpicture}

    \vspace{-4mm}
    \caption{\textbf{[Top]:} Example of long convolution parametrization for ${\sf Hyena}$ operators, with a decay ${\sf Window}(t) = \exp\{- \alpha t\}$. Parameter $\alpha$ is modified across the independent channels of ${\sf Hyena}$ to regularize filters to be of different lengths. In practice, we add a bias term to our window, so that the filters are not constrained to be zeros after a length determined by the decay rate. 
    }
    \vspace{-4mm}
    \label{fig:modul}
\end{figure}
%
Interestingly, we find that long exponentially decaying filters display synergy with high-frequency filters, as they enable the operator to select specific inputs at specific steps\footnote{This observation finds mirrors in the parametrization of the convolutions in H3 \citep{dao2022hungry} as a shift SSM and a diagonal SSM.}. Similarly to \citep{romero2021ckconv}, we use high-frequency periodic activations (sine) in the {$\sf FFN$}. This allows \eqref{filt} to learn filters with high-frequency content, addressing the low-frequency bias of neural networks \citep{basri2020frequency}.
%
Owing to the {$\sf FFN$}, the parametrization in \eqref{filt} can approximate filters obtained through other means, such as S4 \citep{gu2020hippo,gu2021efficiently}, CKConv \citep{romero2021ckconv}, SGConv \citep{li2022makes} and \textit{Fourier Neural Operator} (FNO) \citep{li2020fourier}.
%

\paragraph{Preserving causality}
%
Causality is necessary to train autoregressive language models, in order for the output at a given position to depend only on the past. For example, Transformers mask the attention matrix to be lower triangular. In the case of ${\sf Hyena}$, causality can be guaranteed by parametrizing causal convolutions: 
%
\begin{proposition}[Causal Hyenas]\label{prop:causality}
    If each filter $h^n, ~n=1,\dots, N$ is causal, then the corresponding ${\sf Hyena}_N$ operator is causal.
    %
\end{proposition}
%
In practice, we need not constrain the learning of the filter \eqref{filt} to ensure its \textit{numerical} causality. If we use FFT-based convolution algorithms, all we need is to evaluate the filter at $t=0,\dots,L-1$ and zero-pad the input and filter sequences to $2 L - 1$ before taking FFT. 
%
\paragraph{Efficiency}
%
One bottleneck of long convolution models can be their low utilization of hardware accelerators, especially when they involve iterative numerical methods to materialize the filter\footnote{In contrast, deep learning primitives are designed for high GPU utilization, with {\sf FFNs} and attention usually reaching $50-70\%$ or higher, if optimized.}. Evaluation of \ref{filt} is fast, since it involves a single forward pass of an {$\sf FFN$}, and can be performed in parallel across sequence length and all orders of an ${\sf Hyena}$ operator as displayed in Algorithm \ref{alg:hyenaa}, increasing hardware utilization. An additional source of low utilization is the FFT, which is also shared by other long other convolutional layers. This bottleneck can be partially addressed by blocking \citep{selesnick2017fast}, and optimization of the underlying routines \citep{dao2022hungry}. We benchmark runtime in Section \ref{benchm}.
%

\subsection{{\sf Hyena} Algorithm}
%

A forward pass of {\sf Hyena} is summarized below. 

\setcounter{algorithm}{-1}
\begin{algorithm}[h]
    \caption{{\sf Projection}}\label{alg:hyenaa}
    \caption{{\sf Projection}}
    \begin{algorithmic}
    \REQUIRE Input sequence $u \in \R^{L \times D}$ \\ 
    \STATE {\small 1.} In parallel across $L$: $\hat{z} = {\sf Linear}(u)$, ${\sf Linear}:\R^{D} \rightarrow \R^{(N+1)D}$ \\
    \STATE {\small 2.} In parallel across $D$: $z = {\sf DepthwiseConv1d}(h, \hat{z} )$, $h$ is a short convolution filter \\
    \STATE {\small 3.} Reshape and split $z$ into $x^1, \dots, x^N, v$. Dimensions of one element are $x^n \in \R^{D \times L}$\\ 
    \STATE Return $x^1, \dots, x^N, v$, $x^n$
    \end{algorithmic}
\end{algorithm}

\setcounter{algorithm}{0}
\begin{algorithm}[h]
    \caption{${\sf HyenaFilter}$}\label{alg:hyenaa}
    \caption{${\sf Hyena}$ Filter}
    \begin{algorithmic}
    \REQUIRE Sequence length $L$, positional embedding dimension $D_e$ \\ 
    \STATE {\small 1.} $t  = {\sf PositionalEncoding}(L)$, $t \in \R^{L \times D_e}$ \\
    \STATE {\small 2.} In parallel across $N, L$: $\hat{h} = {\sf FFN}(t)$, ${\sf FFN}: \R^{D_e}\rightarrow \R^{N D}$, $\hat{h} \in \R^{L \times N D}$\\
    \STATE {\small 3.} Reshape to $\hat{h} \in \R^{N \times D \times L}$ \\
    \STATE {\small 4.} $h = \hat{h} \cdot {\sf Window}(t)$, $h \in \R^{N \times D \times L}$ \\
    \STATE {\small 5.} Split $h$ into $h^1, \dots, h^N$
    \STATE Return $h^1, \dots, h^N$
    \end{algorithmic}
\end{algorithm}

\setcounter{algorithm}{1}
\begin{algorithm}[h]
    \caption{${\sf Hyena}$ Operator}\label{alg:hyenaa}
    \caption{Forward pass of ${\sf Hyena}$}
    \begin{algorithmic}
    \REQUIRE Input sequence $u \in \R^{L \times D}$, order $N$, model width $D$, sequence length $L$, positional embedding dimension $D_e$ \\ 
    \STATE {\small 1.} $x^1, \dots, x^N, v = {\sf Projection}(u)$ \\
    \STATE {\small 2.} $h^1, \dots, h^N = {\sf HyenaFilter}(L, D_e)$ \\
    \FOR{$n = 1,\dots,N$}
    \STATE {\small 3.} In parallel across $D$: $v_{t} \leftarrow x^n_{t}\cdot {\sf FFTConv}(h^n, v)_{t}$ \\
    \ENDFOR  
    \STATE Return $y = v$
    \end{algorithmic}
\end{algorithm}

%
\begin{proposition}[Computational Complexity]
    The computational cost of processing an input $u\in\R^{L\x D}$ with an order-$N$ $\sf Hyena$ operator is
    % 
    \[
        %\cO(D N L\log_2 L + L N D^2)
        \cO(N D L(\log_2 L + D))
    \]
    %
\end{proposition}
%