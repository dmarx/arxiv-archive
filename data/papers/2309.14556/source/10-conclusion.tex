\section{Conclusion}

In this work, we utilize the widely accepted Torrance Test of Creative Thinking originally designed for the evaluation of creativity as a process, and align it towards the evaluation of creativity as a product. In particular, we focus on short fiction writing and formulate the Torrance Test for Creative Writing (TTCW). Our collaborative process, involving creative writing experts in both the development and validation stages of the TTCW, has established the test as a robust instrument for assessing creativity in fictional short stories. The experimental data derived from the evaluation of both human-authored and LLM-generated stories offers a rich comparative analysis. It underscores the proficiency of seasoned writers in evoking creativity, outperforming LLMs by a considerable margin. Our analysis also highlights the disparity in creative prowess among different LLMs, revealing that while certain LLMs might exhibit proficiency in some dimensions of creativity, there exists a wide chasm between them and human expertise when assessed holistically. Notably, the TTCW tests also provided a granular perspective on the areas where LLMs falter the most in terms of creativity. Our secondary investigation into the feasibility of LLMs in reproducing expert assessments yielded that, at least in their current state, LLMs are not yet adept at administering TTCW tests. This indicates a dual challenge: not only are LLMs lagging in producing inherently creative content, but they also lack the finesse to evaluate creativity as experts do. We hope that our evaluation framework using TTCW, and findings on the creative capabilities of LLMs coupled with our dataset will steer future work and innovations in creativity research.
