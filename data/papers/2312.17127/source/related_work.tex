\section{Conclusion}
\label{sec:relatedwork}
%
\paragraph{Summary}
We have shown that equational theories for the graph interface to the probabilistic programming language (Ex.~\ref{ex:interfaces}) give rise to graphons (Theorem~\ref{thm:eq-theory-to-graphon}).
Conversely, every graphon arises in this way. We showed this generally using an abstract construction based on Markov categories (Corollary~\ref{corollary:graphon-markov}) and methods from category theory~\cite{hermida-tennent,hu-tholen}. Since this is an abstract method, we also considered two concrete styles of semantic interpretation that give rise to classes of graphons: traditional measure-theoretic interpretations give rise to black-and-white graphons (Prop.~\ref{prop:bw-ok}), and an interpretation using the internal probability theory of Rado-nominal sets gives rise to \ErdosRenyi\ graphons (Corollary~\ref{cor:e-r}).

\paragraph{Further context, and future work}

The idea of studying exchangeable structures through program equations is perhaps first discussed in the abstract~\cite{XRPDA-PPS2017}, whose \S3.2 ends with an open question about semantics of languages with graphs that the present paper addresses.
Subsequent work addressed the simpler setting of exchangeable sequences and beta-bernoulli conjugacy through program equations~\cite{Staton-BB18-short},
and stochastic memoization~\cite{kaddar-staton-stoch-mem}; the latter uses a category similar to $\RadoNom$, although the monad is different. % (see also the extended version \cite{Staton-BB18-full}).
%That work extends 
Beyond sequences~\cite{Staton-BB18-short} and graphs (this paper), a natural question is how to generalize to arbitrary exchangeable interfaces (see e.g.~\cite{6847223}). 
For example, we could consider exchangeable random boolean arrays via the interface
\[
  \mathsf{new\text-row}:\tunit\to\mathsf{row},\quad\mathsf{new\text-column()}:\tunit\to\mathsf{column},\quad
  \mathsf{entry}:\mathsf{row}*\mathsf{column}\to \tbool
\]
and random hypergraphs with the interface 
\[
  \mathsf{new}:\tunit \to \mathsf{vertex},\quad
  \mathsf{hyperedge}_n:\mathsf{vertex}^n\to \tbool\text.
\]
We could also consider interfaces for hierarchical structures, such as arrays where every entry contains a graph. Diverse exchangeable random structures have been considered from the model-theoretic viewpoint~\cite{ackerman-autm-invariant,crane-towsner} and from the perspective of probability theory (e.g.~\cite{kallenberg-2010,Campbell23,jlsy-dags}), but it remains to be seen whether the programming perspective here can provide a unifying view. Another point is that graphons correspond to dense graphs, and so a question is how to accommodate sparse graphs from a programming perspective (e.g.~\cite{10.1111/rssb.12233,10.1214/18-AOS1778}). 

This paper has focused on a very simple programming language~(\S\ref{sec:generic-ppl}).
As mentioned in Section~\ref{sec:intro:practice},
several implementations of probabilistic programming languages do support various Bayesian nonparametric primitives based on exchangeable sequences, partitions, and relations
%e.g. Anglican  \cite{wood-aistats-2014},
%Church \cite{goodman2008church,roy2008stochastic},
%Hansei \cite{ks-prob-first-class-store}, LazyPPL~\cite{lazyppl} and Venture \cite{msp-venture}.
(e.g. \cite{wood-aistats-2014,goodman2008church,roy2008stochastic,ks-prob-first-class-store,lazyppl,msp-venture}).
In particular, the `exchangeable random primitive' (XRP) interface
\cite{XRP-PPS2016,wu-church} 
provides a built-in abstract data type for representing exchangeable sequences.
This aids model design by its abstraction, but also aids inference performance by clarifying the independence relationships. 
%Although the contribution in this paper is theoretical, our results show that the graph interface of Example~\ref{ex:interfaces}(3) is canonical.%\todo{something about conditional independence?}

Aside from practical inference performance, we can ask whether representation and inference are computable.
For the simpler setting of exchangeable sequences, this is dealt with positively by  \cite{freer2012computable, pmlr-v9-freer10a}.
% By \cite{freer2012computable}, computable exchangeable sequences admit a computable representation
%of the so-called mixing measure that characterizes any exchangeable sequence.
%Furthermore, by \cite{pmlr-v9-freer10a}, the relevant 
%posterior predictive distributions (a one-dimensional analogue of the sequence of kernels $k_n$)
%are computable for computable exchangeable sequences.
The question of computability for graphons and exchangeable graphs is considerably subtler,
and some standard representations are noncomputable
\cite{ackerman2019algorithmic} (see also \cite{CompExch-PPS2017}).
This suggests several natural questions about whether certain natural classes of computable exchangeable graphs
can be identified by program analyses in the present context.% \todo{punchy last sentence}




%%% Local Variables:
%%% mode: latex
%%% TeX-master: "popl24"
%%% End:
