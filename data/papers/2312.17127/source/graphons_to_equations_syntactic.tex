\section{From graphons to program equations}\label{sec:graphons-to-equational-theories}

\newcommand{\CatCLG}{\mathbf{RGM}}
\newcommand{\Inj}{\mathbf{FinSetInj}}\label{sec:rgm-cat}
\newcommand{\Nat}{\mathrm{Nat}}

In Section~\ref{sec:prog-eq-to-graphons}, we showed how a distributive Markov category modelling the graph interface (Ex.~\ref{ex:interfaces}(3)) gives rise to a graphon.
In this section, we establish a converse: every graphon arises in this
way (Corollary~\ref{corollary:graphon-markov}). 
Theorem~\ref{thm:Markov-functor-graphon} will establish slightly more:
there is a `generic' distributive Markov category
(\S\ref{sec:generic-dmc}) modelling the graph interface whose
Bernoulli-based quotients are in precise correspondence with graphons
(\S\ref{sec:bernoulli-base-graphon}). This approach also suggests an
operational way of implementing the graph interface for any graphon (\S\ref{sec:abstract-op-sem}).

\subsection{A Generic Distributive Markov Category for the Graph Interface}
\label{sec:generic-dmc}
\newcommand{\CatG}{\mathbf{G}}
\newcommand{\GenericM}{\mathcal{T}}


We construct this generic category in two steps. We first create a distributive Markov category, actually a
distributive category, $\Fam(\CatG\op)$, that supports $(\tvertex,\tedge)$. We then add
$\tnew$ using the monoidal indeterminates method
of~\cite{hermida-tennent}.

\subsubsection{Step 1: A Distributive Category with $\tedge$}\label{sec:edge-cat}
\newcommand{\vertices}[1]{V_{#1}}
\newcommand{\edges}[1]{E_{#1}}

We first define a distributive category that supports
$(\tvertex,\tedge)$.
Let $\CatG$ be the category of finite graphs and functions that
preserve and reflect the edge relation. That is, a morphism $f:g\to
g'$ is a function $f:\vertices g \to \vertices g'$ such that for all
$v,w\in \vertices g$ we have $\edges g(v,w)$ if and only if $\edges
g(f(v),f(w))$. 

Recall (e.g.~\cite{hu-tholen}) that the free finite coproduct completion of a category
$\CatA$, $\Fam(\CatA)$ is given as follows. The objects of
$\Fam(\CatA)$ are sequences
$(X_1\dots X_n)$ of objects of $\CatA$, and the morphisms
$(X_1\dots X_m)\to (Y_1\dots Y_n)$ are pairs $(f,\{f_i\}_{i=1}^m)$
of a function $f:m\to n$ and a sequence of morphisms ${f_1:X_{1}\to Y_{f(1)}},\dots, {f_m:X_{m} \to
Y_{f(m)}}$ in $\CatA$.  

We consider the category $\Fam(\CatG\op)$. Let $\sem\tvertex=(1)$, the
singleton sequence comprising the one-vertex graph. 
\begin{proposition}\label{prop:FamG}
  \begin{enumerate}
  \item The free coproduct completion $\Fam(\CatG\op)$ is a distributive
    category, with the product $\sem\tvertex^n$ being the sequence of all
    graphs with $n$ vertices.
    In particular, $\sem\tvertex^2$ is a sequence with two components,
    the complete graph and the edgeless graph with two vertices. 
  \item Let $\tedge: \sem\tvertex\times \sem\tvertex\to 1+1$
    be the morphism $(\id,\{!,!\})$, intuitively returning true for
    the edge, and false for the edgeless graph. Here the terminal object $1$
    of $\Fam(\CatG\op)$ is the singleton tuple of the empty graph.
    This interpretation satisfies \eqref{eqn:simplegraph}. 
  \end{enumerate}
\end{proposition}
\begin{proof}[Proof notes.]
  Item~(1) follows from \cite{hu-tholen}, which shows that
  limits in $\Fam(\CatG\op)$ amount to ``multi-colimits'' in $\CatG$. 
  For example, the family of all graphs with $n$ vertices is a
  multi-coproduct of the one-vertex graph in $\CatG$,
  hence forms a product in $\Fam(\CatG\op)$.
  Item~(2) is then a quick calculation. All morphisms in $\Fam(\CatG\op)$ are deterministic. 
\end{proof}

\subsubsection{Step 2: Adjoining $\tnew$}
\newcommand{\CatHT}{\Fam(\CatG\op)[\nu]}

In Section~\ref{sec:edge-cat}, we introduced a distributive category
that interprets the interface $(\tvertex,\tedge)$.
But it does not support $\tnew$, and indeed there are no morphisms
$1\to\sem\tvertex$. 
To additionally
interpret $(\tnew)$, we freely adjoin it. We essentially use the `monoidal
indeterminates' method of Hermida and
Tennent~\cite{hermida-tennent} to do this.
Their work was motivated by semantics of dynamic memory allocation, but has also been
related to quantum phenomena~\cite{hs-quantum,ahk-partial-quantum} and to categorical gradient/probabilistic methods~\cite{para,backprop,Shiebler2021categorical}, where it is known as the `para construction'. It is connected to earlier methods for the action calculus~\cite{pavlovic_1997}.

Let $\Inj$ be the category of finite sets and injections. It is a
monoidal category with the disjoint union monoidal structure (e.g.~\cite{fiore-comb,power-local}).
Consider the functor $J:\Inj\op\to \Fam(\CatG\op)$, with
$J(n)=\sem\tvertex^n$, and where the functorial action is by
exchange and projection. This is a strong monoidal functor. (Indeed, it is
the unique monoidal functor with $J(1)=\sem\tvertex$.)

For any monoidal functor, Hermida and Tennent~\cite{hermida-tennent}
provide monoidal indeterminates by introducing a
`polynomial category', by analogy with a polynomial ring.
Unfortunately, a  general version for \emph{distributive} monoidal categories is not yet known,
so we focus on the specific case of $J:\Inj\op\to \Fam(\CatG\op)$. We build a new category
$\Fam(\CatG\op)[\nu:J\,\Inj\op]$, which we abbreviate $\CatHT$. It has the same objects as
$\Fam(\CatG\op)$, but the morphisms $\vec X\to \vec Y$ are equivalence classes of morphisms
\[
[k,f]: \sem\tvertex^{k}\times \vec X\to \vec Y
\]
in $\Fam(\CatG\op)$, modulo reindexing. The reindexing equivalence relation is generated by putting $[k,f]\sim [l,g]$ when
there exist injections $\iota_1\dots \iota_m:k\to l $ such that 
\[  g\  =\  \textstyle\Big(\sem\tvertex^{l}\times \vec X
  \cong \sum_{j=1}^m \sem\tvertex^l\times X_j
  \xrightarrow{\sem\tvertex^{(\iota_j)}\times X_j}
  \sum_{j=1}^m \sem\tvertex^k\times X_j\cong \sem\tvertex^k\times \vec X
  \xrightarrow f
  \vec Y\Big)
\]
where $\vec X=(X_1,\dots, X_m)$. In particular, when $m=1$, i.e. $\vec X=X$ is a singleton sequence, we have
\begin{equation}\label{eqn:CatHT}
  \CatHT(X,\vec Y)\  \cong\ \colim_{k\in\Inj}\Fam(\CatG\op)(\sem\tvertex ^k\times X,\vec Y)\text.
\end{equation}

Composition and monoidal structure accumulate in $\sem\tvertex^k$,
as usual in the monoidal indeterminates (`para') construction, 
e.g.
\[
  \Big(\vec X\xrightarrow {[k,f]} \vec Y
  \xrightarrow{[l,g]}
  \vec Z\Big)
  =
  \Big(\vec X\xrightarrow {[l+k,g\circ (\sem \tvertex^l \times f)]}
  \vec Z\Big)
\]
and although our equivalence relation is slightly coarser, it still respects the symmetric monoidal category structure,
and there is a monoidal functor $\Fam(\CatG\op)\to \CatHT$,
regarding each morphism $f:\vec X\to \vec Y$ in $\Fam(\CatG\op)$ as a morphism $[0,f]$ in $\CatHT$.
But there is also now an adjoined morphism $\nu=[1,\id]:1\to \sem\tvertex$.

This monoidal category $\CatHT$ moreover inherits the distributive coproduct structure from $\Fam(\CatG\op)$, and the functor $\Fam(\CatG\op)\to \CatHT$ is a distributive Markov functor.
To define copairing of $[k,f]:\vec X\to\vec Z$ and $[l,g]:\vec Y\to \vec Z$ we use the reindexing equivalence relation to assume $k=l$
and then define the copairing as $\langle [k,f],[k,g]\rangle=[k,\langle f,g\rangle] : \vec X+\vec Y\to\vec Z$. 

In summary:
%\begin{proposition}\label{prop:GenericM}
  \begin{itemize}\item $\CatHT$ is a distributive Markov category.
  \item $\CatHT$ supports the graph interface, via
    the interpretation of $(\tvertex,\tedge)$ in $\Fam(\CatG\op)$, but also with the 
    interpretation
    $\sem{\tnew}=\nu:1\to \sem\tvertex$.
  % \item The numerals are the random graph model category
  %   of~Section~\ref{sec:rgm-cat}:
  %   $\Fam(\CatG\op[\nu:\Inj\op]_\NN\cong \CatCLG$.
  \end{itemize}
%\end{proposition}
%\todo[inline]{proof}


\subsection{Bernoulli Bases for Random Graph Models}
\label{sec:bernoulli-base-graphon}
The following gives a precise characterization of graphons in terms of the numerals of $\CatHT$.

\begin{theorem}\label{thm:Markov-functor-graphon}
  To give a distributive Markov functor $\CatHT_\NN \to \FinStoch$ is to give a graphon.
\end{theorem}
\begin{proof}[Proof outline]
  We begin by showing a related characterization: that graphons correspond to certain natural transformations. 
  Observe that any distributive Markov category $\CatA$
  gives rise to a symmetric monoidal functor $\CatA(1,-):\FinSet_\NN\to\Set$, regarding the numerals of $\FinSet_\NN$ as objects of $\CatA$ (\S\ref{sec:bernoulli-base}). 
  %the distributive Markov categories$\CatHT$ and
  %$\FinStoch$ give rise to symmetric monoidal functors $\CatHT(1,-),
  %\FinStoch(1,-) : \FinSet_\NN \to \Set$.
  Let $G_k=2^{k(k-1)/2}$ be the set of $k$-vertex graphs. 
  We can characterize the natural transformations $\alpha :
  \CatHT(1,-) \to \FinStoch(1,-)$ as follows.
  \begin{align*}
    &     \Nat(\CatHT(1,-)\ ,\ \FinStoch(1,-))\\
    \cong\ &\Nat\big(\colim_{k\in\Inj}\FinSet(G_k,-)\ ,\ \DistM(-)\big)
    &&\text{(Ex.~\ref{ex:bernoulli-base}(2), Prop.~\ref{prop:FamG}(1) and \eqref{eqn:CatHT})}\\
    \cong\ &\lim_{k\in \Inj\op}\Nat(\FinSet(G_k,-)\ ,\ \DistM(-))&&\text{(universal property of colimits)}\\
    \cong\ &\lim_{k\in \Inj\op}\DistM(G_k)&&\text{(Yoneda lemma)}
  \end{align*}
  An element of this limit of sets is by definition a sequence of distributions $p_k$ on $G_k$ that is invariant under reindexing by $\Inj\op$. Since injections are generated by inclusions and permutations, this is then a sequence that is consistent and exchangeable (Def.~\ref{def:rgm}), respectively.
%Thus to give a natural transformation $\CatHT(1,-)\to
 % \FinStoch(1,-)$
  %is to give a family of distributions $p_k$ on $G_k$ that are
  %consistent and exchangeable (Def.~\ref{def:rgm}).
  Such a natural transformation~$\alpha$ is monoidal if and only if
  the sequence is also \emph{local}. Hence a monoidal natural transformation is the
  same thing as a random graph model.

  In fact, every monoidal natural transformation $\alpha : \CatHT(1,-)
  \to \FinStoch(1,-)$ arises uniquely by restricting a distributive
  Markov functor $F : \CatHT_\NN \to \FinStoch$. We now show this, to
  conclude our proof. 
  Given $\alpha$, let $F_{m,n} : \CatHT_\NN(m,n) \to \FinStoch(m,n)$ be:% given by
  \begin{displaymath}
    \CatHT_\NN(m,n) \cong \CatHT_\NN(1,n)^m \xrightarrow{\alpha_n^m} \FinStoch(1,n)^m \cong \FinStoch(m,n).
  \end{displaymath}
  It is immediate that this $F$ preserves the symmetric monoidal structure and coproduct structure, but not that $F$ is a functor.
  However, the naturality of $\alpha$ in $\FinSet_\NN$ gives us that $F$ preserves postcomposition by morphisms of $\FinSet_\NN$.
  All of this implies that general categorical composition is preserved as well, since, in any distributive Markov category of the form $\CatA_\NN$, for $f : l \to m$ and $g : m \to n$, the composite $g \circ f : l \to n$ is equal to
  \begin{displaymath}
    l = l \otimes 1^{\otimes m}
    \xrightarrow{f \otimes g_1 \otimes \ldots \otimes g_m}
    m \otimes n^{\otimes m} \xrightarrow{\text{eval}} n
  \end{displaymath}
  where $g_i = g \circ \iota_i$ for $i = 1,\ldots,m$ and $\text{eval}$ is just the evaluation map $m \times n^m \to n$ in $\FinSet$.
\end{proof}

\begin{corollary}\label{corollary:graphon-markov}
  Every graphon arises from a distributive Markov category via the random graph model in~\eqref{eqn:programgraph}.
\end{corollary}
\begin{proof}[Proof summary]
  Given a graphon, we consider the distributive Markov functor that
  corresponds to it, $\bbase : \CatHT_\NN \to \FinStoch$, by Theorem~\ref{thm:Markov-functor-graphon}.
  Using the quotient construction of Proposition~\ref{prop:quotient}, we get a distributive Markov category with a Bernoulli base.
  It is straightforward to verify that the random graph model induced by~\eqref{eqn:programgraph} is the original graphon.
\end{proof}

\subsection{Remark on Operational Semantics}
\label{sec:abstract-op-sem}
The interpretation in this section suggests a general purpose operational semantics for closed programs at ground type,
$\vdash t:n$, along the following lines:
\begin{enumerate}
\item Calculate the interpretation $\sem t:1\to n$ in $\CatHT$. There are no probabilistic choices in this step, it is a symbolic manipulation, because the morphisms of the Markov category
  $\CatHT$ are built from tuples of finite graph homomorphisms. In effect, this interpretation pulls all the $\tnew$'s to the front of the term.
\item Apply the Markov functor $\bbase(\sem t)$ to obtain a probability distribution on $n$, and sample from this distribution to return a result.
\end{enumerate}


%%% Local Variables:
%%% mode: latex
%%% TeX-master: "popl24"
%%% End:
