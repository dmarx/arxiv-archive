\section{Interpretation: \ErdosRenyi\ graphons via Rado-nominal sets}
\label{sec:ER-Rado}
In Section~\ref{sec:graphons-to-equational-theories}, we gave a general construction to show that every graphon arises from a Bernoulli-based equational theory.
In Section~\ref{sec:randomfree}, we gave a more concrete interpretation, based on measure-theory, for black-and-white graphons. 
We now consider the \ErdosRenyi\ graphons (Def.~\ref{def:e-r-graphon}), which are not black-and-white. 

Our interpretation is based on Rado-nominal sets. These are also studied elsewhere, but for different purposes (e.g.~\cite{lmcs:1157,bojancyk-place,homomorphisms-fodef}, \cite[\S1.9]{pitts-nom-book}).

Rado-nominal sets (\S\ref{sec:rado-def}) are sets that are equipped with an action of the automorphisms of the Rado graph, which is an infinite graph that contains every finite graph.
There is a particular Rado-nominal set $\VV$ of the vertices of the Rado graph. The type $\tvertex$ will be interpreted as $\VV$;
$\tedge$ is interpreted using the edge relation~$E$ on $\VV$.
The equational theory induced by this interpretation gives rise to the \ErdosRenyi\ graphons (Def.~\ref{def:e-r-graphon}).

Since Rado-nominal sets form a model of ZFA set theory (Prop.~\ref{prop:radonom-topos}), we revisit
probability theory internal to this setting. We consider internal probability measures on Rado-nominal sets (\S\ref{sec:rado-prob}), and we show that there are internal probability measures on $\VV$ that give rise to \ErdosRenyi\ graphons (\S\ref{sec:er-meas}). The key starting point here is that, internal to Rado-nominal sets, the only functions $\VV\to 2$ are the sets of vertices that are definable in the language of graphs (\S\ref{sec:powersets-and-definiable-sets}). 

We organize the probability measures (Def.~\ref{def:probability}) into
a probability monad on Rado-nominal sets (\S\ref{sec:nom-monad}), analogous to the Giry monad. 
Fubini does not routinely hold in this setting (\S\ref{sec:fubini}), but we use a standard technique to cut down to a commutative affine monad (\S\ref{sec:comm-monad}). This gives rise to a Bernoulli-based equational theory, and in fact, this theory corresponds to \ErdosRenyi\ graphons (via \eqref{eqn:programgraph}: Corollary~\ref{cor:e-r}).

\subsection{Definition and First Examples}\label{sec:rado-def}
The Rado graph $(\VV,E)$ (\hspace{1sp}\cite{ackermannWiderspruchsfreiheitAllgemeinenMengenlehre1937,radoUniversalGraphsUniversal1964}, also known as the `random graph'~\cite{erdosRandomGraphs1959}) is the unique graph, up to isomorphism, with a countably infinite set of vertices
that has the extension property:
if $A$, $B$ are disjoint finite subsets of $\VV$, then there is a
vertex $a \in \VV \setminus (A \cup B)$ with an edge to all the vertices in $A$ but none of the
vertices in $B$.

The Rado graph embeds every finite graph, which can be shown by using
the extension property inductively.

An automorphism of the Rado graph is a graph isomorphism $\VV\to
\VV$. 
The automorphisms of the Rado graph relate to isomorphisms between
finite graphs, as follows. First, if $A$ is a finite graph regarded as
a subset of $\VV$, then any automorphism $\sigma$ induces an
isomorphism of finite graphs $A\cong \sigma[A]$. Conversely,
if $f\colon A\cong B$ is an isomorphism of finite graphs, and we regard $A$
and $B$ as disjoint subsets of $\VV$, then there exists an
automorphism $\sigma$ of $\VV$ that restricts to $f$ (i.e.\ $f=\sigma|_A$).

We write $\AutRado$ for the group of automorphisms of $(\VV,E)$. 
(This has been extensively studied in model theory and descriptive set theory, e.g. \cite{MR2140630,MR3274785}.)
%The automorphism group of the Rado graph, $\AutRado$, has been extensively studied in model theory and descriptive set theory. In particular, it has been shown to be extremely amenable \cite{MR2140630} and uniquely ergodic \cite{MR3274785}.

\begin{definition}
  A \emph{Rado-nominal set} is a set $X$ equipped with an action
  $\bullet :\AutRado\times X\to X$ (i.e.~$\id\bullet x=x$; $(\sigma_2\cdot
  \sigma_1) \bullet x= \sigma_2\bullet \sigma_1\bullet x$) such that
  every element has finite support.

  An element $x\in X$ is defined to have \emph{finite support} if there is a finite set
  $A\subseteq \VV$ such that for all automorphisms $\sigma$, if $\sigma$ 
  fixes $A$ (i.e.~$\sigma|_A=\id_A$), it also fixes $x$ (i.e.~$\sigma\bullet x=x$).

  \emph{Equivariant functions between Rado-nominal sets} are functions that
  preserve the group action (i.e.~$f(\sigma\bullet x)=\sigma\bullet (f(x))$). 
\end{definition}
\begin{proposition}[\hspace{1sp}\cite{pitts-nom-book}]
  If finite sets $A,B\subseteq \VV$ both support $x$, so does $A\cap B$. Hence every
  element has a least support. 
\end{proposition}
\begin{example}
  \begin{enumerate}
    \item The set $\VV$ of vertices is a Rado-nominal set, with
      $\sigma\bullet a=\sigma(a)$. The support of vertex $a$ is $\{a\}$.
    \item The set $\VV\times \VV$ of pairs of vertices is a
      Rado-nominal set, with
            $\sigma\bullet (a,b)=(\sigma(a),\sigma(b))$. The support
            of $(a,b)$ is $\{a,b\}$.
            More generally, a finite product of Rado-nominal sets has a
            coordinate-wise group action.
          \item The edge relation $E\subseteq \VV\times \VV$ is a
            Rado-nominal subset (which is formally defined in \S\ref{sec:powersets-and-definiable-sets}) because automorphisms preserve the edge relation. 
            \item Any set $X$ can be regarded with the discrete
              action,
              $\sigma\bullet x=x$, and then every element has empty
              support.
              We regard these sets with the discrete action:
              $1=\{\star\}$; $2=\{0,1\}$; $\NN$; and the unit interval $[0,1]$.
              \end{enumerate}
          \end{example}
\subsection{Powersets and Definable Sets}
\label{sec:powersets-and-definiable-sets}
For any subset $S\subseteq X$ of a Rado-nominal set, we can define
$\sigma\bullet S=\sigma[S]= \{\sigma\bullet x~|~x\in S\}$. 
We let
\begin{equation}2^X=\{S\subseteq X~|~S \text{ has finite support}\}\text.\label{eqn:powerobject}\end{equation}
This is a Rado-nominal set.


\begin{example}\label{example:subset}
  We give some concrete examples of subsets.
\begin{enumerate}
  \item For vertices $b$ and $c$ in $\VV$ with no edge between them,
    the
    set $\{a\in\VV~|~E(a,b) ∧ E(a,c)\}$ is the set of ways of
    forming a horn. It has support $\{b,c\}$. 
  \item $\{(b,c)\in \VV^2~|~E(a,b) ∧ E(a,c) ∧ \neg E(b,c)\}$ is the
    set of horns with apex $a$; it has support $\{a\}$.
  \item $\{(a,b,c)\in \VV^3~|~E(a,b) ∧ E(a,c) ∧ \neg E(b,c)\}$ is the
    set of all oriented horns; it has empty support.
  \item (Non-example) There is a countable totally disconnected subgraph of $\VV$; it does not have finite support as a subset of $\VV$.
%  \item (Non-example) The set of finite graphs that have a Hamiltonian path is not definable as a first-order formula over the theory of graphs. This implies that the set does not have finite support.
  \end{enumerate}
  \end{example}
  In fact, the finitely supported subsets correspond exactly to the
  definable sets in first-order logic over the theory of graphs. The
  following results may be folklore.

  \begin{proposition}\label{prop:radonom-definable-sets}
    Let $S\subseteq \VV^n$, and $A\subseteq \VV$ be finite. The following are equivalent:
    \begin{itemize}
      \item $S=\{(s_1,\dots s_n)~|~\phi(s_1\dots s_n)\}$, for a
        first-order formula
        $\phi$ over the theory of graphs, with parameters in~$A$;
      \item $S$ has support~$A$.
        \end{itemize}
      \end{proposition}
      \begin{proof}
      $(\Rightarrow)$ For all isomorphisms $f\colon \VV \to \VV$ that fix $A$, and for all elements $a_1\dots a_k \in A$ and 
      subsets
      $S=\{(s_1,\dots, s_n)~|~\phi(s_1\dots s_n, a_1\dots a_k)\}$,
      we have 
\[       \phi(f(s_1)\dots f(s_n), a_1\dots a_k) \quad= \quad
       \phi(f(s_1)\dots f(s_n), f(a_1)\dots f(a_k)) \text .
      \]
      Furthermore, $\phi$ is invariant with respect to $f$. Thus, the image $f(S) \subseteq S$. By a similar argument, we have $f\inv(S) \subseteq S$, so that $S \subseteq f(S)$. Thus, $f(S) = S$
      (\hspace{1sp}\cite[Prop. 1.3.5]{markerModelTheoryIntroduction2002}).

      $(\Leftarrow)$
      This is a consequence of the Ryll-Nardzewski theorem for the theory of the Rado graph (which can be shown to be $ω$-categorical by a back-and-forth argument, using the extension property of the Rado graph). But we give here a more direct proof, assuming $n = 1$ for simplicity. Suppose $A \subseteq \VV$ is a finite support for $S$. Then, for any $v,v' \in \VV \backslash A$, if $v$ and $v'$ have the same connectivity to $A$, then they are either both in or not in $S$ since, by the extension property, we can find an automorphism fixing $A$ and sending $v$ to $v'$. The set of vertices with the same connectivity to $A$ as $v$ is definable, and there are only $2^{|A|}$ such sets. Hence, $S \backslash A$ is a union of finitely many definable sets, and as $S \cap A$ is definable (being finite), so is $S = (S \backslash A) \cup (S \cap A)$.
      % More explicitly, the sets with a given finite support $A$ are in one-to-one correspondence with DNF formulas of the form 
      % \begin{multline}
      %   φ_A(U, A')(x) ≝ \bigvee_{e ∈ U} \Big( \bigwedge_{\substack{a ∈ A\\ e_a = 1}} E(x, a) ∧  \bigwedge_{\substack{a ∈ A\\ e_a = 0}} ¬E(x, a)\Big) \\
      %   ∨ \bigvee_{a ∈ A'} (x = a)
      %   \label{eq:definable-dnf}
      % \end{multline}
      % where $A ⊆ 𝕍$, $U ⊆ 2^{A}$ and $A' ⊆ A$.
      \end{proof}

      We note that $2^X$ in \eqref{eqn:powerobject} is a canonical
      notion of internal powerset, from a categorical perspective.
%      it is uniquely determined by the induced bijective correspondence
%      \[
%        \RadoNom(Y\times X,2)\cong \RadoNom(Y,2^X)
%      \]
%      natural in $Y$, where $\RadoNom$ is the category of Rado-nominal sets and equivariant functions.
\begin{proposition}\label{prop:radonom-topos}
$\RadoNom$ is a Boolean Grothendieck topos, with powerobject $2^X$ in \eqref{eqn:powerobject}. 
\end{proposition}
\begin{proof}[Proof notes]
  $\RadoNom$ can be regarded as continuous actions of $\AutRado$, regarded as a topological group with the product topology, and then we invoke standard methods~\cite[Ex.~A2.1.6]{elephant}. It is also equivalent to the category of sheaves over finite graphs and embeddings with the atomic topology. See~\cite{caramello-fraisse,caramello-toposic-galois} for general discussion.
\end{proof}

      \subsection{Probability Measures on Rado-Nominal Sets}\label{sec:rado-prob}
      The finitely supported sets $S\subseteq \VV$ can be regarded as
      `events' to which we would assign a probability. For example,
      if we already have vertices $b$ and $c$, we may want to know the
      chance
      of picking a vertex that forms a horn, and this would be
      the probability of the set in Ex.~\ref{example:subset}(a).
      \begin{definition}\label{def:probability}
        A sequence $S_1,S_2\dots \subseteq X$ is said to be
        \emph{support-bounded} if there is one finite set $A\subseteq\VV$ that
        supports all the sets $S_i$.

        A function $\mu:2^X\to[0,1]$ is \emph{(internally)
          countably additive} if for any
        support-bounded sequence $S_1,S_2\dots \subseteq X$ of disjoint sets, 
        \[\textstyle
          \mu(\biguplus_{i=1}^\infty S_i)=\sum_{i=1}^\infty\mu(S_i)\text.
        \]
        
        A \emph{probability measure} on a Rado-nominal set $X$ is an equivariant function
        $\mu:2^X\to [0,1]$ that is internally countably additive, such that $\mu(X) = 1$. \end{definition}
      We remark that there are two subtleties here. First,
      we restrict to support-bounded sequences. These are the
      correctly internalized notion of sequence in Rado-nominal
      sets,
      since they correspond precisely to finitely-supported functions $\NN\to
      2^X$.
      Second, we consider a Rado-nominal set to be equipped with its internal powerset $2^X$, 
      rather than considering 
      sub-$\sigma$-algebras.

%%% Local Variables:
%%% mode: latex
%%% TeX-master: "popl24.tex"
%%% End:
