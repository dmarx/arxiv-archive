\paragraph*{Denotational semantics} We interpret types in this language as finite sets.
% Every set $X \in \Set$ is seen as a measurable space equipped with the discrete $\sigma$-algebra~$2^{X}$.
\begin{equation}
  \semD 1=\{\star\}
  \quad
  \semD {A\ast B}=\semD A\times \semD B
  \quad
  \semD {\tbool}=\{0,1\}\text.\label{eqn:finset}
\end{equation}

\begin{definition}[e.g.~\cite{jacobs-coalgebra}]\label{def:distribution-monad} The distribution monad $\DistM$ on $\Set$ is defined as follows:
\begin{itemize}
 \item On objects: each set $X$ is mapped to the set of all finitely-supported discrete probability measures $X$, that is, all functions $p : X \to \RR$ that are non-zero for only finitely many elements and satisfy $\sum_{x \in X} p(x) = 1$.
% \item On morphisms: $\DistM f:\DistM X \to \DistM Y$ is the pushforward: 
% \[
% \DistM f(p)(y) = \textstyle\sum_{x \in f^{-1}(y)} p(x)
% \] 
% for all $p \in \DistM X$, $f : X \to Y$, and $y \in Y$.
 \item The unit $\eta_X :  X\to \DistM (X)$ maps $x \in X$ to the indicator function 
 $\lambda y.\, [y=x]$, i.e.~the Dirac distribution~$\delta_x$.
 \item The bind function $(\bind)$ is defined as follows: 
 \[
         (f \bind g)(z)(y) = \textstyle\sum_{x \in X} f(z)(x) \cdot g(z,x)(y)
 \] 
%for $f\colon  Z \to \DistM(X)$, $g \colon Z {\times} X \to \DistM(Y)$, $z \in Z$, 
%${y \in Y}$.
% for all $f : Z \to \DistM(X)$, $g : Z \times X \to \DistM(Y)$, $z \in Z$
 %and $y \in Y$.
% \item The multiplication $\mu_X : \DistM^2X \to \DistM X$ is given by: 
% \[
% \mu_A(P)(x) = \sum_{p \in \DistM X} P(p) \cdot p(x)
% \] 
% for all $P \in \DistM^2(A)$ and $x \in X$.
 \end{itemize}
\end{definition}
By the standard construction for strong monads, each morphism $f : X \to Y$ 
gets mapped to $\DistM f:\DistM X \to \DistM Y$ that is the pushforward in this case: 
$\DistM f(p)(y) = \textstyle\sum_{x \in f^{-1}(y)} p(x)$.
%for all $p \in \DistM X$ and $y \in Y$.

% \paragraph*{Kleisli composition} A map $f:X \to \DistM Y$ is a probability kernel if for every $x \in X$, $f(x)(-) = f(x, -)$ is a probability distribution over $Y$. For two maps $f:X \to \DistM Y$ and $g: Y \to \DistM Z$, the composite is given by marginalizing over $\DistM Y$ (Chapman-Kolmogorov):
% \[
%   (g \comp f)(x)(z) = \sum_{y \in Y} f(x, y)\cdot g(y, z)
% \]

The semantics of the Bernoulli probabilistic language is in the Kleisli category of $\DistM$, where computations $\Gamma \vdash t: A$ are interpreted as maps $\semD{t} : \semD{\Gamma} \to \DistM\semD{A}$~\cite{moggi:computation_and_monads}. It follows the semantics described in Section~\ref{sec:monad}.

% \begin{proposition} $\Set$ is symmetric monoidal category with monoidal product given by cartesian product $X \tensor Y := X \times Y$. Then, the distribution monad $\DistM$ is left- and right-strong. Left-strength $l: X \times \DistM Y \to \DistM(X \times Y)$ sends a pair $(x, p)$ to a distribution $l(x, y) = l(x)(y)$. This gives a map $\productmeas: \DistM X \times \DistM Y \to \DistM(X \times Y)$, which, for a pair of marginal distributions over $X$ and $Y$, gives their product distribution.
% \end{proposition}

% \begin{proposition} There is also a map in the other direction $\marginal: \DistM(X \times Y) \to \DistM X \times \DistM Y$, which, for a distribution $p(-, -)$ over $X \times Y$ gives the pair of marginal distributions over $X$ and $Y$:
% \[
% \marginal(p) = \langle x \mapsto \sum_{y \in Y} p(-, y) , y \mapsto \sum_{x \in Y} p(x, -) \rangle
% \]
% \end{proposition}
% \todo[inline]{What are the properties of the marginal map?}

% \subsection{Semantics of the Bernoulli language}
% Semantics of the Bernoulli probabilistic programming language $\semD{-}$ is given in the Kleisli category for $\DistM$.
% \begin{itemize}

% \item $\semD{-}$ on types gives objects in $\Set$
%     \begin{align*}
%       \semD{\tbool} &= 1 + 1 \\
%       \semD{()} &= 1 \\
%       \semD{A * B} &= \semD{A}\times\semD{B}
%     \end{align*}

% \item $\semD{-}$ on contexts gives objects in $\Set$
%     \begin{align*}
%       \semD{-} &= 1 \\
%       \semD{\Gamma, (x: A)} &= \semD{\Gamma} \times \semD{A}
%     \end{align*}

% \item $\semD{\Gamma \vdash t: A}$ gives a Kleisli map $\semD{t} \in Hom(\semD{\Gamma}, \DistM\semD{A})$
%     \begin{align*}
%       \semD{\Gamma, (x_n: A_n), \Gamma' \vdash x_n: A_n} =& \eta_{A_n} \comp \pi_n \\
%       \semD{\Gamma \vdash \ttrue: \tbool} =& \eta_{\tbool} \comp \iota_1 \comp 1 \\
%       \semD{\Gamma \vdash \tfalse: \tbool} =& \eta_{\tbool} \comp \iota_2 \comp 1 \\
%       \semD{\Gamma \vdash (t_1, t_2): (A_1 * A_2)} =& s \comp \langle \semD{\Gamma \vdash t_1: A_1}, \semD{\Gamma \vdash t_2: A_2}\rangle \\
%       \semD{\Gamma \vdash \pi_1 t: A_1} =& \pi_1 \comp \marginal \comp \semD{\Gamma \vdash t: (A_1 \times A_2)} \\
%       \semD{\Gamma \vdash \pi_2 t: A_2} =& \pi_2 \comp \marginal \comp \semD{\Gamma \vdash t: (A_1 \times A_2)} \\
%       \semD{\Gamma \vdash \letin x t u: B} =& \semD{\Gamma, (x: A) \vdash u: B} \\
%       & \comp (id_\Gamma, \semD{\Gamma \vdash t : A}) \\
%       \semD{\Gamma \vdash \tbernoulli_r(): \tbool} =& (r\delta_{\iota_1} + (1-r)\delta_{\iota_2}) \comp 1 \\
%       \semD{\Gamma \vdash \ite u {t_1} {t_2} A} =& (\semD{\Gamma \vdash t_1: A} + \semD{\Gamma \vdash t_2: A}) \\
%       & \comp (\semD{\Gamma \vdash u: \tbool} \times id) \comp \Delta_{\semD{\Gamma}}
%       \end{align*}
% \end{itemize}
% In case of deterministic terms, we simply lift the semantics in finite sets by $\eta$. If the term does not depend on the context, we pre-compose with $1: \Gamma \to 1$. In the let case, given a map $\semD{ u }: \semD{\Gamma} \to \DistM(1 + 1)$ and two maps $\semD{t_1}, \semD{t_2} : \semD{\Gamma} \to \DistM\semD{A}$ we have to give a map $\semD{\Gamma} \to \DistM\semD{A}$; it is constructed as the composition:
% \begin{align*}
%   \semD{\Gamma}
%   \xrightarrow{\Delta_{\semD{\Gamma}}}
%   \semD{\Gamma} \times& \semD{\Gamma}
%   \xrightarrow{(\semD{u} \times id)}
%   2 \times \semD{\Gamma} \simeq \\
%   &\simeq  \semD{\Gamma} + \semD{\Gamma} \xrightarrow{\semD{t_1} + \semD{t_2}}
%   G\semD{A}
% \end{align*}
Since we interpret types in the language as finite sets, the Kleisli morphisms are just stochastic matrices, that is, matrices where each row sums to $1$. Therefore, terms $\Gamma\vdash t:A$ can be seen as matrices
$  \semD t\in \RR^{\semD{\Gamma }\times \semD A}$
%\begin{equation}
%\end{equation}
and the Kleisli composition is then just matrix multiplication.
