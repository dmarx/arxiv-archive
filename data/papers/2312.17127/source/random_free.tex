
\section{Interpretation: black-and-white graphons via measure-theoretic probability}
\label{sec:randomfree}
In Section~\ref{sec:graphons-to-equational-theories}, we gave a general
syntactic construction for building an equational theory from a
graphon. Since that definition is based on free constructions and quotients, \emph{a priori}, it does not `explain' what the type $\tvertex$ stands for. Like contextual equivalence of programs, \emph{a priori}, it does not give useful compositional reasoning methods. To prove two programs are equal, according to the construction of Prop.~\ref{prop:quotient}, one needs to quantify over all $Z$, $h$, and $k$, in general.

In this section, we show that one class of graphons, black-and-white
graphons (Def.~\ref{def:bw}), admits a straightforward measure-theoretic semantics, and we can thus 
use the equational theory induced by this semantics, rather than the 
method of Section~\ref{sec:graphons-to-equational-theories}. This measure-theoretic semantics is close to 
previous measure-theoretic work on probabilistic programming
languages (e.g.~\cite{Kozen81,s-finite}). 

After recapping measure-theoretic probability (\S\ref{sec:probthy}), in Section~\ref{sec:bw-theory}, we show that every
black-and-white graphon arises from a measure-theoretic
interpretation (Prop.~\ref{prop:bw-ok}). In Section~\ref{sec:all-bw}, by defining `measure-theoretic interpretation' more
generally, we show that, conversely, this measure-theoretic approach can \emph{only} cater
for black-and-white graphons (Prop.~\ref{prop:all-bw}).


\subsection{Black-and-White Graphons from Equational Theories}\label{sec:bw-theory}
\begin{definition}\label{def:bw}[e.g.~\cite{janson}]
  A graphon $W:[0,1]^2\to [0,1]$ is \emph{black-and-white} if there exists
$E:[0,1]^2\to \{0,1\}$ such that 
$W(x,y)=E(x,y)$ for almost all $x,y$.
\end{definition}
Recall that the Giry monad (Def.~\ref{def:giry}) gives rise to a Bernoulli-based distributive Markov category  (\S\ref{sec:probthy}, Ex.~\ref{ex:bernoulli-base}).
For any black-and-white graphon~$W$, we define an interpretation of
the graph interface for the probabilistic programming language
using $\Giry$, as follows.
\begin{itemize}
\item $\sem\tvertex_W = [0,1]$; $\sem \tbool_W= 2$, the discrete two element space;
\item $\sem{\tnew()}_W = \mathrm{Uniform}(0,1)$, the uniform distribution on $[0,1]$;
\item $\sem{\tedge}_W(x,y)=\eta(E(x,y)).
        $
\end{itemize}
\begin{proposition}\label{prop:bw-ok}
  Let $W$ be a black-and-white graphon. The equational theory induced by $\sem-_W$ induces the graphon~$W$ according to the construction in Section~\ref{sec:program-equivalence-graphons}.
\end{proposition}
\begin{proof}
%  Equations~(\ref{cd:let_val})\,--\,(\ref{cd:beta-eta}) follow from Prop.~\ref{prop:monad-eq-theory} and Prop.~\ref{prop:giry-comm}.
  
 % For Bernoulli-based, we just recall that for the basic types, the range of the interpretation is within finite measurable spaces and probability kernels, which, as discussed in Ex.~\ref{giry-bernoulli-lang}, induces the same equational theory as the distribution semantics of the Bernoulli language.

Suppose that $W$ corresponds to the sequence of random graphs $p_1,p_2,\dots $ as in Section~\ref{sec:graphons}. Consider the term $t_n$ in \eqref{eqn:programgraph}, and directly calculate its interpretation. Then, we get $\sem {t_n}_W=p_n$, via~\eqref{eqn:pwg}, as required. 

  The choice of $E$ does not matter in the interpretation of these terms, because $W=E$ almost everywhere.
\end{proof}
\subsection{All Measure-Theoretic Interpretations are
  Black-and-White}\label{sec:all-bw}
Although the model in Section~\ref{sec:bw-theory} is fairly canonical, there are sometimes other enlightening interpretations using the Giry monad. These also correspond to black-and-white graphons.

For example, consider the geometric-graph example from Figure~\ref{fig:graph}. We interpret this using the Giry monad, putting
\begin{itemize}
\item $\sem\tvertex = S_2$, the sphere; $\sem \tbool= 2$;
\item $\sem{\tnew()} = \mathrm{Uniform}(S_2)$, the uniform distribution on the sphere;
\item $\sem{\tedge}(x,y)=\eta(d(x,y)<θ)$, i.e.~an edge if their
  distance is less than $\theta$. 
\end{itemize}
This will again induce a graphon, via~(\ref{eqn:programgraph}).
We briefly look at theories that arise in this more flexible way:
%\begin{definition}\label{def:meas-th-interp}
 % A \emph{measure-theoretic interpretation} of the graph probabilistic programming language is given by specifying a standard Borel space $\sem\tvertex$,
%  a measurable set $\sem\tedge\subseteq \sem\tvertex^2$, 
 % and a probability measure $\sem{\tnew()}$ on $\sem\tvertex$. 
%\end{definition}
%Every measure-theoretic interpretation induces a good equational theory, by Prop.~\ref{prop:monad-eq-theory}, for the same reason as the canonical graphon interpretation (\S\ref{sec:bw-theory}).
\begin{proposition}
  \label{prop:all-bw}
  Consider any interpretation of the graph interface in the Giry monad:
  a measurable space $\sem\tvertex$,
  a measurable set $\sem\tedge\subseteq \sem\tvertex^2$, 
  and a probability measure $\sem{\tnew()}$ on $\sem\tvertex$.
  The induced graphon is black-and-white.
\end{proposition}
\begin{proof}[Proof notes]
 If $\sem\tvertex$ is standard Borel, the randomization lemma~\cite[Lem.~3.22]{kallenberg-2010} gives
  a function $f:[0,1]\to \sem\tvertex$ that pushes the uniform distribution on $[0,1]$ onto the probability measure $\sem{\tnew()}$.
        We define a black-and-white graphon $W$ by $W(x,y)=1$ if
        $(f(x),f(y)) \in \sem\tedge$, and $W(x,y)=0$ otherwise. 
This graphon interpretation $\sem-_W$ gives the same sequence of graphs in~\eqref{eqn:programgraph}, just by reparameterizing the integrals.

  If $\sem\tvertex$ is not standard Borel, we note that there is an equivalent interpretation where it is, because there exists a measure-preserving map $\sem\tvertex\to \Omega$ to a standard Borel space $\Omega$
  and a measurable set $E\subseteq\Omega^2$ that pulls back to $\sem\tedge$, giving rise to the same graphon
  (e.g.~\cite[Lemma~7.3]{janson}).
\end{proof}

\subsubsection*{Discussion}
Proposition~\ref{prop:all-bw} demonstrates that this measure-theoretic
interpretation has limitations.
%For example, consider the Erd\H{o}s-R\'enyi graphons:
\begin{definition}\label{def:e-r-graphon}
  For $α\in(0,1)$, the \emph{\ErdosRenyi\ graphon}  $W_α:[0,1]^2\to
  [0,1]$ is given by $W_α(x,y)=α$.
\end{definition}
The Erd\H{o}s-R\'enyi graphons cannot arise from measure-theoretic
interpretations of the graph interface, because they are not 
black-and-white. In Section~\ref{sec:ER-Rado}, we give an alternative
interpretation for the \ErdosRenyi\ graphons.

The reader might be tempted to interpret an \ErdosRenyi\ graphon by
defining
\[\sem{\tedge}_{W_α}(x,y)
  \quad=\quad
  \tbernoulli(\alpha).
\]
However, this interpretation does not provide a model
for the basic equations of the language, because this $\sem\tedge$
is not deterministic, and
derivable equations such as
\eqref{eqn:intro:det}
will fail. Intuitively, once an edge has been sampled between two
given nodes, its presence (or absence) remains unchanged in the rest
of the program, \ie the edge is not resampled again, it is memoized
(see also~\cite{kaddar-staton-stoch-mem,roy2008stochastic}).

Although not all graphons are black-and-white, these are
still a widely
studied and useful class. They are often called
`random-free'.
For example, an alternative characterization is that 
the random graph model of
Prop.~\ref{prop:lovasz}
has subquadratic entropy function~\cite[\S 10.6]{janson}. 
              
%%% Local Variables:
%%% mode: latex
%%% TeX-master: "popl24"
%%% End:
