
\begin{tcolorbox}[title = {Prompt Template},breakable]
\small
Compare which text \{criterion\} \\
Your judgement should not be influenced by the language the text is written in, the length of the text and the order in which the texts are presented. \\
If the texts have similar quality, you should still make a relative judgement and choose the label of the preferred text.  \\
You must respond with format: \\
"Choice: 1 or 2\textbackslash{n}Why: reason of choice"  \\ \\
Text 1:
... \{text\_1\} ...\\ \\
Text 2:
... \{text\_2\} ... \\ \\
Now you have to choose between either 1 or 2. Note that respond only with the format mentioned.
\end{tcolorbox}

\begin{tcolorbox}[title = {Educational Value},breakable]
\small
has more educational value. It has more educational value if it includes clear explanations, step-by-step reasoning, or detailed concepts which is clear enough for children to understand. \\
Prefer text which has more detailed ideas or explanations which is sufficiently clear to convey them to a child.
\end{tcolorbox}

\begin{tcolorbox}[title = {Expertise},breakable]
\small
requires greater expertise and deeper prerequisite knowledge to understand it. \\
For example, "The relativistic Dirac equation, which combines principles of quantum mechanics and special relativity, predicts the existence of antimatter and elucidates the intrinsic spin of fundamental particles." requires great physics expertise to understand.
\end{tcolorbox}

\begin{tcolorbox}[title = {Fact and Trivia},breakable]
\small
contains more facts and trivia. The facts and trivia should be accurate. \\
Prefer text which have more number of facts. Put lower priority to facts which contain mathematical calculations and with too deep "concepts and explanations" .
\end{tcolorbox}

\begin{tcolorbox}[title = {Reasoning Level},breakable]
\small
has higher reasoning level. It has high reasoning level when it requires more reasoning, logical and mathematical thinking skills or chain of thought thinking.
\end{tcolorbox}

\begin{tcolorbox}[title = {Scarcity},breakable]
\small
is more relatively unknown. It should be truthful and little known to the general public. \\
Prefer unpopular accurate facts over fictional stories.
\end{tcolorbox}

\begin{tcolorbox}[title = {Structural Format},breakable]
\small
has better structural format. It has better structural format when it has a well-defined structure such as outline format, Markdown, numbered list, bulleted list, JSON, table format, headings and subheadings format or other organizational templates. \\
First, consider the visual structure of text. Then, only consider the content or logical flow of text.
\end{tcolorbox}

\begin{tcolorbox}[title = {Story-likeness},breakable]
\small
is more likely to be a story. It is more like a story when it narrates a story or it describes a scene or situation in details.
\end{tcolorbox}

\begin{tcolorbox}[title = {Subjectivity},breakable]
\small
contains more subjectivity, e.g, it includes more subjective perspectives, opinions, personal views or feelings. Avoid choosing text which conveys objective, factual and widely accepted, accurate knowledge. \\
Prefer text which personal opinions such as dialogues or feelings over text which seems like a formal examination question and answer.
\end{tcolorbox}

\begin{tcolorbox}[title = {Generate Structural Format Data},breakable]
\small
You are tasked with generating text data that has clear and organized formatting structures. Some structural format are list, markdown, headings and subheadings, table, json, html, xml, latex, columnar formats etc. \\
The data should maintain a coherent structure with organized sections, numbering, tables, code formatting, hierarchical structure, outlines or other organizational templates where appropriate. You should not include all of the formats in one data. One data can mix of one, two or three formats. \\
You can add various knowledge and facts into data to make data more informative and longer. \\
Please generate 3 lengthy and informative examples about `<topic>` showcasing different formatting styles and content. Split examples with <split>
\end{tcolorbox}
