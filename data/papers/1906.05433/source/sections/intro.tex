\documentclass[11pt]{report}
\usepackage[margin=2cm]{geometry}
\usepackage{graphicx}
\usepackage{float}
\usepackage{times}
\usepackage{url}
\usepackage[dvipsnames]{xcolor}
\usepackage{hyperref}

\newcommand{\specialcell}[2][c]{\begin{tabular}[#1]{@{}c@{}}#2\end{tabular}}

\newcommand{\Gap}{\texorpdfstring{\hfill}{}}
\newcommand{\Rec}{\texorpdfstring{{\small\emph{\color{ccai-blue}{\fbox{High Leverage}}}}}{}}
\newcommand{\HighRisk}{\texorpdfstring{{\small\emph{\color{ccai-yellow-darker}{\fbox{Uncertain Impact}}}}}{}}
\newcommand{\Longterm}{\texorpdfstring{{\small\emph{\color{ccai-green}{\fbox{Long-term}}}}}{}}

\begin{document}

\begin{abstract}
Climate change is one of the greatest challenges facing humanity, and we, as machine learning experts, may wonder how we can help. Here we describe how machine learning can be a powerful tool in reducing greenhouse gas emissions and helping society adapt to a changing climate. From smart grids to disaster management, we identify high impact problems where existing gaps can be filled by machine learning, in collaboration with other fields. Our recommendations encompass exciting research questions as well as promising business opportunities. We call on the machine learning community to join the global effort against climate change.
\vskip .5in
\end{abstract}

\part*{Introduction}
The effects of climate change are increasingly visible.\footnote{For a layman's introduction to the topic of climate change, see \cite{romm2018climate, archer2010climate}.} Storms, droughts, fires, and flooding have become stronger and more frequent \cite{field2012managing}. Global ecosystems are changing, including the natural resources and agriculture on which humanity depends. The 2018 intergovernmental report on climate change estimated that the world will face catastrophic consequences unless global greenhouse gas emissions are eliminated within thirty years \cite{ipcc_global_2018}. Yet year after year, these emissions rise.

Addressing climate change involves mitigation (reducing emissions) and adaptation (preparing for unavoidable consequences). Both are multifaceted issues. Mitigation of greenhouse gas (GHG) emissions requires changes to electricity systems, transportation, buildings, industry, and land use. Adaptation requires planning for resilience and disaster management, given an understanding of climate and extreme events. Such a diversity of problems can be seen as an opportunity: there are many ways to have an impact.

In recent years, machine learning (ML) has been recognized as a broadly powerful tool for technological progress. Despite the growth of movements applying ML and AI to problems of societal and global good,\footnote{See the AI for social good movement (e.g.~\cite{hager2019artificial, berendt2019ai}), ML for the developing world~\cite{de2018machine}, the computational sustainability movement (e.g.~\cite{kelling2018computational, joppa2017case, lassig2016computational, gomes2009computational, dietterich2009machine}, the American Meteorological Society's Committee on AI Applications to Environmental Science, and the field of Climate Informatics (\url{www.climateinformatics.org}) \cite{Monteleoni2013chapter}, as well as the relevant survey papers \cite{faghmous2014big, kaack2019challenges, ford2016opinion}.} there remains the need for a concerted effort to identify how these tools may best be applied to tackle climate change. Many ML practitioners wish to act, but are uncertain how. On the other side, many fields have begun actively seeking input from the ML community.

This paper aims to provide an overview of where machine learning can be applied with high impact in the fight against climate change, through either effective engineering or innovative research. The strategies we highlight include climate mitigation and adaptation, as well as meta-level tools that enable other strategies. In order to maximize the relevance of our recommendations, we have consulted experts across many fields (see \hyperref[sec:acknowledgments]{{\small{Acknowledgments}}}) in the preparation of this paper.


\begin{table}
\begin{small}
\begin{center}
\begin{tabular}{l l l l l l l l l l l l}  \toprule
     \multicolumn{2}{l}{ }
         & \small{\rotatebox{90}{\parbox{2.2cm}{Causal\\inference}}}
         & \small{\rotatebox{90}{\parbox{2.2cm}{Computer\\vision}}}
         & \small{\rotatebox{90}{\parbox{2.2cm}{Interpretable\\models}}}
         & \small{\rotatebox{90}{NLP}}
         & \small{\rotatebox{90}{\parbox{2.2cm}{RL \& Control}}}
        %  & \small{\rotatebox{90}{Robotics}}
         & \small{\rotatebox{90}{\parbox{2.2cm}{Time-series analysis}}}
         & \small{\rotatebox{90}{\parbox{2.2cm}{Transfer\\learning}}}
         & \small{\rotatebox{90}{\parbox{2.2cm}{Uncertainty\\quantification}}}
         & \small{\rotatebox{90}{\parbox{2.2cm}{Unsupervised\\learning}}}
    \\ \midrule
    \rowcolor{ccai-blue-lightest}
    \multicolumn{2}{l}{1 \hyperref[sec:electricity-systems]{Electricity systems}} 
        & % Causal inf
        &  % Comp vision
        & % Interpretable ml
        & % nlp
        & % rl + control
        & % time series
        & % transfer
        & % UQ
        & \\% unsupervised \ref{sub
    & \hyperref[sec:electricity-lowCarbon]{Enabling low-carbon electricity}
        & % Causal inf
        & $\bullet$% Comp vision
        & $\bullet$% % Interpretable ml
        & % % nlp
        & $\bullet$%% rl + control
        & $\bullet$% % time series
        & % transfer
        & $\bullet$% % UQ
        & $\bullet$\\% unsupervised 
    & \hyperref[sec:electricity-currentSystemImpact]{Reducing current-system impacts}
        & % Causal inf
        & $\bullet$% Comp vision
        & % Interpretable ml
        & % nlp
        & % rl + control
        & $\bullet$% % time series
        & % transfer
        & $\bullet$% % UQ
        & $\bullet$\\% unsupervised 
    & \hyperref[sec:electricity-developing]{Ensuring global impact}
        & % Causal inf
        & $\bullet$% Comp vision
        & % Interpretable ml
        & % nlp
        & % rl + control
        & % time series
        & $\bullet$ % transfer
        & % UQ
        & $\bullet$\\% unsupervised 
    \rowcolor{ccai-blue-lightest}
    \multicolumn{2}{l}{2 \hyperref[sec:transportation]{Transportation}} 
        & % Causal inf
        & % Comp vision
        &% Interpretable ml
        & % nlp
        & % rl + control
        & % time series
        & % transfer
        & % UQ
        & \\% unsupervised 
    & \hyperref[sec:TReducing]{Reducing transport activity}
        & % Causal inf
        & $\bullet$% Comp vision
        & % Interpretable ml
        & % nlp
        & % rl + control
        & $\bullet$% time series
        & % transfer
        & $\bullet$% UQ
        & $\bullet$\\% unsupervised     
   & \hyperref[sec:TEfficient]{Improving vehicle efficiency}
        & % Causal inf
        & $\bullet$% Comp vision
        & % Interpretable ml
        & % nlp
        & $\bullet$% rl + control
        & % time series
        & % transfer
        & % UQ
        & \\% unsupervised    
   & \hyperref[sec:TFuels]{Alternative fuels \& electrification}
        & % Causal inf
        & % Comp vision
        & % Interpretable ml
        & % nlp
        & $\bullet$% rl + control
        & % time series
        & % transfer
        & % UQ
        & $\bullet$ \\% unsupervised    
   & \hyperref[sec:modalshift]{Modal shift}
        & $\bullet$% Causal inf
        & $\bullet$% Comp vision
        & % Interpretable ml
        & % nlp
        & % rl + control
        & $\bullet$% time series
        & % transfer
        & $\bullet$% UQ
        & \\% unsupervised    
    \rowcolor{ccai-blue-lightest}
    \multicolumn{2}{l}{3 \hyperref[sec:buildings-cities]{Buildings and cities}} 
        & % Causal inf
        & % Comp vision
        & % Interpretable ml
        & % nlp
        & % rl + control
        & % time series
        & % transfer
        & % UQ
        & \\% unsupervised 
    & \hyperref[sec:indv]{Optimizing buildings}
        & $\bullet$% Causal inf
        & % Comp vision
        & % Interpretable ml
        & % nlp
        & $\bullet$% rl + control
        & $\bullet$% time series
        & $\bullet$% transfer
        & % UQ
        & \\% unsupervised 
    & \hyperref[sec:distr]{Urban planning}
        & % Causal inf
        & $\bullet$% Comp vision
        & % Interpretable ml
        & % nlp
        & % rl + control
        & $\bullet$% time series
        & $\bullet$% transfer
        & % UQ
        & $\bullet$\\% unsupervised 
    & \hyperref[sec:cities]{The future of cities}
        & % Causal inf
        & % Comp vision
        & % Interpretable ml
        & $\bullet$%% nlp
        & % rl + control
        & %% time series
        & $\bullet$%% transfer
        & $\bullet$% UQ
        & $\bullet$\\% unsupervised 
    \rowcolor{ccai-blue-lightest}
    \multicolumn{2}{l}{4 \hyperref[sec:industry]{Industry}} 
        & % Causal inf
        & % Comp vision
        & % Interpretable ml
        & % nlp
        & % rl + control
        & % time series
        & % transfer
        & % UQ
        & \\% unsupervised 
    & \hyperref[sec:supplychains]{Optimizing supply chains}
        & % Causal inf
        & $\bullet$ %% Comp vision
        & % Interpretable ml
        & % nlp
        & $\bullet$ % rl + control
        & $\bullet$ % time series
        & % transfer
        & % UQ
        & \\% unsupervised 
    & \hyperref[sec:materialsandconstruction]{Improving materials}
        & %% Causal inf
        & % Comp vision
        & % Interpretable ml
        & % nlp
        & % rl + control
        & % time series
        & %% transfer
        & % UQ
        & $\bullet$ \\% unsupervised 
    & \hyperref[sec:demandresponse]{Production \& energy}
        & %% Causal inf
        & $\bullet$%% Comp vision
        & $\bullet$ %% Interpretable ml
        & % nlp
        & $\bullet$% rl + control
        & %% time series
        & %% transfer
        & % UQ
        & \\% unsupervised 
    \rowcolor{ccai-blue-lightest}
    \multicolumn{2}{l}{5 \hyperref[sec:afolu]{Farms \& forests}} 
        & % Causal inf
        & % Comp vision
        & % Interpretable ml
        & % nlp
        & % rl + control
        & % time series
        & % transfer
        & % UQ
        & \\% unsupervised 
    & \hyperref[sec:emissions-detection]{Remote sensing of emissions}
        & % Causal inf
        & $\bullet$% Comp vision
        & % Interpretable ml
        & % nlp
        & % rl + control
        & % time series
        & % transfer
        & % UQ
        & \\% unsupervised 
    & \hyperref[sec:agriculture]{Precision agriculture}
        & % Causal inf
        & $\bullet$% Comp vision
        & % Interpretable ml
        & % nlp
        & $\bullet$% rl + control
        & $\bullet$% time series
        & % transfer
        & % UQ
        & \\% unsupervised 
    & \hyperref[sec:peatlands]{Monitoring peatlands}
        & % Causal inf
        & $\bullet$% Comp vision
        & % Interpretable ml
        & % nlp
        & % rl + control
        & % time series
        & % transfer
        & % UQ
        & \\% unsupervised 
    & \hyperref[sec:forests]{Managing forests}
        & % Causal inf
        & $\bullet$% Comp vision
        & % Interpretable ml
        & % nlp
        & $\bullet$ % rl + control
        & $\bullet$ % time series
        & % transfer
        & % UQ
        & \\% unsupervised 
    \rowcolor{ccai-blue-lightest}
    \multicolumn{2}{l}{6 \hyperref[sec:ccs]{Carbon dioxide removal}}
        & % Causal inf
        & % Comp vision
        & % Interpretable ml
        & % nlp
        & % rl + control
        & % time series
        & % transfer
        & % UQ
        & \\
    & \hyperref[sec:ccs]{Direct air capture}
        & % Causal inf
        & % Comp vision
        & % Interpretable ml
        & % nlp
        & % rl + control
        & % time series
        & % transfer
        & % UQ
        & $\bullet$\\% unsupervised 
    & \hyperref[subsubsec: sequestrativervin]{Sequestering~\cd}
        & % Causal inf
        & $\bullet$% Comp vision
        & % Interpretable ml
        & % nlp
        & % rl + control
        & % time series
        & % transfer
        & $\bullet$% UQ
        & $\bullet$\\% unsupervised 
    \rowcolor{ccai-blue-lightest}
    \multicolumn{2}{l}{7 \hyperref[sec: climate prediction]{Climate prediction}} 
        & % Causal inf
        & % Comp vision
        & % Interpretable ml
        & % nlp
        & % rl + control
        & % time series
        & % transfer
        & % UQ
        & \\% unsupervised 
    & \hyperref[sec:climate-models-params]{Uniting data, ML \& climate science}
        & % Causal inf
        & $\bullet$% Comp vision
        & $\bullet$% Interpretable ml
        & % nlp
        & % rl + control
        & $\bullet$% time series
        & % transfer
        & $\bullet$% UQ
        & \\% unsupervised 
    & \hyperref[sec:models-extreme-events]{Forecasting extreme events}
        & % Causal inf
        & $\bullet$% Comp vision
        & $\bullet$% Interpretable ml
        & % nlp
        & % rl + control
        & $\bullet$% time series
        & % transfer
        & $\bullet$% UQ
        & \\% unsupervised 
    \rowcolor{ccai-blue-lightest}
    \multicolumn{2}{l}{8 \hyperref[sec:societal-impacts]{Societal impacts}} 
        & % Causal inf
        & % Comp vision
        & % Interpretable ml
        & % nlp
        & % rl + control
        & % time series
        & % transfer
        & % UQ
        & \\% unsupervised 
    & \hyperref[subsub:ecology]{Ecology}
        & % Causal inf
        & $\bullet$% Comp vision
        & % Interpretable ml
        & % nlp
        & % rl + control
        & % time series
        & $\bullet$% transfer
        & % UQ
        & \\% unsupervised 
    & \hyperref[subsub:infrastructure]{Infrastructure}
        & % Causal inf
        & % Comp vision
        & % Interpretable ml
        & % nlp
        & $\bullet$% rl + control
        & $\bullet$% time series
        & % transfer
        & $\bullet$% UQ
        & \\% unsupervised 
    & \hyperref[subsub:social_systems]{Social systems}
        & % Causal inf
        & $\bullet$% Comp vision
        & % Interpretable ml
        & % nlp
        & % rl + control
        & $\bullet$% time series
        & % transfer
        & % UQ
        & $\bullet$\\% unsupervised 
    & \hyperref[subsub:crisis]{Crisis}
        & % Causal inf
        & $\bullet$% Comp vision
        & % Interpretable ml
        & $\bullet$% nlp
        & % rl + control
        & % time series
        & % transfer
        & % UQ
        & \\% unsupervised 
    \rowcolor{ccai-blue-lightest}
    \multicolumn{2}{l}{9 \hyperref[sec:geoengineering]{Solar geoengineering}} 
        & % Causal inf
        & % Comp vision
        & % Interpretable ml
        & % nlp
        & % rl + control
        & % time series
        & % transfer
        & % UQ
        & \\% unsupervised 
    & \hyperref[subsub:better-aerosols]{Understanding \& improving aerosols}
        & % Causal inf
        & % Comp vision
        & % Interpretable ml
        & % nlp
        & % rl + control
        & $\bullet$% time series
        & % transfer
        & $\bullet$% UQ
        & \\% unsupervised 
    & \hyperref[subsub:planetary-control]{Engineering a planetary control system}
        & % Causal inf
        & % Comp vision
        & % Interpretable ml
        & % nlp
        & $\bullet$% rl + control
        & % time series
        & % transfer
        & $\bullet$% UQ
        & \\% unsupervised 
    & \hyperref[subsub:impact-models]{Modeling impacts}
        & % Causal inf
        & % Comp vision
        & % Interpretable ml
        & % nlp
        & % rl + control
        & $\bullet$% time series
        & % transfer
        & $\bullet$% UQ
        & \\% unsupervised 
    \rowcolor{ccai-blue-lightest}
    \multicolumn{2}{l}{10 \hyperref[sec:tools-individuals]{Individual action}} 
        & % Causal inf
        & % Comp vision
        & % Interpretable ml
        & % nlp
        & % rl + control
        & % time series
        & % transfer
        & % UQ
        & \\% unsupervised 
    & \hyperref[sec:personal_carbon_footprint]{Understanding personal footprint}
        & $\bullet$% Causal inf
        & % Comp vision
        & % Interpretable ml
        & $\bullet$% nlp
        & $\bullet$% rl + control
        & $\bullet$% time series
        & % transfer
        & % UQ
        & \\% unsupervised 
    & \hyperref[sec:behavior_change]{Facilitating behavior change}
        & % Causal inf
        & % Comp vision
        & % Interpretable ml
        & $\bullet$% nlp
        & % rl + control
        & % time series
        & % transfer
        & % UQ
        & $\bullet$\\% unsupervised 
    \rowcolor{ccai-blue-lightest}
    \multicolumn{2}{l}{11 \hyperref[sec:toolsforsociety]{Collective decisions}} 
        & % Causal inf
        & % Comp vision
        & % Interpretable ml
        & % nlp
        & % rl + control
        & % time series
        & % transfer
        & % UQ
        &  \\% unsupervised 
    & \hyperref[sec:coordination]{Modeling social interactions}
        & % Causal inf
        & % Comp vision
        & $\bullet$ % Interpretable ml
        & % nlp
        & $\bullet$ % rl + control
        & % time series
        & % transfer
        & % UQ
        & \\% unsupervised 
    & \hyperref[sec:decisionmaking]{Informing policy}
        & $\bullet$ % Causal inf
        & $\bullet$ % Comp vision
        & % Interpretable ml
        & $\bullet$% nlp
        & % rl + control
        & % time series
        & % transfer
        & $\bullet$% UQ
        & $\bullet$\\% unsupervised 
    & \hyperref[subsec:markets]{Designing markets}
        & % Causal inf
        & % Comp vision
        & % Interpretable ml
        & % nlp
        & $\bullet$% rl + control
        & $\bullet$% time series
        & % transfer
        & % UQ
        & $\bullet$\\% unsupervised 
    \rowcolor{ccai-blue-lightest}
    \multicolumn{2}{l}{12 \hyperref[sec:education]{Education}} 
        & % Causal inf
        & % Comp vision
        & % Interpretable ml
        & $\bullet$% nlp
        & $\bullet$% rl + control
        & % time series
        & % transfer
        & % UQ
        & \\% unsupervised 
    \rowcolor{ccai-blue-lightest}
    \multicolumn{2}{l}{13 \hyperref[sec:finance]{Finance}} 
        & % Causal inf
        & % Comp vision
        & % Interpretable ml
        & $\bullet$% nlp
        & % rl + control
        & $\bullet$% time series
        & % transfer
        & $\bullet$% UQ
        & \\% unsupervised 
    \bottomrule
\end{tabular}
\caption{Climate change solution domains, corresponding to sections of this paper, matched with selected areas of ML that are relevant to each. }
\label{tab:summary}
\end{center}
\end{small}
\end{table}


\subsection*{Who is this paper written for?}

We believe that our recommendations will prove valuable to several different audiences (detailed below). In our writing, we have assumed some familiarity with basic terminology in machine learning, but do not assume any prior familiarity with application domains (such as agriculture or electric grids).\\

\textbf{Researchers and engineers:}
We identify many problems that require conceptual innovation and can advance the field of ML, as well as being highly impactful. For example, we highlight how climate models afford an exciting domain for interpretable ML (see \S\ref{sec: climate prediction}).
We encourage researchers and engineers across fields to use their expertise in solving urgent problems relevant to society.\\

\textbf{Entrepreneurs and investors:} We identify many problems where existing ML techniques could have a major impact without further research, and where the missing piece is deployment. We realize that some of the recommendations we offer here will make valuable startups and nonprofits. For example, we highlight techniques for providing fine-grained solar forecasts for power companies (see \S\ref{sec:electricity-lowCarbon}), tools for helping reduce personal energy consumption (see \S\ref{sec:behavior_change}), and predictions for the financial impacts of climate change (see \S\ref{sec:finance}). We encourage entrepreneurs and investors to fill what is currently a wide-open space.\\

\textbf{Corporate leaders:} We identify problems where ML can lead to massive efficiency gains if adopted at scale by corporate players. For example, we highlight means of optimizing supply chains to reduce waste (see \S\ref{sec:supplychains}) and software/hardware tools for precision agriculture (see \S\ref{sec:agriculture}). We encourage corporate leaders to take advantage of opportunities offered by ML to benefit both the world and the bottom line.\\

\textbf{Local and national governments:} We identify problems where ML can improve public services, help gather data for decision-making, and guide plans for future development. For example, we highlight intelligent transportation systems (see \S\ref{sec:modalshift}), techniques for automatically assessing the energy consumption of buildings in cities (see \S\ref{sec:indv}),
and tools for improving disaster management (see \S\ref{subsub:crisis}). We encourage governments to consult ML experts while planning infrastructure and development, as this can lead to better, more cost-effective outcomes. We further encourage public entities to release data that may be relevant to climate change mitigation and adaptation goals.\\

\subsection*{How to read this paper} \label{sub:howtoread}
The paper is broken into sections according to application domain (see Table \ref{tab:summary}). To help the reader, we have also included the following flags at the level of individual strategies.
\begin{itemize}
\item \textbf{\Rec} $\,$ denotes bottlenecks that domain experts have identified in climate change mitigation or adaptation and that we believe to be particularly well-suited to tools from ML. These areas may be especially fruitful for ML practitioners wishing to have an outsized impact, though applications not marked with this flag are also valuable and should be pursued.
\item \textbf{\Longterm} $\,$ denotes applications that will have their primary impact after 2040. While extremely important, these may in some cases be less pressing than those which can help act on climate change in the near term.
\item \textbf{\HighRisk} $\,$ denotes applications where the impact on GHG emissions is uncertain (for example, the \emph{Jevons paradox} may apply\footnote{The Jevons paradox in economics refers to a situation where increased efficiency nonetheless results in higher overall demand. For example, autonomous vehicles could cause people to drive far more, so that overall GHG emissions could increase even if each ride is more efficient. In such cases, it becomes especially important to make use of specific policies, such as carbon pricing, to direct new technologies and the ML behind them. See also the literature on rebound effects and induced demand.}) or where there is  potential for undesirable side effects (\emph{negative externalities}).
\end{itemize}

These flags should not be taken as definitive; they represent our understanding of more rigorous analyses within the domains we consider, combined with our subjective evaluation of the potential role of ML in these various applications.

Despite the length of the paper, we cannot cover everything. There will certainly be many applications that we have not considered, or that we have erroneously dismissed. We look forward to seeing where future work leads.

\subsection*{A call for collaboration}

All of the problems we highlight in this paper require collaboration across fields. As the language used to refer to problems often varies between disciplines, we have provided keywords and background reading within each section of the paper. Finding collaborators and relevant data can sometimes be difficult; for additional resources, please visit the website that accompanies this paper: \url{https://www.climatechange.ai/}.

Collaboration makes it easier to develop effective strategies. Working with domain experts reduces the chance of using powerful tools when simple tools will do the job, of working on a problem that isn't actually relevant to practitioners, of overly simplifying a complex issue,
or of failing to anticipate risks.

Collaboration can also help ensure that new work reaches the audience that will use it. To be impactful, ML code should be accessible and published using a language and a platform that are already popular with the intended users. For maximal impact, new code can be integrated into an existing, widely used tool.

We emphasize that machine learning is not a silver bullet. The applications we highlight are impactful, but no one solution will ``fix'' climate change. There are also many areas of action where ML is inapplicable, and we omit these entirely. Furthermore, technology alone is not enough -- technologies that would address climate change have been available for years, but have largely not been adopted at scale by society. While we hope that ML will be useful in reducing the costs associated with climate action, humanity also must decide to act.

\end{document}
