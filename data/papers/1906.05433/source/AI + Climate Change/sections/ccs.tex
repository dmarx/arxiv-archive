\documentclass[11pt]{report}
\usepackage[margin=2cm]{geometry}
\usepackage{graphicx}
\usepackage{float}
\usepackage{times}
\usepackage{url}
\usepackage[dvipsnames]{xcolor}
\newcommand{\carbon}{\texorpdfstring{CO\textsubscript{2}}{CO2}}

\newcommand{\Gap}{\texorpdfstring{\hfill}{}}
\newcommand{\Rec}{\texorpdfstring{{\small\emph{\color{blue}{\fbox{High Leverage}}}}}{}}
\newcommand{\HighRisk}{\texorpdfstring{{\small\emph{\color{orange}{\fbox{Uncertain Impact}}}}}{}}
\newcommand{\Longterm}{\texorpdfstring{{\small\emph{\color{OliveGreen}{\fbox{Long-term}}}}}{}}

\begin{document}

\section{Carbon Dioxide Removal\texorpdfstring{\hfill\textit{by Andrew S.~Ross and Evan D.~Sherwin}}{}}
\label{sec:ccs}
Even if we could cut emissions to zero today, we would still face significant climate consequences from greenhouse gases already in the atmosphere. 
Eliminating emissions entirely may also be tricky, given the sheer diversity of sources (such as airplanes and cows). Instead, many experts argue that to meet critical climate goals, global emissions must become net-negative---that is, we must remove more \carbon~from the atmosphere than we release \cite{fuss_betting_2014,gasser2015negative}. Although there has been significant progress in negative emissions research \cite{board2019negative,icef2018roadmap,minx2018negative,fuss2018negative,nemet2018negative}, the actual \carbon~removal industry is still in its infancy. As such, many of the ML applications we outline in this section are either speculative or in the early stages of development or commercialization.

Many of the primary candidate technologies for \carbon~removal directly harness the same natural processes which have (pre-)historically shaped our atmosphere.
One of the most promising methods is simply allowing or encouraging more natural uptake of \carbon~by plants (whose ML applications we discuss in \textsection\ref{sec:afolu}).
Other plant-based methods include bioenergy with carbon capture and \emph{biochar}, where plants are grown specifically to absorb \carbon~and then burned in a way that sequesters it (while creating energy or fertilizer as a useful byproduct) \cite{creutzig2015bioenergy,robledo2017bioenergy,board2019negative}. Finally, the way most of Earth's \carbon~has been removed over geologic timescales is the slow process of mineral weathering, which also initiates further \carbon~absorption in the ocean due to alkaline runoff \cite{schuiling2006enhanced}. These processes can both be massively accelerated by human activity to achieve necessary scales of \carbon~removal \cite{board2019negative}. However, although these biomass, mineral, and ocean-based methods are all promising enough as techniques to merit mention, they may have drawbacks in terms of land use and potentially serious environmental impacts, and (more relevantly for this paper) they would not likely benefit significantly from ML.

\subsection{Direct air capture\Gap \Longterm}
\label{subsec:dac}
Another approach is to build facilities to extract \carbon~from power plant exhaust, industrial processes, or even ambient air \cite{rubin_cost_2015}. While this ``direct air capture'' (DAC) approach faces technical hurdles, it requires little land and has, according to current understanding, minimal negative environmental impacts \cite{creutzig2019mutual}. The basic idea behind DAC is to blow air onto \carbon~sorbents (essentially like sponges, but for gas), which are either solid or in solution, then use heat-powered chemical processes to release the \carbon~in purified form for sequestration \cite{board2019negative, icef2018roadmap}. Several companies have recently been started to pilot these methods.\footnote{\url{https://carbonengineering.com/}}\footnote{\url{https://www.climeworks.com/}}\footnote{\url{https://globalthermostat.com/}}

While \carbon~sorbents are improving significantly \cite{zelevnak2008amine,cashin2018surface}, issues still remain with efficiency and degradation over time, offering potential (though still speculative) opportunities for ML. ML could be used (as in \S\ref{sec:electricity-materials}) to accelerate materials discovery and process engineering workflows \cite{raccuglia2016machine,liu2017materials,gomez2018automatic,butler2018machine} to maximize sorbent reusability and \carbon~uptake while minimizing the energy required for \carbon~release. ML might also help to develop corrosion-resistant components capable of withstanding high temperatures, as well as optimize their geometry for air-sorbent contact (which strongly impacts efficiency \cite{holmes2012air}).

\subsection{Sequestering \carbon \Gap \Rec
\Longterm \HighRisk}
\label{subsubsec: sequestrativervin}
Once \carbon~is captured, it must be sequestered or stored, securely and at scale, to prevent re-release back into the atmosphere.
The best-understood form of \carbon~sequestration is direct injection into geologic formations such as saline aquifers, which are generally similar to oil and gas reservoirs \cite{board2019negative}.
A Norwegian oil company has successfully sequestered \carbon~from an offshore natural gas field in a saline aquifer for more than twenty years \cite{zoback_earthquake_2012}.
Another promising option is to sequester \carbon~in volcanic basalt formations, which is being piloted in Iceland \cite{snaebjornsdottir2016co2}.

Machine learning may be able to help with many aspects of \carbon~sequestration.
First, ML can help identify and characterize potential storage locations. Oil and gas companies have had promising results using ML for subsurface imaging based on raw seismograph traces \cite{araya2018deep}. These models and the data behind them could likely be repurposed to help trap \carbon~rather than release it.
Second, ML can help monitor and maintain active sequestration sites. Noisy sensor measurements must be translated into inferences about subsurface \carbon~flow and remaining injection capacity \cite{celia2015status}; recently, \cite{mo2019deep} found success using convolutional image-to-image regression techniques for uncertainty quantification in a global \carbon~storage simulation study. Additionally, it is important to monitor for \carbon~leaks \cite{moriarty2014rapid}. ML techniques have recently been applied to
monitoring potential \carbon~leaks from wells \cite{chen2018geologic}; computer vision approaches for emissions detection (see \cite{wang2019machine} and \textsection\ref{sec:emissions-detection}) may also be applicable.

\subsection{Discussion}
Given limits on how much more \carbon~humanity can safely emit and the difficulties associated with eliminating emissions entirely, \carbon~removal may have a critical role to play in tackling climate change.
Promising applications for ML in  \carbon~removal include informing research and development of novel component materials, characterizing geologic resource availability, and monitoring underground \carbon~in sequestration facilities.
Although many of these applications are speculative, the industry is growing, which will create more data and more opportunities for ML approaches to help.
\end{document}