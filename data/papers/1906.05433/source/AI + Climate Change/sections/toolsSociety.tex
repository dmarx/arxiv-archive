\documentclass[11pt]{report}
\usepackage[margin=2cm]{geometry}
\usepackage{graphicx}
\usepackage{float}
\usepackage{times}

\newcommand{\Gap}{\texorpdfstring{\hfill}{}}
\newcommand{\Rec}{\texorpdfstring{{\small\emph{\color{blue}{\fbox{High Leverage}}}}}{}}
\newcommand{\HighRisk}{\texorpdfstring{{\small\emph{\color{orange}{\fbox{Uncertain Impact}}}}}{}}
\newcommand{\Longterm}{\texorpdfstring{{\small\emph{\color{OliveGreen}{\fbox{Long-term}}}}}{}}

\usepackage[dvipsnames]{xcolor}

\begin{document}
\section{Collective Decisions\texorpdfstring{\hfill\textit{by Tegan Maharaj and Nikola Milojevic-Dupont}}{}}
\label{sec:toolsforsociety}

Addressing climate change requires swift and effective decision-making by groups at multiple levels -- communities, unions, NGOs, businesses, governments, intergovernmental organizations, and many more. Such collective decision-making encompasses many kinds of action -- for example, negotiating international treaties to reduce GHG emissions, designing carbon markets, building resilient infrastructure, and establishing community-owned solar farms.
These decisions often involve multiple stakeholders with different goals and priorities, requiring difficult trade-offs. The economic and societal systems involved are often extremely complex, and the impacts of climate-related decisions can play out globally across long time horizons. 
To address some of these challenges, researchers are using empirical and mathematical methods from fields such as policy analysis, operations research, economics, game theory, and
computational social science; there are many opportunities for ML to support and supplement these methods.  


\subsection{Modeling social interactions}\label{sec:coordination}

When designing climate change strategies, it is critical to understand how organizations and  individuals act and interact in response to different incentives and constraints. Agent-based models (ABMs) \cite{de2014agent, 10.2307/j.ctt7rxj1} represent one approach used in simulating the actions and interactions of \emph{agents} (people, companies, etc.) in their environment. ABMs have been applied to a multitude of problems relevant to climate change, in particular to study low-carbon technology adoption \cite{RAI2015163, Haelg_2018, zhang2012agent, NOORI2016215}. 
For example, when modeling solar PV adoption \cite{Zhang2016abm}, agents may represent individuals who act based on factors such as financial interest and the behavior of their peers \cite{rai2016overcoming, bollinger2012peer}; the goal is then to study how these agents interact in response to different conditions, such as electricity rates, subsidy programs, and geographical considerations. 
Other applications of ABMs include modeling how behavior under social norms changes with external pressures \cite{doi:10.1098/rspb.2015.2431}, how the economy and climate may evolve given a diversity of political and economic beliefs \cite{geisendorf}, and how individuals may migrate in response to environmental changes \cite{thober}. While agent and environment models in ABMs are often hand-designed by experts, ML can help integrate data-driven insights into these models \cite{Zhang2019}, for example by learning rules or models for agents based on observational data \cite{Zhang2016abm, gunaratne2018evolutionary}, or by using unsupervised methods such as VAEs or GANs to discover salient features useful in modeling a complex environment. While the hope of learning or tuning behavior from data is promising for generalization, many data-driven approaches lose the interpretability for which ABMs are valued; work in interpretable ML methods could potentially help with this.

In addition to ABMs, techniques from game theory can be valuable in modeling behavior, e.g.~to explore cooperation in the face of a depleting resource \cite{hilbe2018evolution}. Multi-agent reinforcement learning can also be applied to understand the behavior of groups of agents who need to cooperate; see \cite{panait2005MARL} for an overview and \cite{lee2019imitate, jaques2018motivation} for recent examples. Combined with mechanism design\footnote{Mechanism design is often called ``inverse game theory'' -- rather than determining optimal strategies for players, mechanism design seeks to design games such that certain strategies are incentivized.}, such approaches can be used to design methods for cooperation that lead to mutually beneficial outcomes, for example when formalizing procedures around international climate agreements \cite{mechclim, Procaccia:2013:CCJ:2483852.2483870}.

\subsection{Informing policy}
\label{sec:decisionmaking}

The actions required to address climate change, both in mitigation and adaptation, require making policies\footnote{\emph{Policy} can refer, for example, to laws, measures, standards, or best practices.} at the local, national, and international levels \cite{sterner2019}. Various institutions act as policy-makers:~for instance, governments, international organizations, non-governmental organizations, standards committees, and professional institutions.  Tools from \emph{policy analysis} -- the process of evaluating the outcomes of past policies and assessing future policy alternatives\footnote{The former is often referred to as \textit{ex-post policy analysis} and the latter as \textit{ex-ante policy analysis}.} -- can help inform the choices these institutions make.
Policy analysis uses quantitative tools from statistics, economics, and operations research such as cost-benefit analysis, uncertainty analysis, and multi-criteria decision making to inform the policy-making process; see \cite{morgan_2017, patton2015policy} for an introduction.
ML can provide data for policy analysis, help improve existing tools for assessing policy options, and provide new tools for evaluating the effects of policies.

\paragraph*{Gathering data}\Gap\mbox{\Rec}\\
When creating policies, decision-makers must often negotiate fundamental uncertainties in the underlying data.
ML can help alleviate some of this uncertainty by providing data. For instance, as detailed elsewhere in this paper, ML can help pinpoint sources of emissions (\S\ref{sec:electricity-methane},\ref{sec:emissions-detection}), approximate traffic patterns (\S\ref{sec:transport-data}), identify infrastructure at risk (\S\ref{subsub:infrastructure}), and mine information from companies' financial disclosures (\S\ref{sec:climate-analytics}).
Natural language processing, network analysis, and clustering techniques can also be used to analyze social media data to understand public opinions and discourse around climate change \cite{veltri2017climate, williams2015network, kirilenko2014public}.
These data can then be used to identify areas of intervention, compute the benefits and costs of a project, or evaluate the effectiveness of a policy after it has been implemented. 

\paragraph*{Assessing policy options}\Gap\mbox{}\\
Decision-makers often construct mathematical models to help them assess or trade off between different policy alternatives. ML is particularly relevant to approaches that model large and complex socio-economic systems to assess outcomes of particular strategies, as well as optimization-based tools that help with navigating the decision.

Policy-makers often wish to analyze how different policy alternatives may contribute to achieving a particular objective. Computational approaches such as simulation and (partial) equilibrium models can be used to compare different policy options, assess the effects of underlying assumptions, or propose strategies that are consistent with the objectives of decision-makers.
Of particular relevance to climate change mitigation are \emph{integrated assessment models} (IAMs), which incorporate economic models, climate models, and policy information (see \cite{10.1093/reep/rew018} for an overview). IAMs are used to explore future societal pathways that are consistent with climate goals (e.g.~1.5$^{\circ}$C mean global temperature increase), and play a prominent role in the IPCC assessments \cite{moss2010iams}. 
While these models can simulate interactions between many variables in great detail, this comes at the cost of computational complexity and presents opportunities for machine learning.
Much as with Earth system models (\S\ref{sec: climate prediction}), ML can be applied within any of the various sub-models that make up an IAM. One set of applications involves deriving results at the appropriate spatial resolution, since different components of an IAM operate at different scales. Outputs with high resolution may be aggregated via clustering methods to provide insights \cite{dietrich2013reducing}, while at coarser resolution, statistical \emph{downscaling} can help to disaggregate data to an appropriate spatial resolution, as seen in applications such as crop yield \cite{folberth2019spatio}, wind speed \cite{li2019geographically} or surface temperature \cite{li2019evaluation}.
ML also has the potential to help with sensitivity and uncertainty analysis \cite{JAXAROZEN2018245}, with finding numerical solutions for computational expensive submodels \cite{scheidegger2019machine, duarte2018}, and assessing the validity of the models \cite{Mori2018}.

In addition to assessing the outcomes of various policies, policy-makers may also employ optimization-based tools to figure out what decisions to make.
For example, combinatorial optimization is a powerful tool used widely for decision-making in operations research. See \cite{2018arXiv181106128B} for a survey of how ML can be employed to help solve combinatorial optimization problems.

Tools from the field of \emph{multi-criteria decision-making} can also help policy-makers manage trade-offs between different policies by reconciling competing objectives and minimizing negative side-effects; in particular, in cases where policy objectives and constraints can be mathematically formalized, \emph{multi-objective optimization} can provide a pragmatic approach to making decisions. 
Here, a decision-maker would formulate their decision-making process as an optimization problem by combining multiple optimization objectives subject to physical or other types of constraints;
the goal is to then find a solution (or set of solutions) that is \emph{Pareto-optimal} with respect to all of the objective functions.
However, finding these solutions is often computationally expensive.
Practitioners have applied bio-inspired algorithms such as particle swarm, genetic, or evolutionary algorithms to search for or compute Pareto-optimal solutions that satisfy the constraints.
This approach has been applied in a number of climate change-related fields, including energy and infrastructure planning \cite{pohekar, mattiussi, atabaki, shi_dendritic, wu2018efficiently, jhhan}, industry \cite{hassine, chaabane}, land use \cite{lakicevic, varma}, and more \cite{gutierrez, miniciardi, chen, lithuania}. 
Previous work has also employed parallel surrogate search, assisted by ML, to efficiently solve multi-objective optimization problems \cite{akhtar2019efficient}.
Optimization algorithms which have been successful in the context of hyperparameter tuning (e.g. Bayesian optimization \cite{shahriari2015taking, Snoek:2012:PBO:2999325.2999464}) or guided search algorithms (e.g. tree search algorithms \cite{alphazero}) could also potentially be applied to this problem.

\paragraph*{Evaluating policy effects}\Gap\mbox{\textbf{\Rec}}\\
When creating new policies, decision-makers may wish to understand previous policies (e.g.~from other jurisdictions) and how these policies performed.
ML can help analyze previous policy actions automatically and at scale by improving computational text analysis. In particular, natural language processing methods are already used in the field of political science to analyze political texts and legislation \cite{grimmer_stewart_2013}; these approaches could be promising for systematically studying climate change policies.
Causal inference techniques can also help assess the effect of a particular policy or climate-related event from observed outcomes.
ML can play a role in causal inference \cite{pearl2019, athey2019machine, doi:10.1080/09332480.2019.1579578}, including in the context of policy problems \cite{kreif2019m, Athey483} and in climate-relevant scenarios such as estimating the effects of temperature on human mortality \cite{hovdahl2019use} and the effects of World Bank projects on vegetative cover \cite{10.1007/978-3-319-71273-4_17}.

\subsection{Designing markets}
\label{subsec:markets}

In economics, GHG emissions can be seen as a \emph{negative externality}:~while a changing climate results in a cost for society, this cost is often not reflected in the market price of goods or services that cause GHG emissions. This is problematic, since organizations and individuals making decisions solely on the basis of market prices will tend to favor cheaper goods, even if those goods emit a large amount of GHGs. Market-based tools\footnote{For background on market-based strategies, see \cite{stiglitz2017report,stern2008economics,ellerman2010pricing}.} such as carbon taxes aim to enforce prices reflecting the societal cost of GHGs and thus encourage socially beneficial behavior through market forces. ML can help in understanding the impacts of market instruments; assessing their effectiveness at reducing emissions; and supporting a swift, effective and fair implementation.\footnote{For a review on ML for energy economics and finance, see \cite{ghoddusi2019machine}.}


\paragraph*{Predicting carbon prices}\Gap\mbox{}\\
There are several approaches to pricing GHG emissions. Carbon taxes and quotas aim to influence the behavior of organizations by shaping supply and demand within an existing market. By contrast, cap-and-trade approaches such as those within the European Union involve a completely new market, an \emph{Emissions Trading Scheme}, within which companies can buy and sell a limited number of GHG emissions permits.
Prices within such cap-and-trade markets are highly sensitive to control elements such as the number of permits released at a given time. 
ML can be used to analyze prices within these markets, for example by predicting prices via supervised learning \cite{zhu2017forecasting,sun2018analysis,zhu2018novel,wei} or analyzing the main drivers of prices via hierarchical clustering \cite{zhu2015carbon}.

\paragraph*{Non-carbon markets}\Gap\mbox{}\\
Market design can influence GHG emissions even in settings where such emissions are not directly penalized. For instance, dynamic pricing in electricity markets -- varying the price of electricity to consumers based on, e.g.,  how much wind power is available -- can shape demand for low-carbon energy sources (see \S\ref{sec:dispatchDR}). 
Following seminal research on modeling pricing in markets as a bandit problem \cite{rothschild}, many works have applied bandit and other reinforcement learning (RL) algorithms to determine prices or other market values. For example, RL has been applied to predict bids \cite{ragupathi} and market power \cite{nanduri} in electricity markets, and to set dynamic prices in more general settings 
\cite{maestre}. ML can also help solve auctions in supply chains \cite{Sandholm_very-large-scalegeneralized}.

\paragraph*{Assessing market effects}\Gap\mbox{}\\
When designing market-based strategies, it is necessary to understand how effectively each strategy will reduce emissions, as well as how the underlying socio-technical system may be affected. 
Studies have considered effects of carbon pricing on economic growth and energy intensity \cite{fang2017investigating,fang2018optimize}, or on electricity prices \cite{nagapurkar2019techno}. 
Effects of pricing mechanisms can also be indirect, as companies' strategic decisions can have longer-term effects. ML can be useful in analyzing these effects. For example, self-organizing maps have been used to analyze how R\&D investment in green technologies changes in response to fuel prices \cite{barbieri2016fuel}, while a game theoretical framework using neural networks has been used to study the optimal production strategies for companies under carbon quotas \cite{zheng2018optimal}.


To ensure that market-based strategies are effective and equitable, it is important to understand their distributional effects, as certain social groups or classes of stakeholders may be affected more than others. For example, a flat carbon tax on gasoline will have a larger effect on lower-income populations, as fuel expenses are a bigger share of their total budget. Here, clustering can help identify permit allocation schemes that maximize social welfare \cite{wu2019research}, and supervised learning has been used to predict winners and losers from changing electricity tariff schemes \cite{granell2014predicting}. \emph{Hedonic pricing} can also help identify how much different consumers may be willing to pay for a environmental good or a service, which is a noisy measure for the monetary value of that good or service;  these values are typically inferred using regression or ML techniques on historical market data \cite{hedonic_residential, hedonic_park, hedonic_theory,hedonic_air}.
It is also important to analyze which organizations or individuals can actually participate in a given market.
For example, carbon markets can be more flexible if viable offsets exist, including those offered by landowners who sequester carbon through forest conservation and management; ML has been used to examine the factors influencing the financial viability of such projects \cite{kerchner2015california}. 


\subsection{Discussion}
The complexity, scale, and fundamental uncertainty inherent in the problems of climate change can pose challenges for collective decision-making.
ML can help supplement existing mathematical frameworks that are employed to alleviate some of these challenges, including agent-based models, integrated assessment models, multi-objective optimization, and market design. 
Interpretable and fair ML techniques may be of particular importance in this context, as they may enable decision-makers to more effectively and equitably employ insights from ML models.
While these quantitative assessment tools can provide useful input to the decision-making process, it is worth noting that decisions regarding climate change may ultimately depend on qualitative discussions around norms, values, or equity considerations that may not be captured in quantitative models.
\end{document}
