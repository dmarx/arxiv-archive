\documentclass[11pt]{report}
\usepackage[margin=2cm]{geometry}
\usepackage{graphicx}
\usepackage{float}
\usepackage{times}
\usepackage[dvipsnames]{xcolor}
\newcommand{\carbon}{CO\textsubscript{2}}

\newcommand{\Gap}{\texorpdfstring{\hfill}{}}
\newcommand{\Rec}{\texorpdfstring{{\small\emph{\color{blue}{\fbox{High Leverage}}}}}{}}
\newcommand{\HighRisk}{\texorpdfstring{{\small\emph{\color{orange}{\fbox{Uncertain Impact}}}}}{}}
\newcommand{\Longterm}{\texorpdfstring{{\small\emph{\color{OliveGreen}{\fbox{Long-term}}}}}{}}

\begin{document}

\section{Education\texorpdfstring{\hfill\textit{by Alexandra Luccioni}}{}}
\label{sec:education}

Access to quality education is a key part of sustainable development, with significant benefits for climate and society at large. Education contributes to improving quality of life, helps individuals make informed decisions, and trains the next generation of innovators. Education is also paramount in helping people across societies understand and address the causes and consequences of climate change and provides the skills and tools necessary for adapting to its impacts. For instance, education can both improve the resilience of communities, particularly in developing countries that will be disproportionately affected by climate change~\cite{unesco2015}, and empower individuals, especially from developed countries, to adopt more sustainable lifestyles~\cite{fisher2016}. As climate change itself may diminish educational outcomes for some populations, due to its negative effects on agricultural productivity and household income~\cite{randell2016, randell2019}, this makes providing high-quality educational interventions globally all the more important.

\paragraph*{AI in Education}\Gap\textbf{\Longterm}\label{sec:aied}\mbox{}\\There are a number of ways that AI and ML can contribute to education and teaching -- for instance by improving access to educational opportunities, helping personalize the teaching process, and stepping in when teachers have limited time.
The field of AIED (Artificial Intelligence in EDucation) has existed for over 30 years, and until recently relied on explicitly modeling content, learners, and tutoring strategies based on psychological theories of learning.
However, AIED is increasingly incorporating data-driven insights derived from ML techniques.

One important area of AIED research has been Intelligent Tutoring Systems (ITSs), which can adapt their behavior in real time according to the needs of individuals or to support collaborative learning~\cite{chaplot2018}. 
While ITSs have traditionally been defined and constructed by hand, recent approaches have applied ML techniques such as multi-armed bandit techniques to adaptively personalize sequences of learning activities~\cite{clement2013}, LSTMs to generate questions to evaluate language comprehension~\cite{du2017}, and reinforcement learning to improve the strategies used within the ITS~\cite{iglesias2009,koedinger2013}. However, there remains much work to be done to bridge the performance gap between digital and human tutors, and ML-based approaches have an important role to play in this endeavor -- for example, via natural language processing techniques for creating conversational agents~\cite{gnewuch2018}, learner analytics for classifying student profiles,~\cite{romero2008}, and adaptive learning approaches to propose relevant educational activities and exercises ~\cite{svihla2012}.~\footnote{For further background on this area, see~\cite{nkambou2010,joksimovic2018,pinkwart2016}.}

While ITSs generally focus on individualized or small-group instruction, AIED can also help provide tools that improve educational outcomes at scale for larger groups of learners. For instance, scalable, adaptive online courses could give hundreds of thousands of learners access to learning resources that they would not usually have in their local educational facilities~\cite{roll2018}. Furthermore, giving teachers guidance derived from computational teaching algorithms or heuristics could help them design better educational curricula and improve student learning outcomes~\cite{dede2009}. In this context, AIED applications can be used either as a standalone tool for independent learners or as an educational resource that frees up teachers to have more one-on-one time with students. Key considerations for creating AIED tools that can be applied across the globe include adapting to local technological and cultural needs, addressing barriers such as access to electricity and internet~\cite{khandker2009welfare1, khandker2009welfare2}, and taking into account students' computing skills, language, and culture~\cite{cakmak2014,nye2015}. 

\paragraph*{Learning about climate}
\label{sec:climate-ed}\mbox{}\\
Research has shown that educational activities centered on climate change and carbon footprints can engage learners in understanding the connection between personal and collective actions and their impact on global climate, and can enable individuals to make climate-friendly lifestyle choices such as reducing energy use~\cite{cordero2008}. There have also been proposals for interactive websites explaining climate science as well as educational interventions focusing on local and actionable aspects of sustainable development~\cite{anderson2012}. In these contexts, ML can help create personalized educational tools, for instance by generating images of future impacts of extreme weather events based on a learner's address ~\cite{schmidt2019visualizing} or by anchoring an individual's learning experience in a digital replica of their real-life location and allowing them to explore the way that climate change will impact a specific location~\cite{angel2015}.
\end{document}


