\documentclass[11pt]{report}
\usepackage[margin=2cm]{geometry}
\usepackage{graphicx}
\usepackage{float}
\usepackage{times}
\usepackage{url}

\newcommand{\Gap}{\texorpdfstring{\hfill}{}}
\newcommand{\Rec}{\texorpdfstring{{\small\emph{\color{blue}{\fbox{High Leverage}}}}}{}}
\newcommand{\HighRisk}{\texorpdfstring{{\small\emph{\color{orange}{\fbox{Uncertain Impact}}}}}{}}
\newcommand{\Longterm}{\texorpdfstring{{\small\emph{\color{OliveGreen}{\fbox{Long-term}}}}}{}}

\usepackage[dvipsnames]{xcolor}

\begin{document}

\section{Individual Action\texorpdfstring{\hfill\textit{by Natasha Jaques}}{}}
\label{sec:tools-individuals}
Individuals may worry that they are powerless to affect climate change, or lack clarity on which of their behaviors are most important to change. In fact, there are actions which can meaningfully reduce each person's carbon footprint, and, if widely adopted, could have a significant impact on mitigating global emissions \cite{ccneedsbehaviorchange,hawken2017drawdown}. AI can help to identify those behaviors, inform individuals, and provide constructive opportunities by modeling individual behavior. 

\subsection{Understanding personal carbon footprint}
\label{sec:personal_carbon_footprint}
We as individuals are constantly confronted with decisions that affect our carbon footprint, but we may lack the data and knowledge to know which decisions are most impactful. Fortunately, ML can help determine an individual's carbon footprint from their personal and household data\footnote{See e.g.~\url{https://www.tmrow.com/}}. For example, natural language processing can be used to extract the flights a person takes from their email, or determine specific grocery items purchased from a bill, making it possible to predict the associated emissions. Systems that combine this information with data obtained from the user's smartphone (e.g.~from a ride-sharing app) can then help consumers who wish to identify which behaviors result in the highest emissions. Given such a ML model, counterfactual reasoning can potentially be used to demonstrate to consumers how much their emissions would be reduced for each behavior they changed. As a privacy-conscious alternative, emissions estimates could be directly incorporated into grocery labels \cite{supermarketfuture} or interfaces for purchasing flights. Such information can empower people to understand how they can best help mitigate climate change through behavior change.

Residences are responsible for a large share of GHG emissions \cite{ipcc_global_2018} (see also \S\ref{sec:buildings-cities}). 
A large meta-analysis found that significant residential energy savings can be achieved \cite{ehrhardt2010advanced}, by targeting the right interventions to the right households \cite{albert2016predictive, allcott2011social, allcott2014short}. ML can predict a household's emissions in transportation, energy, water, waste, foods, goods, and services, as a function of its characteristics \cite{jones2011quantifying}. These predictions can be used to tailor customized interventions for high-emissions households \cite{jones2014spatial}.
Changing behavior both helps mitigate climate change and benefits individuals; studies have shown that many carbon mitigation strategies also provide cost savings to consumers \cite{jones2011quantifying}. 

Household energy disaggregation breaks down overall electricity consumption into energy use by individual appliances (see also \S\ref{sec:indv}) \cite{armel2013disaggregation}, which can help facilitate behavior change \cite{sundramoorthy2011domesticating}. For example, it can be used to inform consumers of high-energy appliances of which they were previously unaware. This alone could have a significant impact, since many devices consume a large amount of electricity even when not in use; standby power consumption accounts for roughly 8\% of residential electricity demand \cite{mackay2008sustainable}. A variety of ML techniques have been used to effectively disaggregate household energy, such as spectral clustering, Hidden Markov Models, and neural networks  \cite{armel2013disaggregation}.

ML can also be used to predict the marginal emissions of energy consumption in real time, on a scale of hours\footnote{\url{https://www.watttime.org/}}, potentially allowing consumers to effectively schedule activities such as charging an electric vehicle when the emissions (and prices \cite{klenert2018making}) will be lowest
\cite{olivierElectricitymap}. Combining these predictions with disaggregated energy data allows for the efficient automation of household energy consumption, ideally through products that present interpretable insights to the consumer (e.g.~\cite{strbac2008demand, schweppe1989algorithms}). Methods like reinforcement learning can be used to learn how to optimally schedule household appliances to consume energy more efficiently and sustainably \cite{mocanu2018line, remani2018residential}. Multi-agent learning has also been applied to this problem, to ensure that groups of homes can coordinate to balance energy consumption to keep peak demand low \cite{ramchurn2011agent2, ygge1999homebots}.

\subsection{Facilitating behavior change \Gap \Rec}
\label{sec:behavior_change}
ML is highly effective at modeling human preferences, and this can be leveraged to help mitigate climate change. Using ML, we can model and cluster individuals based on their climate knowledge, preferences, demographics, and consumption characteristics (e.g.~\cite{beiser2018assessing,carr2011exploring,de2018graph,gabe2016householders,yang2013review}), and thus predict who will be most amenable to new technologies and sustainable behavior change. Such techniques have improved the enrollment rate of customers in an energy savings program by 2-3x \cite{albert2016predictive}. Other works have used ML to predict how much consumers are willing to pay to avoid potential environmental harms of energy consumption \cite{de2013willingness}, finding that some groups were totally insensitive to cost and would pay the maximum amount to mitigate harm, while other groups were willing to pay nothing. Given such disparate types of consumers, targeting interventions toward particular households may be especially worthwhile; all the more so because data show that the size and composition of household carbon footprints varies dramatically across geographic regions and demographics \cite{jones2011quantifying}.

Citizens who would like to engage with policy decisions, or explore different options to reduce their personal carbon footprint, can have difficulty understanding existing laws and policies due to their complexity. They may benefit from tools that make policy information more manageable and relevant to the individual (e.g.~based on where the individual lives).
There is the potential for natural language processing to derive understandable insights from policy texts for these applications, similar to automated compliance checking \cite{doi:10.1061/(ASCE)CP.1943-5487.0000427, bell2016systems}.

Understanding individual behavior can help signal how it can be nudged. For example, path analysis has shown that an individual's \textit{psychological distance} to climate change (on geographic, temporal, social, and uncertainty dimensions) fully mediates their level of climate change concern \cite{jones2017future}. This suggests that interventions minimizing psychological distance to the effects of climate change may be most effective. Similarly, ML has revealed that cross-cultural support for international climate programs is not reduced, even when individuals are exposed to information about other countries' climate behavior  \cite{beiser2019commitment}.
To make the effects of climate change more real for consumers, and thus help motivate those who wish to act, image generation techniques such as CycleGANs have been used to visualize the potential consequences of extreme weather events on houses and cities \cite{schmidt2019visualizing}. 
Gamification via deep learning has been proposed to further allow individuals to explore their personal energy usage \cite{konstantakopoulos2019deep}. All of these programs may be an incredibly cost-effective way to reduce energy consumption; behavior change programs can cost as little as 3 cents to save a kilowatt hour of electricity, whereas generating one kWh would cost 5-6 cents with a coal or wind power plant, and 10 cents with solar \cite{peerpressureOPower, forbesEnergyCost}. 

\subsection{Discussion}
While individuals can sometimes feel that their contributions to climate change are dwarfed by other factors, in reality individual actions can have a significant impact in mitigating climate change. ML can aid this process by empowering consumers to understand which of their behaviors lead to the highest emissions, automatically scheduling energy consumption, and providing insights into how to facilitate behavior change.
\end{document}
