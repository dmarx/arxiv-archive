\subsubsection{SMPL parametric body model}

\smpl is a parametric body model which allows pose and shape control. A subject's body mesh in the rest pose is defined as
\begin{equation}
    T_P(\shapecoeff,\posecoeff, \expcoeff) = \template + B_{S}(\shapecoeff;\shapespace) +  B_{P}(\posecoeff;\posespace)
\end{equation}
where $\template \in \mathbb{R}^{\numverts \times 3}$ is a template of body shape in the rest pose, $\shapecoeff \in \shapespaceexpl$ is the body identity parameters, and $B_{S}(\shapecoeff;\shapespace): \shapespaceexpl \rightarrow \mathbb{R}^{\numverts \times 3}$ are the identity blend shapes. More specifically, $B_S(\shapecoeff;\shapespace)=\sum_{i=1}^{|\shapecoeff|}\shapecoeff_i\shapespace_i$ where $\shapecoeff_i$ is the $i$-th linear coefficient and $\shapespace_i$ is the $i$-th orthonormal principle component. $\posecoeff\in\posespaceexpl$ denotes the pose parameters. Similar to the shape space $\shapespace$, $B_{P}(\posecoeff;\posespace): \mathbb{R}^{|\posecoeff|} \rightarrow \mathbb{R}^{\numverts \times 3}$ denotes the pose blend shapes ($\posespace$ is the pose space) from the \smpl model.
To capture more geometric details, we use an upsampled version of \smpl with $\numverts=110,210$ vertices and $\numfaces=220,416$ faces.


\subsubsection{Pose-dependent Deformation} 
Given the monocular video, we need to model the movement of this subject. 
Since our initialized avatar is based on \smpl, it provides a good way to capture the pose deformation.
For each frame of given video, we estimate the \smpl parameters  $\posecoeff \in \mathbb{R}^{|\posecoeff|}$. 
Then we can deform the head/body to the observation pose using the linear blend skinning function (\ie, $\lbs$).  
The deformation for the explicit \smpl mesh model is modeled by a differential function $\vsmplx(\shapecoeff, \posecoeff, \expcoeff, \bm{O})$ that outputs a 3D human body mesh $(\bm{V},\bm{F})$ where $\bm{V}\in\mathbb{R}^{\numverts\times 3}$ is a set of $\numverts$ vertices and $\bm{F}\in\mathbb{R}^{\numfaces\times 3}$ is a set of $\numfaces$ faces with a fixed topology:
\begin{equation}
        \vsmplx(\shapecoeff, \posecoeff, \expcoeff, \bm{O}) =  \lbs(\tilde{T}_P(\shapecoeff, \posecoeff, \expcoeff, \bm{O}), \joints(\shapecoeff), \posecoeff, \bm{W}),
\end{equation}
in which $\thickmuskip=2mu \medmuskip=2mu \bm{W} \in \mathbb{R}^{\numjoints \times \numverts}$ is the blend skinning weights used in the $\lbs$ function. $\thickmuskip=2mu \medmuskip=2mu\joints(\shapecoeff)\in\mathbb{R}^{n_k\times 3}$ is a function of body shape~\cite{SMPL:2015}, representing the shape-dependent joints. Given a template vertex $\bm{t}_i$, the vertex $\bm{v}_i$ can be computed with simple linear transformation. Specifically, the forward vertex-wise deformation can be written as the following equation in the homogeneous coordinates:
\begin{equation}
\begin{aligned}
    \underbrace{\bm{v}_i}_{\textnormal{Posed vertex}} = \underbrace{\sum_{k=1}^{n_k}\bm{W}_{k,i}G_k(\bm{\theta},J(\bm{\beta}))\cdot
        \begin{bmatrix}
     \bm{I} &  \bm{o}_i+ \bm{b}_i \\
      \bm{0} &  1 
  \end{bmatrix}}_{\vsmplx_i(\bm{\beta},\bm{\theta},\bm{\psi},\bm{O})\textnormal{:~Deformation to the posed space}}\cdot\underbrace{\bm{t}_i}_{\textnormal{Template vertex}}\nonumber,
\end{aligned}
\end{equation}
% \begin{equation}
% \begin{aligned}
%     \underbrace{\bm{v}_i^h}_{\textnormal{Posed vertex}} = \underbrace{\sum_{k=1}^{n_k}\bm{W}_{k,i}G_k(\bm{\theta},J(\bm{\beta}))\cdot
%         \begin{bmatrix}
%      \bm{I} &  \bm{o}_i+ \bm{b}_i \\
%       \bm{0} &  1 
%   \end{bmatrix}}_{\vsmplx_i(\bm{\beta},\bm{\theta},\bm{\psi},\bm{O})\textnormal{:~Deformation to the posed space}}\cdot\underbrace{\bm{t}_i^h}_{\textnormal{Template vertex}}\nonumber,
% \end{aligned}
% \end{equation}
where $\vsmplx_i(\bm{\beta},\bm{\theta},\bm{\psi},\bm{O}) \in \mathbb{R}^{4\times 4}$ is the deformation function of template vertex $\bm{t}_i$.   
$\bm{W}_{k,i}$ is the $(k,i)$-th element of the blend weight matrix $\bm{W}$, $G_k(\bm{\theta},J(\bm{\beta}))\in\mathbb{R}^{4\times 4}$ is the world transformation of the $k$-th joint and $\bm{b}_i$ is the $i$-th vertex of the sum of all blend shapes $\bm{B} := B_S(\bm{\beta})+B_P(\bm{\theta})+B_E(\bm{\psi})$. We denote $\bm{V}$ as the vertex set of the posed avatar ($\bm{v}_i\in\bm{V}$). Both $\bm{v}_i$ and $\bm{t}_i$ are the homogeneous coordinates when applying this deformation function. 
% of the posed vertex $\bm{v}_i$ and the rest template vertex $\bm{t}_i$, respectively.









% from method (move to supplementary)

\paragraph{SMPL~\cite{SMPL:2015}} is a parametric body model which allows pose and shape control. 
%
A body mesh in the rest pose (\ie, T-pose) is defined as
%
\begin{equation}
    T_P(\shapecoeff,\posecoeff, \expcoeff) = \template + B_{S}(\shapecoeff;\shapespace) +  B_{P}(\posecoeff;\posespace),
\end{equation}
%
where $\template \, {\in} \, \mathbb{R}^{\numverts \times 3}$ is a template of body shape in the rest pose, 
%
$\shapecoeff \, {\in} \, \shapespaceexpl$ are the body identity parameters, 
%
and $B_{S}(\shapecoeff;\shapespace){:} \, \shapespaceexpl \, {\rightarrow} \, \mathbb{R}^{\numverts {\times} 3}$ is the identity blend shape.
%
% and $B_{P}(\posecoeff;\posespace){:} \, \mathbb{R}^{|\posecoeff|} \, {\rightarrow} \, \mathbb{R}^{\numverts {\times} 3}$ are the identity and pose blend shapes, respectively. 
%
Specifically, $B_S(\shapecoeff;\shapespace)=\sum_{i=1}^{|\shapecoeff|}\shapecoeff_i\shapespace_i$, where $\shapecoeff_i$ is the $i$-th element in $\shapecoeff$ and $\shapespace_i$ is the $i$-th orthonormal principle component of the shape space $\shapespace$. 
%
Similarly, $B_{P}(\posecoeff;\posespace){:} \, \mathbb{R}^{|\posecoeff|} \, {\rightarrow} \, \mathbb{R}^{\numverts {\times} 3}$ is the pose blend shape that is calculated by linearly combining the principle components of the pose space $\posespace$ with the pose parameters $\posecoeff \, {\in} \, \posespaceexpl$.
%
% To capture more geometric details, we use an upsampled version of \smpl with $\numverts \, {=} \, 110,210$ vertices and $\numfaces \, {=} \, 220,416$ faces.


%\paragraph{Pose-dependent Deformation.} 
%Given the monocular video, we need to model the movement of this subject. 
%Since our initialized avatar is based on \smpl, it provides a good way to capture the pose deformation.
%For each frame of given video, we estimate the \smpl parameters  $\posecoeff \in \mathbb{R}^{|\posecoeff|}$. 
%Then we can deform the head/body to the observation pose using the linear blend skinning function (\ie, $\lbs$).  
%
The deformation of the \smpl mesh model is modeled by a differentiable function $\vsmplx(\shapecoeff, \posecoeff)$ that takes the shape and pose parameters as input and outputs a 3D human body mesh $(\bm{V},\bm{F})$ with $\numverts$ vertices $\bm{V}\in\mathbb{R}^{\numverts\times 3}$  and  $\numfaces$ faces $\bm{F}\in\mathbb{R}^{\numfaces\times 3}$ with a fixed topology. 
%
It is given by
\begin{equation}
\vsmplx(\shapecoeff, \posecoeff) =  \lbs(\tilde{T}_P(\shapecoeff, \posecoeff), \joints(\shapecoeff), \posecoeff, \bm{W}),
\end{equation}
where $\bm{W} \, {\in} \, \mathbb{R}^{\numjoints {\times} \numverts}$ is the linear blend skinning weights, $\numjoints$ is the number of joints, the linear blend skinning ($\lbs$) function will be introduced below,  
$\joints(\shapecoeff)\, {\in} \, \mathbb{R}^{n_k {\times} 3}$ is a function of body shape~\cite{SMPL:2015}, representing the shape-dependent joints. Given a template vertex $\bm{t}_i$, the vertex $\bm{v}_i$ can be computed with simple linear transformation. Specifically, the forward vertex-wise deformation can be written as the following equation in the homogeneous coordinates:


\begin{equation}
\small
\begin{aligned}
    \underbrace{\bm{v}_i}_{\textnormal{Posed vertex}} = \underbrace{\sum_{k=1}^{n_k}\bm{W}_{k,i} \, G_k(\bm{\theta},J(\bm{\beta}))\cdot
        \begin{bmatrix}
     \bm{I} &  \bm{b}_i \\
      \bm{0} &  1 
  \end{bmatrix}}_{\vsmplx_i(\bm{\beta},\bm{\theta})\textnormal{:~Deformation to the posed space}}\cdot\underbrace{\bm{t}_i}_{\textnormal{Template vertex}},
\end{aligned}
\end{equation}
%
where $\vsmplx_i(\bm{\beta},\bm{\theta}) \, {\in} \, \mathbb{R}^{4 {\times} 4}$ is the deformation function of template vertex $\bm{t}_i$.   
$\bm{W}_{k,i}$ is the $(k,i)$-th element of the blend weight matrix $\bm{W}$, $G_k(\bm{\theta},J(\bm{\beta}))\in\mathbb{R}^{4\times 4}$ is the world transformation of the $k$-th joint and $\bm{b}_i$ is the $i$-th vertex of the sum of all blend shapes $\bm{B} := B_S(\bm{\beta})+B_P(\bm{\theta})+B_E(\bm{\psi})$. We denote $\bm{V}$ as the vertex set of the posed avatar ($\bm{v}_i\in\bm{V}$). Both $\bm{v}_i$ and $\bm{t}_i$ are the homogeneous coordinates when applying this deformation function. 
% of the posed vertex $\bm{v}_i$ and the rest template vertex $\bm{t}_i$, respectively.