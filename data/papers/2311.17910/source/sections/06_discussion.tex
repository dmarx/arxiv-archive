\section{Conclusion}
We have proposed \acronym, a new method for novel-view and novel-pose synthesis of a human embedded in the scene by bringing a deformable model into the Gaussian Splatting framework. The method is able to reconstruct human and scene representations from in-the-wild monocular videos containing a small number of frames (50-100). \acronym enables fast training (in 30 mins) and rendering (60 FPS), ~ $100\times$ faster than the previous methods~\cite{jiang2022neuman, guo2023vid2avatar}, while at the same time significantly improving rendering quality as measured by PSNR, SSIM and LPIPS metrics.

\paragraph{Limitations and Future Work.} HUGS is limited by the underlying shape model SMPL~\cite{SMPL:2015} and linear blend skinning that may not capture the general deformable structure of loose clothing such as dresses. In addition, HUGS is trained on in-the-wild videos that do not cover the pose-space of the human body. Future work will aim to alleviate these problems by modeling non-linear clothing deformation. In addition, the lack of data maybe alleviated by learning an appearance prior on human-poses using generative approaches such as GNARF~\cite{bergman2022gnarf} and AG3D~\cite{dong2023ag3d} or by distilling from image diffusion models~\cite{poole2022dreamfusion, lin2023magic3d}. Furthermore, our model does not account for environment lighting that may effect the composition of the human in a different scene with a different environment light which can be addressed by factoring out an illumination representation~\cite{verbin2022refnerf, ranjan2023facelit}. 

% \acronym enables modeling deformable shapes using 3D Gaussian Splatting that can be directly rendered with different views and deformations. As such, in future, our method could be extended to other deformable entities such as hands, faces, animals etc. \jg{Is "poseable" a better word than deformable? This only works with an existing poseable and blend-shapable mesh models and isn't open to general deformations.}

%We have proposed \acronym, a new method for novel-view and novel-pose synthesis of a human embedded in the scene by bringing a deformable model into the Gaussian Splatting framework. This has led to over {$\sim 300 \times $} improvement in training and rendering times over state-of-the-art methods~\cite{jiang2022neuman, guo2023vid2avatar} in neural rendering of humans with over \ar{X \% } improvement in reconstruction quality (PSNR). \acronym enables modeling deformable shapes using 3D Gaussian Splatting that can be directly rendered with different views and deformations. As such, in future, our method could be extended to other deformable entities such as hands, faces, animals etc. \jg{Is "poseable" a better word than deformable? This only works with an existing poseable and blend-shapable mesh models and isn't open to general deformations.}

%Our method has certain limitiations .. limitations due to SMPL body model, limitations due to data ... 

