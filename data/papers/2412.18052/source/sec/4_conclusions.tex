\section{Conclusions}
\label{sec:conclusions}

In this work, we introduced Gradient Agreement Filtering (GAF) as an alternative to traditional micro-gradient averaging in distributed training. Our experiments on CIFAR-100 and CIFAR-100N-Fine demonstrate the effectiveness of GAF, particularly in scenarios with label noise. By aggregating gradients based on cosine distance, GAF provides a robust approach that improves model performance. Specifically, we observe a 0.2\% improvement on CIFAR-100 without added noise, with progressively larger improvements over baseline training methods as label noise increases, reaching up to an 18.4\% gain at a 60\% noise rate. On the CIFAR-100N-Fine dataset, GAF achieves a 9.3\% improvement over the baseline. We also observe that we are able to maintain the performance improvement even as the microbatch size was reduced, suggesting that we can improve model performance while reducing computational costs.

These results indicate that GAF is a promising approach for improving training stability and accuracy, particularly in noisy environments. The use of cosine distance to filter gradients shows potential not only in mitigating the impact of label noise but also in reducing the computational cost of large-scale distributed training by focusing resources on more aligned gradients.

\section{Future Research Directions}
\label{sec:future_work}

While GAF has demonstrated promising results, several avenues for further research could expand upon its potential and applicability:

\begin{itemize}
    \item \textbf{Alternative Similarity Metrics}: While cosine distance proved effective, other similarity metrics, such as Mahalanobis distance, could be explored to evaluate their impact on GAF’s performance. This could help in tailoring GAF to different types of datasets and noise structures.

    \item \textbf{Adaptive Thresholding}: In this work, we used a fixed cosine distance threshold throughout training. An adaptive threshold that dynamically adjusts based on training progress or model convergence rates may yield improved results, especially in tasks with fluctuating noise levels or diverse data distributions.

    % Commented this one out to hit the page limit of 8 pages
    % \item \textbf{Adaptive Learning Rates}: As we have shown in this paper, disagreement in gradient direction from noisy data signals a lack of confidence in the step. Perhaps it would be appropriate to reduce the step size as a function of the gradient noise across the micro-gradients. For example, if the average micro-gradient cosine distance is small, then we have a lot of confidence in our step and perhaps should increase the step size for this iteration. Alternatively, if the average micro-gradient cosine distance is large, then we have very little confidence in our step and perhaps should decrease the step size for this iteration. 

    \item \textbf{Application to Other Tasks}: GAF was applied to image classification in this study. Extending this technique to other domains, such as natural language processing, speech recognition, or reinforcement learning, could uncover broader benefits and challenges associated with GAF in non-vision tasks. 

    \item \textbf{Memory and Computation Efficiency}: As GAF requires tracking only pairwise cosine distances between micro-gradients, applying this to Ring-AllReduce would be straightforward but would require applying cosine distance to buckets at a time. Ensuring GAF's improvement is maintained despite this is an area of future research, as well as other avenues to optimize compute and memory overhead.
    
    \item \textbf{Order Indifference Techniques}: As GAF is sensitive to the order in which microgradients are processed, perhaps there is a way to augment Ring-AllReduce where during the AllGather phase, the GPU with the highest (or lowest) agreement is the one distributed to all other nodes. 

    \item \textbf{Integration with Advanced Optimizers}: We used standard optimizers like SGD and Adam in our experiments. Investigating how GAF interacts with other advanced optimization techniques, such as Adam, AdamW, LAMB or SHAMPOO, could enhance GAF’s performance, particularly in large-scale or fine-tuning scenarios.

    \item \textbf{Analysis of Gradient Disagreement Dynamics}: Further study of the dynamics of gradient disagreement over the course of training could yield insights into how models converge under noisy conditions and how GAF influences the loss landscape. This might lead to improvements in convergence rates and generalization.

\end{itemize}

Further research in these directions highlight potential improvements and adaptations of GAF, aiming to make it more efficient, robust, and applicable across various deep learning domains. %We hope that our work encourages further exploration of selective gradient aggregation techniques and contributes to advances in distributed deep learning.

